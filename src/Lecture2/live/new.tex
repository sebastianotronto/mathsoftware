\documentclass[10pt,a5paper]{article}
\usepackage[utf8]{inputenc}
\usepackage{amsmath}
\usepackage{amsfonts}
\usepackage{amssymb}
\usepackage{amsthm}
\usepackage[left=2cm,right=2cm,top=2cm,bottom=2cm]{geometry}

\usepackage{enumitem}

\title{Latex Example Live}
\author{Sebastiano Tronto}
\date{12-03-2021}



\newtheorem{mythm}{My Theorem}[section]

\theoremstyle{definition}
\newtheorem{prop}[mythm]{Proposition}
\newtheorem{defi}{Definition}

\theoremstyle{remark}
\newtheorem*{warning}{Achtung}

\renewcommand{\thesection}{\S\arabic{section}}
\renewcommand{\themythm}{\arabic{section},\Alph{mythm}}


\numberwithin{equation}{section}

\renewcommand{\theequation}{\arabic{section}.\arabic{equation}}

\begin{document}

\maketitle


\section{Text}

abc

\begin{align}
1+1=2
\end{align}

The section \ref{ts} contains theorems

\section{inbetween}

\section{Last section}
\label{ts}


\begin{prop}
A less important fact
\end{prop}

\begin{mythm}[Gauss]
\label{gaussthm}
The equation \(2+x=4\) is true for \(x=2\).
\begin{align}
\label{eq}
\sum_{i=1}^ni
\end{align}
\end{mythm}

The above equation \eqref{eq} is not an equation

\arabic{equation}


\setcounter{defi}{23}
\begin{defi}
a definition
\end{defi}

\setcounter{mythm}{10}
\begin{mythm}
Another important fact.
\end{mythm}

\begin{warning}
It is a common mistake to think that \(2+2=5\)
\end{warning}


\vspace{1cm}

\begin{enumerate}[label=(\Roman*)]
	\item an item
	\item another one
\end{enumerate}

\arabic{enumi}

\ref{gaussthm} is a theorem by gauss

\end{document}