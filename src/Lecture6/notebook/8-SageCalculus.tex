\documentclass[11pt]{article}

    \usepackage[breakable]{tcolorbox}
    \usepackage{parskip} % Stop auto-indenting (to mimic markdown behaviour)
    
    \usepackage{iftex}
    \ifPDFTeX
    	\usepackage[T1]{fontenc}
    	\usepackage{mathpazo}
    \else
    	\usepackage{fontspec}
    \fi

    % Basic figure setup, for now with no caption control since it's done
    % automatically by Pandoc (which extracts ![](path) syntax from Markdown).
    \usepackage{graphicx}
    % Maintain compatibility with old templates. Remove in nbconvert 6.0
    \let\Oldincludegraphics\includegraphics
    % Ensure that by default, figures have no caption (until we provide a
    % proper Figure object with a Caption API and a way to capture that
    % in the conversion process - todo).
    \usepackage{caption}
    \DeclareCaptionFormat{nocaption}{}
    \captionsetup{format=nocaption,aboveskip=0pt,belowskip=0pt}

    \usepackage[Export]{adjustbox} % Used to constrain images to a maximum size
    \adjustboxset{max size={0.9\linewidth}{0.9\paperheight}}
    \usepackage{float}
    \floatplacement{figure}{H} % forces figures to be placed at the correct location
    \usepackage{xcolor} % Allow colors to be defined
    \usepackage{enumerate} % Needed for markdown enumerations to work
    \usepackage{geometry} % Used to adjust the document margins
    \usepackage{amsmath} % Equations
    \usepackage{amssymb} % Equations
    \usepackage{textcomp} % defines textquotesingle
    % Hack from http://tex.stackexchange.com/a/47451/13684:
    \AtBeginDocument{%
        \def\PYZsq{\textquotesingle}% Upright quotes in Pygmentized code
    }
    \usepackage{upquote} % Upright quotes for verbatim code
    \usepackage{eurosym} % defines \euro
    \usepackage[mathletters]{ucs} % Extended unicode (utf-8) support
    \usepackage{fancyvrb} % verbatim replacement that allows latex
    \usepackage{grffile} % extends the file name processing of package graphics 
                         % to support a larger range
    \makeatletter % fix for grffile with XeLaTeX
    \def\Gread@@xetex#1{%
      \IfFileExists{"\Gin@base".bb}%
      {\Gread@eps{\Gin@base.bb}}%
      {\Gread@@xetex@aux#1}%
    }
    \makeatother

    % The hyperref package gives us a pdf with properly built
    % internal navigation ('pdf bookmarks' for the table of contents,
    % internal cross-reference links, web links for URLs, etc.)
    \usepackage{hyperref}
    % The default LaTeX title has an obnoxious amount of whitespace. By default,
    % titling removes some of it. It also provides customization options.
    \usepackage{titling}
    \usepackage{longtable} % longtable support required by pandoc >1.10
    \usepackage{booktabs}  % table support for pandoc > 1.12.2
    \usepackage[inline]{enumitem} % IRkernel/repr support (it uses the enumerate* environment)
    \usepackage[normalem]{ulem} % ulem is needed to support strikethroughs (\sout)
                                % normalem makes italics be italics, not underlines
    \usepackage{mathrsfs}
    

    
    % Colors for the hyperref package
    \definecolor{urlcolor}{rgb}{0,.145,.698}
    \definecolor{linkcolor}{rgb}{.71,0.21,0.01}
    \definecolor{citecolor}{rgb}{.12,.54,.11}

    % ANSI colors
    \definecolor{ansi-black}{HTML}{3E424D}
    \definecolor{ansi-black-intense}{HTML}{282C36}
    \definecolor{ansi-red}{HTML}{E75C58}
    \definecolor{ansi-red-intense}{HTML}{B22B31}
    \definecolor{ansi-green}{HTML}{00A250}
    \definecolor{ansi-green-intense}{HTML}{007427}
    \definecolor{ansi-yellow}{HTML}{DDB62B}
    \definecolor{ansi-yellow-intense}{HTML}{B27D12}
    \definecolor{ansi-blue}{HTML}{208FFB}
    \definecolor{ansi-blue-intense}{HTML}{0065CA}
    \definecolor{ansi-magenta}{HTML}{D160C4}
    \definecolor{ansi-magenta-intense}{HTML}{A03196}
    \definecolor{ansi-cyan}{HTML}{60C6C8}
    \definecolor{ansi-cyan-intense}{HTML}{258F8F}
    \definecolor{ansi-white}{HTML}{C5C1B4}
    \definecolor{ansi-white-intense}{HTML}{A1A6B2}
    \definecolor{ansi-default-inverse-fg}{HTML}{FFFFFF}
    \definecolor{ansi-default-inverse-bg}{HTML}{000000}

    % commands and environments needed by pandoc snippets
    % extracted from the output of `pandoc -s`
    \providecommand{\tightlist}{%
      \setlength{\itemsep}{0pt}\setlength{\parskip}{0pt}}
    \DefineVerbatimEnvironment{Highlighting}{Verbatim}{commandchars=\\\{\}}
    % Add ',fontsize=\small' for more characters per line
    \newenvironment{Shaded}{}{}
    \newcommand{\KeywordTok}[1]{\textcolor[rgb]{0.00,0.44,0.13}{\textbf{{#1}}}}
    \newcommand{\DataTypeTok}[1]{\textcolor[rgb]{0.56,0.13,0.00}{{#1}}}
    \newcommand{\DecValTok}[1]{\textcolor[rgb]{0.25,0.63,0.44}{{#1}}}
    \newcommand{\BaseNTok}[1]{\textcolor[rgb]{0.25,0.63,0.44}{{#1}}}
    \newcommand{\FloatTok}[1]{\textcolor[rgb]{0.25,0.63,0.44}{{#1}}}
    \newcommand{\CharTok}[1]{\textcolor[rgb]{0.25,0.44,0.63}{{#1}}}
    \newcommand{\StringTok}[1]{\textcolor[rgb]{0.25,0.44,0.63}{{#1}}}
    \newcommand{\CommentTok}[1]{\textcolor[rgb]{0.38,0.63,0.69}{\textit{{#1}}}}
    \newcommand{\OtherTok}[1]{\textcolor[rgb]{0.00,0.44,0.13}{{#1}}}
    \newcommand{\AlertTok}[1]{\textcolor[rgb]{1.00,0.00,0.00}{\textbf{{#1}}}}
    \newcommand{\FunctionTok}[1]{\textcolor[rgb]{0.02,0.16,0.49}{{#1}}}
    \newcommand{\RegionMarkerTok}[1]{{#1}}
    \newcommand{\ErrorTok}[1]{\textcolor[rgb]{1.00,0.00,0.00}{\textbf{{#1}}}}
    \newcommand{\NormalTok}[1]{{#1}}
    
    % Additional commands for more recent versions of Pandoc
    \newcommand{\ConstantTok}[1]{\textcolor[rgb]{0.53,0.00,0.00}{{#1}}}
    \newcommand{\SpecialCharTok}[1]{\textcolor[rgb]{0.25,0.44,0.63}{{#1}}}
    \newcommand{\VerbatimStringTok}[1]{\textcolor[rgb]{0.25,0.44,0.63}{{#1}}}
    \newcommand{\SpecialStringTok}[1]{\textcolor[rgb]{0.73,0.40,0.53}{{#1}}}
    \newcommand{\ImportTok}[1]{{#1}}
    \newcommand{\DocumentationTok}[1]{\textcolor[rgb]{0.73,0.13,0.13}{\textit{{#1}}}}
    \newcommand{\AnnotationTok}[1]{\textcolor[rgb]{0.38,0.63,0.69}{\textbf{\textit{{#1}}}}}
    \newcommand{\CommentVarTok}[1]{\textcolor[rgb]{0.38,0.63,0.69}{\textbf{\textit{{#1}}}}}
    \newcommand{\VariableTok}[1]{\textcolor[rgb]{0.10,0.09,0.49}{{#1}}}
    \newcommand{\ControlFlowTok}[1]{\textcolor[rgb]{0.00,0.44,0.13}{\textbf{{#1}}}}
    \newcommand{\OperatorTok}[1]{\textcolor[rgb]{0.40,0.40,0.40}{{#1}}}
    \newcommand{\BuiltInTok}[1]{{#1}}
    \newcommand{\ExtensionTok}[1]{{#1}}
    \newcommand{\PreprocessorTok}[1]{\textcolor[rgb]{0.74,0.48,0.00}{{#1}}}
    \newcommand{\AttributeTok}[1]{\textcolor[rgb]{0.49,0.56,0.16}{{#1}}}
    \newcommand{\InformationTok}[1]{\textcolor[rgb]{0.38,0.63,0.69}{\textbf{\textit{{#1}}}}}
    \newcommand{\WarningTok}[1]{\textcolor[rgb]{0.38,0.63,0.69}{\textbf{\textit{{#1}}}}}
    
    
    % Define a nice break command that doesn't care if a line doesn't already
    % exist.
    \def\br{\hspace*{\fill} \\* }
    % Math Jax compatibility definitions
    \def\gt{>}
    \def\lt{<}
    \let\Oldtex\TeX
    \let\Oldlatex\LaTeX
    \renewcommand{\TeX}{\textrm{\Oldtex}}
    \renewcommand{\LaTeX}{\textrm{\Oldlatex}}
    % Document parameters
    % Document title
    \title{Calculus and more with SageMath}
    \author{Sebastiano Tronto - \texttt{sebastiano.tronto@uni.lu}}
    \date{2021-05-07}
    
    
    
    
    
% Pygments definitions
\makeatletter
\def\PY@reset{\let\PY@it=\relax \let\PY@bf=\relax%
    \let\PY@ul=\relax \let\PY@tc=\relax%
    \let\PY@bc=\relax \let\PY@ff=\relax}
\def\PY@tok#1{\csname PY@tok@#1\endcsname}
\def\PY@toks#1+{\ifx\relax#1\empty\else%
    \PY@tok{#1}\expandafter\PY@toks\fi}
\def\PY@do#1{\PY@bc{\PY@tc{\PY@ul{%
    \PY@it{\PY@bf{\PY@ff{#1}}}}}}}
\def\PY#1#2{\PY@reset\PY@toks#1+\relax+\PY@do{#2}}

\expandafter\def\csname PY@tok@w\endcsname{\def\PY@tc##1{\textcolor[rgb]{0.73,0.73,0.73}{##1}}}
\expandafter\def\csname PY@tok@c\endcsname{\let\PY@it=\textit\def\PY@tc##1{\textcolor[rgb]{0.25,0.50,0.50}{##1}}}
\expandafter\def\csname PY@tok@cp\endcsname{\def\PY@tc##1{\textcolor[rgb]{0.74,0.48,0.00}{##1}}}
\expandafter\def\csname PY@tok@k\endcsname{\let\PY@bf=\textbf\def\PY@tc##1{\textcolor[rgb]{0.00,0.50,0.00}{##1}}}
\expandafter\def\csname PY@tok@kp\endcsname{\def\PY@tc##1{\textcolor[rgb]{0.00,0.50,0.00}{##1}}}
\expandafter\def\csname PY@tok@kt\endcsname{\def\PY@tc##1{\textcolor[rgb]{0.69,0.00,0.25}{##1}}}
\expandafter\def\csname PY@tok@o\endcsname{\def\PY@tc##1{\textcolor[rgb]{0.40,0.40,0.40}{##1}}}
\expandafter\def\csname PY@tok@ow\endcsname{\let\PY@bf=\textbf\def\PY@tc##1{\textcolor[rgb]{0.67,0.13,1.00}{##1}}}
\expandafter\def\csname PY@tok@nb\endcsname{\def\PY@tc##1{\textcolor[rgb]{0.00,0.50,0.00}{##1}}}
\expandafter\def\csname PY@tok@nf\endcsname{\def\PY@tc##1{\textcolor[rgb]{0.00,0.00,1.00}{##1}}}
\expandafter\def\csname PY@tok@nc\endcsname{\let\PY@bf=\textbf\def\PY@tc##1{\textcolor[rgb]{0.00,0.00,1.00}{##1}}}
\expandafter\def\csname PY@tok@nn\endcsname{\let\PY@bf=\textbf\def\PY@tc##1{\textcolor[rgb]{0.00,0.00,1.00}{##1}}}
\expandafter\def\csname PY@tok@ne\endcsname{\let\PY@bf=\textbf\def\PY@tc##1{\textcolor[rgb]{0.82,0.25,0.23}{##1}}}
\expandafter\def\csname PY@tok@nv\endcsname{\def\PY@tc##1{\textcolor[rgb]{0.10,0.09,0.49}{##1}}}
\expandafter\def\csname PY@tok@no\endcsname{\def\PY@tc##1{\textcolor[rgb]{0.53,0.00,0.00}{##1}}}
\expandafter\def\csname PY@tok@nl\endcsname{\def\PY@tc##1{\textcolor[rgb]{0.63,0.63,0.00}{##1}}}
\expandafter\def\csname PY@tok@ni\endcsname{\let\PY@bf=\textbf\def\PY@tc##1{\textcolor[rgb]{0.60,0.60,0.60}{##1}}}
\expandafter\def\csname PY@tok@na\endcsname{\def\PY@tc##1{\textcolor[rgb]{0.49,0.56,0.16}{##1}}}
\expandafter\def\csname PY@tok@nt\endcsname{\let\PY@bf=\textbf\def\PY@tc##1{\textcolor[rgb]{0.00,0.50,0.00}{##1}}}
\expandafter\def\csname PY@tok@nd\endcsname{\def\PY@tc##1{\textcolor[rgb]{0.67,0.13,1.00}{##1}}}
\expandafter\def\csname PY@tok@s\endcsname{\def\PY@tc##1{\textcolor[rgb]{0.73,0.13,0.13}{##1}}}
\expandafter\def\csname PY@tok@sd\endcsname{\let\PY@it=\textit\def\PY@tc##1{\textcolor[rgb]{0.73,0.13,0.13}{##1}}}
\expandafter\def\csname PY@tok@si\endcsname{\let\PY@bf=\textbf\def\PY@tc##1{\textcolor[rgb]{0.73,0.40,0.53}{##1}}}
\expandafter\def\csname PY@tok@se\endcsname{\let\PY@bf=\textbf\def\PY@tc##1{\textcolor[rgb]{0.73,0.40,0.13}{##1}}}
\expandafter\def\csname PY@tok@sr\endcsname{\def\PY@tc##1{\textcolor[rgb]{0.73,0.40,0.53}{##1}}}
\expandafter\def\csname PY@tok@ss\endcsname{\def\PY@tc##1{\textcolor[rgb]{0.10,0.09,0.49}{##1}}}
\expandafter\def\csname PY@tok@sx\endcsname{\def\PY@tc##1{\textcolor[rgb]{0.00,0.50,0.00}{##1}}}
\expandafter\def\csname PY@tok@m\endcsname{\def\PY@tc##1{\textcolor[rgb]{0.40,0.40,0.40}{##1}}}
\expandafter\def\csname PY@tok@gh\endcsname{\let\PY@bf=\textbf\def\PY@tc##1{\textcolor[rgb]{0.00,0.00,0.50}{##1}}}
\expandafter\def\csname PY@tok@gu\endcsname{\let\PY@bf=\textbf\def\PY@tc##1{\textcolor[rgb]{0.50,0.00,0.50}{##1}}}
\expandafter\def\csname PY@tok@gd\endcsname{\def\PY@tc##1{\textcolor[rgb]{0.63,0.00,0.00}{##1}}}
\expandafter\def\csname PY@tok@gi\endcsname{\def\PY@tc##1{\textcolor[rgb]{0.00,0.63,0.00}{##1}}}
\expandafter\def\csname PY@tok@gr\endcsname{\def\PY@tc##1{\textcolor[rgb]{1.00,0.00,0.00}{##1}}}
\expandafter\def\csname PY@tok@ge\endcsname{\let\PY@it=\textit}
\expandafter\def\csname PY@tok@gs\endcsname{\let\PY@bf=\textbf}
\expandafter\def\csname PY@tok@gp\endcsname{\let\PY@bf=\textbf\def\PY@tc##1{\textcolor[rgb]{0.00,0.00,0.50}{##1}}}
\expandafter\def\csname PY@tok@go\endcsname{\def\PY@tc##1{\textcolor[rgb]{0.53,0.53,0.53}{##1}}}
\expandafter\def\csname PY@tok@gt\endcsname{\def\PY@tc##1{\textcolor[rgb]{0.00,0.27,0.87}{##1}}}
\expandafter\def\csname PY@tok@err\endcsname{\def\PY@bc##1{\setlength{\fboxsep}{0pt}\fcolorbox[rgb]{1.00,0.00,0.00}{1,1,1}{\strut ##1}}}
\expandafter\def\csname PY@tok@kc\endcsname{\let\PY@bf=\textbf\def\PY@tc##1{\textcolor[rgb]{0.00,0.50,0.00}{##1}}}
\expandafter\def\csname PY@tok@kd\endcsname{\let\PY@bf=\textbf\def\PY@tc##1{\textcolor[rgb]{0.00,0.50,0.00}{##1}}}
\expandafter\def\csname PY@tok@kn\endcsname{\let\PY@bf=\textbf\def\PY@tc##1{\textcolor[rgb]{0.00,0.50,0.00}{##1}}}
\expandafter\def\csname PY@tok@kr\endcsname{\let\PY@bf=\textbf\def\PY@tc##1{\textcolor[rgb]{0.00,0.50,0.00}{##1}}}
\expandafter\def\csname PY@tok@bp\endcsname{\def\PY@tc##1{\textcolor[rgb]{0.00,0.50,0.00}{##1}}}
\expandafter\def\csname PY@tok@fm\endcsname{\def\PY@tc##1{\textcolor[rgb]{0.00,0.00,1.00}{##1}}}
\expandafter\def\csname PY@tok@vc\endcsname{\def\PY@tc##1{\textcolor[rgb]{0.10,0.09,0.49}{##1}}}
\expandafter\def\csname PY@tok@vg\endcsname{\def\PY@tc##1{\textcolor[rgb]{0.10,0.09,0.49}{##1}}}
\expandafter\def\csname PY@tok@vi\endcsname{\def\PY@tc##1{\textcolor[rgb]{0.10,0.09,0.49}{##1}}}
\expandafter\def\csname PY@tok@vm\endcsname{\def\PY@tc##1{\textcolor[rgb]{0.10,0.09,0.49}{##1}}}
\expandafter\def\csname PY@tok@sa\endcsname{\def\PY@tc##1{\textcolor[rgb]{0.73,0.13,0.13}{##1}}}
\expandafter\def\csname PY@tok@sb\endcsname{\def\PY@tc##1{\textcolor[rgb]{0.73,0.13,0.13}{##1}}}
\expandafter\def\csname PY@tok@sc\endcsname{\def\PY@tc##1{\textcolor[rgb]{0.73,0.13,0.13}{##1}}}
\expandafter\def\csname PY@tok@dl\endcsname{\def\PY@tc##1{\textcolor[rgb]{0.73,0.13,0.13}{##1}}}
\expandafter\def\csname PY@tok@s2\endcsname{\def\PY@tc##1{\textcolor[rgb]{0.73,0.13,0.13}{##1}}}
\expandafter\def\csname PY@tok@sh\endcsname{\def\PY@tc##1{\textcolor[rgb]{0.73,0.13,0.13}{##1}}}
\expandafter\def\csname PY@tok@s1\endcsname{\def\PY@tc##1{\textcolor[rgb]{0.73,0.13,0.13}{##1}}}
\expandafter\def\csname PY@tok@mb\endcsname{\def\PY@tc##1{\textcolor[rgb]{0.40,0.40,0.40}{##1}}}
\expandafter\def\csname PY@tok@mf\endcsname{\def\PY@tc##1{\textcolor[rgb]{0.40,0.40,0.40}{##1}}}
\expandafter\def\csname PY@tok@mh\endcsname{\def\PY@tc##1{\textcolor[rgb]{0.40,0.40,0.40}{##1}}}
\expandafter\def\csname PY@tok@mi\endcsname{\def\PY@tc##1{\textcolor[rgb]{0.40,0.40,0.40}{##1}}}
\expandafter\def\csname PY@tok@il\endcsname{\def\PY@tc##1{\textcolor[rgb]{0.40,0.40,0.40}{##1}}}
\expandafter\def\csname PY@tok@mo\endcsname{\def\PY@tc##1{\textcolor[rgb]{0.40,0.40,0.40}{##1}}}
\expandafter\def\csname PY@tok@ch\endcsname{\let\PY@it=\textit\def\PY@tc##1{\textcolor[rgb]{0.25,0.50,0.50}{##1}}}
\expandafter\def\csname PY@tok@cm\endcsname{\let\PY@it=\textit\def\PY@tc##1{\textcolor[rgb]{0.25,0.50,0.50}{##1}}}
\expandafter\def\csname PY@tok@cpf\endcsname{\let\PY@it=\textit\def\PY@tc##1{\textcolor[rgb]{0.25,0.50,0.50}{##1}}}
\expandafter\def\csname PY@tok@c1\endcsname{\let\PY@it=\textit\def\PY@tc##1{\textcolor[rgb]{0.25,0.50,0.50}{##1}}}
\expandafter\def\csname PY@tok@cs\endcsname{\let\PY@it=\textit\def\PY@tc##1{\textcolor[rgb]{0.25,0.50,0.50}{##1}}}

\def\PYZbs{\char`\\}
\def\PYZus{\char`\_}
\def\PYZob{\char`\{}
\def\PYZcb{\char`\}}
\def\PYZca{\char`\^}
\def\PYZam{\char`\&}
\def\PYZlt{\char`\<}
\def\PYZgt{\char`\>}
\def\PYZsh{\char`\#}
\def\PYZpc{\char`\%}
\def\PYZdl{\char`\$}
\def\PYZhy{\char`\-}
\def\PYZsq{\char`\'}
\def\PYZdq{\char`\"}
\def\PYZti{\char`\~}
% for compatibility with earlier versions
\def\PYZat{@}
\def\PYZlb{[}
\def\PYZrb{]}
\makeatother


    % For linebreaks inside Verbatim environment from package fancyvrb. 
    \makeatletter
        \newbox\Wrappedcontinuationbox 
        \newbox\Wrappedvisiblespacebox 
        \newcommand*\Wrappedvisiblespace {\textcolor{red}{\textvisiblespace}} 
        \newcommand*\Wrappedcontinuationsymbol {\textcolor{red}{\llap{\tiny$\m@th\hookrightarrow$}}} 
        \newcommand*\Wrappedcontinuationindent {3ex } 
        \newcommand*\Wrappedafterbreak {\kern\Wrappedcontinuationindent\copy\Wrappedcontinuationbox} 
        % Take advantage of the already applied Pygments mark-up to insert 
        % potential linebreaks for TeX processing. 
        %        {, <, #, %, $, ' and ": go to next line. 
        %        _, }, ^, &, >, - and ~: stay at end of broken line. 
        % Use of \textquotesingle for straight quote. 
        \newcommand*\Wrappedbreaksatspecials {% 
            \def\PYGZus{\discretionary{\char`\_}{\Wrappedafterbreak}{\char`\_}}% 
            \def\PYGZob{\discretionary{}{\Wrappedafterbreak\char`\{}{\char`\{}}% 
            \def\PYGZcb{\discretionary{\char`\}}{\Wrappedafterbreak}{\char`\}}}% 
            \def\PYGZca{\discretionary{\char`\^}{\Wrappedafterbreak}{\char`\^}}% 
            \def\PYGZam{\discretionary{\char`\&}{\Wrappedafterbreak}{\char`\&}}% 
            \def\PYGZlt{\discretionary{}{\Wrappedafterbreak\char`\<}{\char`\<}}% 
            \def\PYGZgt{\discretionary{\char`\>}{\Wrappedafterbreak}{\char`\>}}% 
            \def\PYGZsh{\discretionary{}{\Wrappedafterbreak\char`\#}{\char`\#}}% 
            \def\PYGZpc{\discretionary{}{\Wrappedafterbreak\char`\%}{\char`\%}}% 
            \def\PYGZdl{\discretionary{}{\Wrappedafterbreak\char`\$}{\char`\$}}% 
            \def\PYGZhy{\discretionary{\char`\-}{\Wrappedafterbreak}{\char`\-}}% 
            \def\PYGZsq{\discretionary{}{\Wrappedafterbreak\textquotesingle}{\textquotesingle}}% 
            \def\PYGZdq{\discretionary{}{\Wrappedafterbreak\char`\"}{\char`\"}}% 
            \def\PYGZti{\discretionary{\char`\~}{\Wrappedafterbreak}{\char`\~}}% 
        } 
        % Some characters . , ; ? ! / are not pygmentized. 
        % This macro makes them "active" and they will insert potential linebreaks 
        \newcommand*\Wrappedbreaksatpunct {% 
            \lccode`\~`\.\lowercase{\def~}{\discretionary{\hbox{\char`\.}}{\Wrappedafterbreak}{\hbox{\char`\.}}}% 
            \lccode`\~`\,\lowercase{\def~}{\discretionary{\hbox{\char`\,}}{\Wrappedafterbreak}{\hbox{\char`\,}}}% 
            \lccode`\~`\;\lowercase{\def~}{\discretionary{\hbox{\char`\;}}{\Wrappedafterbreak}{\hbox{\char`\;}}}% 
            \lccode`\~`\:\lowercase{\def~}{\discretionary{\hbox{\char`\:}}{\Wrappedafterbreak}{\hbox{\char`\:}}}% 
            \lccode`\~`\?\lowercase{\def~}{\discretionary{\hbox{\char`\?}}{\Wrappedafterbreak}{\hbox{\char`\?}}}% 
            \lccode`\~`\!\lowercase{\def~}{\discretionary{\hbox{\char`\!}}{\Wrappedafterbreak}{\hbox{\char`\!}}}% 
            \lccode`\~`\/\lowercase{\def~}{\discretionary{\hbox{\char`\/}}{\Wrappedafterbreak}{\hbox{\char`\/}}}% 
            \catcode`\.\active
            \catcode`\,\active 
            \catcode`\;\active
            \catcode`\:\active
            \catcode`\?\active
            \catcode`\!\active
            \catcode`\/\active 
            \lccode`\~`\~ 	
        }
    \makeatother

    \let\OriginalVerbatim=\Verbatim
    \makeatletter
    \renewcommand{\Verbatim}[1][1]{%
        %\parskip\z@skip
        \sbox\Wrappedcontinuationbox {\Wrappedcontinuationsymbol}%
        \sbox\Wrappedvisiblespacebox {\FV@SetupFont\Wrappedvisiblespace}%
        \def\FancyVerbFormatLine ##1{\hsize\linewidth
            \vtop{\raggedright\hyphenpenalty\z@\exhyphenpenalty\z@
                \doublehyphendemerits\z@\finalhyphendemerits\z@
                \strut ##1\strut}%
        }%
        % If the linebreak is at a space, the latter will be displayed as visible
        % space at end of first line, and a continuation symbol starts next line.
        % Stretch/shrink are however usually zero for typewriter font.
        \def\FV@Space {%
            \nobreak\hskip\z@ plus\fontdimen3\font minus\fontdimen4\font
            \discretionary{\copy\Wrappedvisiblespacebox}{\Wrappedafterbreak}
            {\kern\fontdimen2\font}%
        }%
        
        % Allow breaks at special characters using \PYG... macros.
        \Wrappedbreaksatspecials
        % Breaks at punctuation characters . , ; ? ! and / need catcode=\active 	
        \OriginalVerbatim[#1,codes*=\Wrappedbreaksatpunct]%
    }
    \makeatother

    % Exact colors from NB
    \definecolor{incolor}{HTML}{303F9F}
    \definecolor{outcolor}{HTML}{D84315}
    \definecolor{cellborder}{HTML}{CFCFCF}
    \definecolor{cellbackground}{HTML}{F7F7F7}
    
    % prompt
    \makeatletter
    \newcommand{\boxspacing}{\kern\kvtcb@left@rule\kern\kvtcb@boxsep}
    \makeatother
    \newcommand{\prompt}[4]{
        \ttfamily\llap{{\color{#2}[#3]:\hspace{3pt}#4}}\vspace{-\baselineskip}
    }
    

    
    % Prevent overflowing lines due to hard-to-break entities
    \sloppy 
    % Setup hyperref package
    \hypersetup{
      breaklinks=true,  % so long urls are correctly broken across lines
      colorlinks=true,
      urlcolor=urlcolor,
      linkcolor=linkcolor,
      citecolor=citecolor,
      }
    % Slightly bigger margins than the latex defaults
    
    \geometry{verbose,tmargin=1in,bmargin=1in,lmargin=1in,rmargin=1in}
    
    

\begin{document}
    
    \maketitle
    
    

    
    \hypertarget{symbolic-expressions}{%
\section{Symbolic expressions}\label{symbolic-expressions}}

\textbf{Reference:}
{[}\href{https://doc.sagemath.org/html/en/reference/calculus/sage/symbolic/expression.html}{1}{]}

Last time we saw the basics of symbolic expressions: * How to define and
manipulate symbolic expressions * How to introduce new variables (in the
Mathematical sense) with \texttt{var()} * How to solve equations and
inequalities * Some of the Mathematical constants that are included in
Sage, and how to approximate them using \texttt{n()}

Here are some examples to remind you of these basic things:

    \begin{tcolorbox}[breakable, size=fbox, boxrule=1pt, pad at break*=1mm,colback=cellbackground, colframe=cellborder]
\prompt{In}{incolor}{2}{\boxspacing}
\begin{Verbatim}[commandchars=\\\{\}]
\PY{n}{var}\PY{p}{(}\PY{l+s+s1}{\PYZsq{}}\PY{l+s+s1}{y}\PY{l+s+s1}{\PYZsq{}}\PY{p}{,} \PY{l+s+s1}{\PYZsq{}}\PY{l+s+s1}{z}\PY{l+s+s1}{\PYZsq{}}\PY{p}{)} \PY{c+c1}{\PYZsh{} Define new variables (x is already defined by Sage)}
\PY{n}{f} \PY{o}{=} \PY{n}{x}\PY{o}{\PYZca{}}\PY{l+m+mi}{2} \PY{o}{+} \PY{n}{pi}
\PY{n}{g} \PY{o}{=} \PY{n}{y}\PY{o}{\PYZca{}}\PY{l+m+mi}{2} \PY{o}{+} \PY{n}{y} \PY{o}{\PYZhy{}} \PY{l+m+mi}{2} \PY{o}{\PYZgt{}} \PY{l+m+mi}{0}
\PY{n+nb}{print}\PY{p}{(} \PY{n}{solve}\PY{p}{(}\PY{n}{f}\PY{o}{==}\PY{l+m+mi}{0}\PY{p}{,} \PY{n}{x}\PY{p}{)} \PY{p}{)}
\PY{n+nb}{print}\PY{p}{(} \PY{n}{solve}\PY{p}{(}\PY{n}{z}\PY{o}{\PYZca{}}\PY{l+m+mi}{2} \PY{o}{\PYZhy{}} \PY{n}{f}\PY{p}{,} \PY{n}{z}\PY{p}{)} \PY{p}{)}
\PY{n+nb}{print}\PY{p}{(} \PY{n}{solve}\PY{p}{(}\PY{n}{g}\PY{p}{,} \PY{n}{y}\PY{p}{)} \PY{p}{)}
\PY{n+nb}{print}\PY{p}{(} \PY{l+m+mi}{2}\PY{o}{*}\PY{n}{pi} \PY{o}{+} \PY{n}{e}\PY{p}{,} \PY{l+s+s2}{\PYZdq{}}\PY{l+s+s2}{is approximately}\PY{l+s+s2}{\PYZdq{}}\PY{p}{,} \PY{n}{n}\PY{p}{(}\PY{l+m+mi}{2}\PY{o}{*}\PY{n}{pi} \PY{o}{+} \PY{n}{e}\PY{p}{)} \PY{p}{)}
\end{Verbatim}
\end{tcolorbox}

    \begin{Verbatim}[commandchars=\\\{\}]
[
x == -sqrt(-pi),
x == sqrt(-pi)
]
[
z == -sqrt(pi + x\^{}2),
z == sqrt(pi + x\^{}2)
]
[[y < -2], [y > 1]]
2*pi + e is approximately 9.00146713563863
    \end{Verbatim}

    Now we will see some more details about solving equations and
manipulating their solutions.

    \hypertarget{solving-equations-and-inequalities}{%
\subsection{Solving equations and
inequalities}\label{solving-equations-and-inequalities}}

\textbf{Reference}
{[}\href{https://doc.sagemath.org/html/en/reference/calculus/sage/symbolic/expression.html}{1}{]}
for the details of \texttt{solve()} and \texttt{find\_root()},
{[}\href{https://doc.sagemath.org/html/en/reference/calculus/sage/symbolic/relation.html\#solving}{2}{]}
for examples.

Other than equations and inequalities, we can also solve systems: it is
enough to give Sage a list of expressions and a list of variables with
respect to which we want to solve. For example the system

\begin{align*}
    \begin{cases}
        x + y = 2 \\
        2x - y = 6
    \end{cases}
\end{align*}

Can be solved as

    \begin{tcolorbox}[breakable, size=fbox, boxrule=1pt, pad at break*=1mm,colback=cellbackground, colframe=cellborder]
\prompt{In}{incolor}{40}{\boxspacing}
\begin{Verbatim}[commandchars=\\\{\}]
\PY{n}{solve}\PY{p}{(}\PY{p}{[}\PY{n}{x}\PY{o}{+}\PY{n}{y} \PY{o}{==} \PY{l+m+mi}{2}\PY{p}{,} \PY{l+m+mi}{2}\PY{o}{*}\PY{n}{x} \PY{o}{\PYZhy{}} \PY{n}{y} \PY{o}{==} \PY{l+m+mi}{6}\PY{p}{]}\PY{p}{,} \PY{p}{[}\PY{n}{x}\PY{p}{,}\PY{n}{y}\PY{p}{]}\PY{p}{)}
\end{Verbatim}
\end{tcolorbox}

            \begin{tcolorbox}[breakable, size=fbox, boxrule=.5pt, pad at break*=1mm, opacityfill=0]
\prompt{Out}{outcolor}{40}{\boxspacing}
\begin{Verbatim}[commandchars=\\\{\}]
[[x == (8/3), y == (-2/3)]]
\end{Verbatim}
\end{tcolorbox}
        
    \textbf{Exercise.} Find the intersection of the circle of radius \(2\)
centered in the origin and the parabula of equation \(y=x^2-2x^2+1\).

    \hypertarget{the-set-of-solutions}{%
\subsubsection{The set of solutions}\label{the-set-of-solutions}}

One would expect the result of \texttt{solve()} to be a list of
solutions, but it is actually a list of expressions (technically it is
not a list but a different type of Python collection, but this is not so
important)

    \begin{tcolorbox}[breakable, size=fbox, boxrule=1pt, pad at break*=1mm,colback=cellbackground, colframe=cellborder]
\prompt{In}{incolor}{37}{\boxspacing}
\begin{Verbatim}[commandchars=\\\{\}]
\PY{n}{solutions} \PY{o}{=} \PY{n}{solve}\PY{p}{(}\PY{n}{x}\PY{o}{\PYZca{}}\PY{l+m+mi}{2}\PY{o}{\PYZhy{}}\PY{l+m+mi}{9} \PY{o}{==} \PY{l+m+mi}{0}\PY{p}{,} \PY{n}{x}\PY{p}{)}
\PY{n}{solutions}\PY{p}{[}\PY{l+m+mi}{0}\PY{p}{]} \PY{c+c1}{\PYZsh{} This is the expression \PYZsq{}x == \PYZhy{}3\PYZsq{}}
\end{Verbatim}
\end{tcolorbox}

            \begin{tcolorbox}[breakable, size=fbox, boxrule=.5pt, pad at break*=1mm, opacityfill=0]
\prompt{Out}{outcolor}{37}{\boxspacing}
\begin{Verbatim}[commandchars=\\\{\}]
x == -3
\end{Verbatim}
\end{tcolorbox}
        
    To read the actual solution without the \texttt{x\ ==} part you can use
the \texttt{rhs()} or \texttt{lhs()} functions, which can be applied to
any expression containing a relation operator (like \texttt{==},
\texttt{\textless{}}, \texttt{\textgreater{}=}\ldots) and return the
\emph{right hand side} and \emph{left hand side} of the expression,
respectively

    \begin{tcolorbox}[breakable, size=fbox, boxrule=1pt, pad at break*=1mm,colback=cellbackground, colframe=cellborder]
\prompt{In}{incolor}{41}{\boxspacing}
\begin{Verbatim}[commandchars=\\\{\}]
\PY{n}{f} \PY{o}{=}  \PY{n}{x} \PY{o}{==} \PY{l+m+mi}{2}
\PY{n+nb}{print}\PY{p}{(}\PY{l+s+s2}{\PYZdq{}}\PY{l+s+s2}{rhs:}\PY{l+s+s2}{\PYZdq{}}\PY{p}{,} \PY{n}{f}\PY{o}{.}\PY{n}{rhs}\PY{p}{(}\PY{p}{)}\PY{p}{)}
\PY{n+nb}{print}\PY{p}{(}\PY{l+s+s2}{\PYZdq{}}\PY{l+s+s2}{lhs:}\PY{l+s+s2}{\PYZdq{}}\PY{p}{,} \PY{n}{f}\PY{o}{.}\PY{n}{lhs}\PY{p}{(}\PY{p}{)}\PY{p}{)}
\end{Verbatim}
\end{tcolorbox}

    \begin{Verbatim}[commandchars=\\\{\}]
rhs: 2
lhs: x
    \end{Verbatim}

    When you solve an inequality or a system, the set of solutions can be
more complicated to describe. In this case the result is a list
containing lists of expressions that have to be \texttt{True} at the
same time. It is easier to explain with an example:

    \begin{tcolorbox}[breakable, size=fbox, boxrule=1pt, pad at break*=1mm,colback=cellbackground, colframe=cellborder]
\prompt{In}{incolor}{38}{\boxspacing}
\begin{Verbatim}[commandchars=\\\{\}]
\PY{n+nb}{print}\PY{p}{(}\PY{l+s+s2}{\PYZdq{}}\PY{l+s+s2}{Simple inequality:}\PY{l+s+s2}{\PYZdq{}}\PY{p}{,} \PY{n}{solve}\PY{p}{(}\PY{n}{x}\PY{o}{\PYZca{}}\PY{l+m+mi}{2}\PY{o}{\PYZhy{}}\PY{l+m+mi}{9} \PY{o}{\PYZgt{}} \PY{l+m+mi}{0}\PY{p}{,} \PY{n}{x}\PY{p}{)}\PY{p}{)}
\PY{n+nb}{print}\PY{p}{(}\PY{l+s+s2}{\PYZdq{}}\PY{l+s+s2}{System of inequalities:}\PY{l+s+se}{\PYZbs{}n}\PY{l+s+s2}{\PYZdq{}}\PY{p}{,} \PY{n}{solve}\PY{p}{(}\PY{p}{[}\PY{n}{x}\PY{o}{\PYZca{}}\PY{l+m+mi}{2}\PY{o}{\PYZhy{}}\PY{l+m+mi}{9} \PY{o}{\PYZgt{}} \PY{l+m+mi}{0}\PY{p}{,} \PY{n}{x} \PY{o}{\PYZlt{}} \PY{l+m+mi}{6}\PY{p}{]}\PY{p}{,} \PY{n}{x}\PY{p}{)}\PY{p}{)}
\end{Verbatim}
\end{tcolorbox}

    \begin{Verbatim}[commandchars=\\\{\}]
Simple inequality: [[x < -3], [x > 3]]
System of inequalities:
 [
[3 < x, x < 6],
[x < -3]
]
    \end{Verbatim}

    In the last example (system of inequalities), Sage is telling us that
the system \begin{align*}
    \begin{cases}
        x^2-9 > 9 \\
        x < 6
    \end{cases}
\end{align*} has two solutions: * \(x\) is between \(3\) and \(6\); *
\(x\) is less than \(-3\).

Since in Sage (and in Python) expressions can have at most on relational
operator like \texttt{\textless{}}, the first solution requires two
expressions to be described. Hence the ``list of lists''.

    \textbf{Exercise.} In the first exercise you were asked to solve a
system of equations, but some of its solutions were complex numbers.
Select only the real solutions and print them as pairs \((x,y)\).

    When solving a system of equations (not inequalities), you can use the
option \texttt{solution\_dict=True} to have the solutions arranged as a
\emph{dictionary}, which is a type of Python collection that we did not
treat in this course

    \begin{tcolorbox}[breakable, size=fbox, boxrule=1pt, pad at break*=1mm,colback=cellbackground, colframe=cellborder]
\prompt{In}{incolor}{44}{\boxspacing}
\begin{Verbatim}[commandchars=\\\{\}]
\PY{n}{solve}\PY{p}{(}\PY{p}{[}\PY{n}{x}\PY{o}{+}\PY{n}{y} \PY{o}{==} \PY{l+m+mi}{2}\PY{p}{,} \PY{l+m+mi}{2}\PY{o}{*}\PY{n}{x} \PY{o}{\PYZhy{}} \PY{n}{y} \PY{o}{==} \PY{l+m+mi}{6}\PY{p}{]}\PY{p}{,} \PY{p}{[}\PY{n}{x}\PY{p}{,}\PY{n}{y}\PY{p}{]}\PY{p}{,} \PY{n}{solution\PYZus{}dict}\PY{o}{=}\PY{k+kc}{True}\PY{p}{)}
\end{Verbatim}
\end{tcolorbox}

            \begin{tcolorbox}[breakable, size=fbox, boxrule=.5pt, pad at break*=1mm, opacityfill=0]
\prompt{Out}{outcolor}{44}{\boxspacing}
\begin{Verbatim}[commandchars=\\\{\}]
[\{x: 8/3, y: -2/3\}]
\end{Verbatim}
\end{tcolorbox}
        
    \hypertarget{alternative-method-for-real-roots-find_root}{%
\subsubsection{\texorpdfstring{Alternative method for real roots:
\texttt{find\_root()}}{Alternative method for real roots: find\_root()}}\label{alternative-method-for-real-roots-find_root}}

The \texttt{solve()} method is very useful when solving \emph{symbolic}
equations, for example when you have two variables and you want to solve
for one of them in terms of the other. However, it does not always find
explicit solutions.

When you want to find an explicit, even if approximate, solution, it can
be better to use \texttt{find\_root()}. This function works
\emph{numerically}, which means that it finds an approximation of the
root. It only works for real solutions and you need to specify an
interval where you want the root to be searched:

    \begin{tcolorbox}[breakable, size=fbox, boxrule=1pt, pad at break*=1mm,colback=cellbackground, colframe=cellborder]
\prompt{In}{incolor}{52}{\boxspacing}
\begin{Verbatim}[commandchars=\\\{\}]
\PY{n}{f} \PY{o}{=} \PY{n}{e}\PY{o}{\PYZca{}}\PY{n}{x} \PY{o}{+} \PY{n}{x} \PY{o}{\PYZhy{}} \PY{l+m+mi}{10}
\PY{n+nb}{print}\PY{p}{(}\PY{l+s+s2}{\PYZdq{}}\PY{l+s+s2}{Using solve():}\PY{l+s+se}{\PYZbs{}n}\PY{l+s+s2}{\PYZdq{}}\PY{p}{,} \PY{n}{solve}\PY{p}{(}\PY{n}{f}\PY{p}{,} \PY{n}{x}\PY{p}{)}\PY{p}{)}
\PY{n+nb}{print}\PY{p}{(}\PY{l+s+s2}{\PYZdq{}}\PY{l+s+s2}{Using find\PYZus{}root():}\PY{l+s+s2}{\PYZdq{}}\PY{p}{,} \PY{n}{f}\PY{o}{.}\PY{n}{find\PYZus{}root}\PY{p}{(}\PY{l+m+mi}{0}\PY{p}{,}\PY{l+m+mi}{100}\PY{p}{)}\PY{p}{)}
\end{Verbatim}
\end{tcolorbox}

    \begin{Verbatim}[commandchars=\\\{\}]
Using solve():
 [
x == -e\^{}x + 10
]
Using find\_root(): 2.070579904980303
    \end{Verbatim}

    \hypertarget{evaluating-functions}{%
\subsection{Evaluating functions}\label{evaluating-functions}}

If an expression contains only one variable you can evaluate it easily,
even if it is not a function.

    \begin{tcolorbox}[breakable, size=fbox, boxrule=1pt, pad at break*=1mm,colback=cellbackground, colframe=cellborder]
\prompt{In}{incolor}{21}{\boxspacing}
\begin{Verbatim}[commandchars=\\\{\}]
\PY{n}{var}\PY{p}{(}\PY{l+s+s1}{\PYZsq{}}\PY{l+s+s1}{y}\PY{l+s+s1}{\PYZsq{}}\PY{p}{)}
\PY{n}{f} \PY{o}{=} \PY{n}{x}\PY{o}{\PYZca{}}\PY{l+m+mi}{2}\PY{o}{\PYZhy{}}\PY{l+m+mi}{3}
\PY{n}{g} \PY{o}{=} \PY{n}{x} \PY{o}{\PYZgt{}} \PY{n}{x}\PY{o}{\PYZca{}}\PY{l+m+mi}{2}

\PY{n+nb}{print}\PY{p}{(}\PY{n}{f}\PY{p}{(}\PY{l+m+mi}{2}\PY{p}{)}\PY{p}{)}
\PY{n+nb}{print}\PY{p}{(}\PY{n}{g}\PY{p}{(}\PY{l+m+mi}{3}\PY{o}{+}\PY{n}{y}\PY{p}{)}\PY{p}{)}
\end{Verbatim}
\end{tcolorbox}

    \begin{Verbatim}[commandchars=\\\{\}]
1
y + 3 > (y + 3)\^{}2
    \end{Verbatim}

    If an expression contains more than one variable, you can specify a
value for each of them and they will be substituted in alphabetic order.
You can also specify a value only for some of the variables.

    \begin{tcolorbox}[breakable, size=fbox, boxrule=1pt, pad at break*=1mm,colback=cellbackground, colframe=cellborder]
\prompt{In}{incolor}{38}{\boxspacing}
\begin{Verbatim}[commandchars=\\\{\}]
\PY{n}{var}\PY{p}{(}\PY{l+s+s1}{\PYZsq{}}\PY{l+s+s1}{y}\PY{l+s+s1}{\PYZsq{}}\PY{p}{,}\PY{l+s+s1}{\PYZsq{}}\PY{l+s+s1}{z}\PY{l+s+s1}{\PYZsq{}}\PY{p}{)}

\PY{n}{f} \PY{o}{=} \PY{n}{y}\PY{o}{*}\PY{n}{z}\PY{o}{\PYZca{}}\PY{l+m+mi}{2} \PY{o}{\PYZhy{}} \PY{n}{y} \PY{o}{==} \PY{n}{z}
\PY{n+nb}{print}\PY{p}{(}\PY{n}{f}\PY{p}{(}\PY{l+m+mi}{2}\PY{p}{,} \PY{l+m+mi}{0}\PY{p}{)}\PY{p}{)}
\PY{n+nb}{print}\PY{p}{(}\PY{n}{f}\PY{p}{(}\PY{n}{z}\PY{o}{=}\PY{l+m+mi}{2}\PY{p}{)}\PY{p}{)}
\end{Verbatim}
\end{tcolorbox}

    \begin{Verbatim}[commandchars=\\\{\}]
-2 == 0
3*y == 2
    \end{Verbatim}

    \hypertarget{symbolic-computations}{%
\subsection{Symbolic computations}\label{symbolic-computations}}

Sage can understand and simplify symbolic expressions such as sums
(finite or infinite) and products. In the following cell, we compute the
following sums using the
\href{https://doc.sagemath.org/html/en/reference/calculus/sage/symbolic/expression.html\#sage.symbolic.expression.Expression.sum}{\texttt{sum()}}
function:

\begin{align*}
    \begin{array}{llcc}
        (1) & \sum_{k=0}^nk                 &=&\frac{n^2+n}{2}\\
        (2) & \sum_{k=0}^nk^4               &=&\frac{6n^5+15n^4+10n^3-n}{30}\\
        (3) & \sum_{k=0}^n\binom nk         &=& 2^n\\
        (4) & \sum_{k=0}^\infty \frac1{k^2} &=& \frac{\pi^2}{6}
    \end{array}
\end{align*}

    \begin{tcolorbox}[breakable, size=fbox, boxrule=1pt, pad at break*=1mm,colback=cellbackground, colframe=cellborder]
\prompt{In}{incolor}{22}{\boxspacing}
\begin{Verbatim}[commandchars=\\\{\}]
\PY{n}{var}\PY{p}{(}\PY{l+s+s1}{\PYZsq{}}\PY{l+s+s1}{k}\PY{l+s+s1}{\PYZsq{}}\PY{p}{,} \PY{l+s+s1}{\PYZsq{}}\PY{l+s+s1}{n}\PY{l+s+s1}{\PYZsq{}}\PY{p}{)} \PY{c+c1}{\PYZsh{} Remember to declare all variables}

\PY{n}{s} \PY{o}{=} \PY{p}{[}\PY{p}{]}
\PY{n}{s}\PY{o}{.}\PY{n}{append}\PY{p}{(}  \PY{n+nb}{sum}\PY{p}{(}\PY{n}{k}\PY{p}{,} \PY{n}{k}\PY{p}{,} \PY{l+m+mi}{0}\PY{p}{,} \PY{n}{n}\PY{p}{)}  \PY{p}{)}
\PY{n}{s}\PY{o}{.}\PY{n}{append}\PY{p}{(}  \PY{n+nb}{sum}\PY{p}{(}\PY{n}{k}\PY{o}{\PYZca{}}\PY{l+m+mi}{4}\PY{p}{,} \PY{n}{k}\PY{p}{,} \PY{l+m+mi}{0}\PY{p}{,} \PY{n}{n}\PY{p}{)}  \PY{p}{)}
\PY{n}{s}\PY{o}{.}\PY{n}{append}\PY{p}{(}  \PY{n+nb}{sum}\PY{p}{(}\PY{n}{binomial}\PY{p}{(}\PY{n}{n}\PY{p}{,}\PY{n}{k}\PY{p}{)}\PY{p}{,} \PY{n}{k}\PY{p}{,} \PY{l+m+mi}{0}\PY{p}{,} \PY{n}{n}\PY{p}{)} \PY{p}{)}
\PY{n}{s}\PY{o}{.}\PY{n}{append}\PY{p}{(}  \PY{n+nb}{sum}\PY{p}{(}\PY{l+m+mi}{1}\PY{o}{/}\PY{n}{k}\PY{o}{\PYZca{}}\PY{l+m+mi}{2}\PY{p}{,} \PY{n}{k}\PY{p}{,} \PY{l+m+mi}{1}\PY{p}{,} \PY{n}{infinity}\PY{p}{)}  \PY{p}{)}

\PY{k}{for} \PY{n}{i} \PY{o+ow}{in} \PY{n+nb}{range}\PY{p}{(}\PY{n+nb}{len}\PY{p}{(}\PY{n}{s}\PY{p}{)}\PY{p}{)}\PY{p}{:}
    \PY{n+nb}{print}\PY{p}{(}\PY{l+s+s2}{\PYZdq{}}\PY{l+s+s2}{(}\PY{l+s+si}{\PYZob{}\PYZcb{}}\PY{l+s+s2}{) }\PY{l+s+si}{\PYZob{}\PYZcb{}}\PY{l+s+s2}{\PYZdq{}}\PY{o}{.}\PY{n}{format}\PY{p}{(}\PY{n}{i}\PY{o}{+}\PY{l+m+mi}{1}\PY{p}{,} \PY{n}{s}\PY{p}{[}\PY{n}{i}\PY{p}{]}\PY{p}{)}\PY{p}{)}
\end{Verbatim}
\end{tcolorbox}

    \begin{Verbatim}[commandchars=\\\{\}]
(1) 1/2*n\^{}2 + 1/2*n
(2) 1/5*n\^{}5 + 1/2*n\^{}4 + 1/3*n\^{}3 - 1/30*n
(3) 2\^{}n
(4) 1/6*pi\^{}2
    \end{Verbatim}

    An alternative notation is \texttt{expression.sum(k,\ a,\ b)}. There is
an analogous
\href{https://doc.sagemath.org/html/en/reference/calculus/sage/symbolic/expression.html\#sage.symbolic.expression.Expression.prod}{\texttt{prod()}}
for products.

    Sometimes Sage tries to keep an expression in its original form without
expanding out sums and products. To change this behavior you can use the
\href{https://doc.sagemath.org/html/en/reference/calculus/sage/symbolic/expression.html\#sage.symbolic.expression.Expression.expand}{\texttt{expand()}}
function:

    \begin{tcolorbox}[breakable, size=fbox, boxrule=1pt, pad at break*=1mm,colback=cellbackground, colframe=cellborder]
\prompt{In}{incolor}{30}{\boxspacing}
\begin{Verbatim}[commandchars=\\\{\}]
\PY{n}{f} \PY{o}{=} \PY{p}{(}\PY{n}{x}\PY{o}{+}\PY{l+m+mi}{1}\PY{p}{)}\PY{o}{\PYZca{}}\PY{l+m+mi}{2} \PY{o}{\PYZhy{}} \PY{p}{(}\PY{n}{x}\PY{o}{\PYZhy{}}\PY{l+m+mi}{1}\PY{p}{)}\PY{o}{\PYZca{}}\PY{l+m+mi}{2}
\PY{n+nb}{print}\PY{p}{(}\PY{n}{f}\PY{p}{)}
\PY{n+nb}{print}\PY{p}{(}\PY{n}{f}\PY{o}{.}\PY{n}{expand}\PY{p}{(}\PY{p}{)}\PY{p}{)}
\end{Verbatim}
\end{tcolorbox}

    \begin{Verbatim}[commandchars=\\\{\}]
(x + 1)\^{}2 - (x - 1)\^{}2
4*x
    \end{Verbatim}

    \hypertarget{the-symbolic-ring}{%
\subsubsection{The Symbolic Ring}\label{the-symbolic-ring}}

\textbf{Reference:}
{[}\href{https://doc.sagemath.org/html/en/reference/calculus/sage/symbolic/ring.html}{3}{]}

The symbolic expressions that we have seen so far live in a ring called
\emph{symbolic ring} and denoted by \texttt{SR} in Sage. This ring works
like the ring \texttt{ZZ} of integers or \texttt{RR} of reals numbers.
In particular, you can define matrices and other objects using it as a
``basis''.

    \begin{tcolorbox}[breakable, size=fbox, boxrule=1pt, pad at break*=1mm,colback=cellbackground, colframe=cellborder]
\prompt{In}{incolor}{45}{\boxspacing}
\begin{Verbatim}[commandchars=\\\{\}]
\PY{n}{var}\PY{p}{(}\PY{l+s+s1}{\PYZsq{}}\PY{l+s+s1}{a}\PY{l+s+s1}{\PYZsq{}}\PY{p}{,} \PY{l+s+s1}{\PYZsq{}}\PY{l+s+s1}{b}\PY{l+s+s1}{\PYZsq{}}\PY{p}{,} \PY{l+s+s1}{\PYZsq{}}\PY{l+s+s1}{c}\PY{l+s+s1}{\PYZsq{}}\PY{p}{,} \PY{l+s+s1}{\PYZsq{}}\PY{l+s+s1}{d}\PY{l+s+s1}{\PYZsq{}}\PY{p}{)}

\PY{n}{M} \PY{o}{=} \PY{n}{matrix}\PY{p}{(}\PY{p}{[}\PY{p}{[}\PY{n}{a}\PY{p}{,}\PY{n}{b}\PY{p}{]}\PY{p}{,} \PY{p}{[}\PY{n}{c}\PY{p}{,}\PY{n}{d}\PY{p}{]}\PY{p}{]}\PY{p}{)}
\PY{n+nb}{print}\PY{p}{(}\PY{n}{M}\PY{o}{.}\PY{n}{determinant}\PY{p}{(}\PY{p}{)}\PY{p}{)}

\PY{n}{polring}\PY{o}{.}\PY{o}{\PYZlt{}}\PY{n}{x}\PY{o}{\PYZgt{}} \PY{o}{=} \PY{n}{SR}\PY{p}{[}\PY{p}{]}
\PY{n}{f} \PY{o}{=} \PY{n}{x}\PY{o}{\PYZca{}}\PY{l+m+mi}{2} \PY{o}{+} \PY{l+m+mi}{2}\PY{o}{*}\PY{n}{a}\PY{o}{*}\PY{n}{x} \PY{o}{+} \PY{n}{a}\PY{o}{\PYZca{}}\PY{l+m+mi}{2}
\PY{n+nb}{print}\PY{p}{(}\PY{n}{f}\PY{o}{.}\PY{n}{roots}\PY{p}{(}\PY{p}{)}\PY{p}{)}
\end{Verbatim}
\end{tcolorbox}

    \begin{Verbatim}[commandchars=\\\{\}]
-b*c + a*d
[(-a, 2)]
    \end{Verbatim}

    \textbf{Exercise.} Compute the eigenvalues of the matrix \begin{align*}
\begin{pmatrix}
\cos \alpha & \sin \alpha\\
-\sin\alpha & \cos \alpha
\end{pmatrix}
\end{align*}

    \hypertarget{calculus}{%
\section{Calculus}\label{calculus}}

\textbf{Reference:}
{[}\href{https://doc.sagemath.org/html/en/reference/calculus/index.html}{4}{]}
for an overview, but most functions are described in
{[}\href{https://doc.sagemath.org/html/en/reference/calculus/sage/symbolic/expression.html}{1}{]}

    \hypertarget{limits-and-series}{%
\subsection{Limits and series}\label{limits-and-series}}

\textbf{References:}
{[}\href{https://doc.sagemath.org/html/en/reference/calculus/sage/calculus/calculus.html\#sage.calculus.calculus.limit}{5}{]}
for limits,
{[}\href{https://doc.sagemath.org/html/en/reference/calculus/sage/symbolic/expression.html\#sage.symbolic.expression.Expression.series}{6}{]}
for series

You can compute limits

    \begin{tcolorbox}[breakable, size=fbox, boxrule=1pt, pad at break*=1mm,colback=cellbackground, colframe=cellborder]
\prompt{In}{incolor}{54}{\boxspacing}
\begin{Verbatim}[commandchars=\\\{\}]
\PY{n}{f} \PY{o}{=} \PY{n}{sin}\PY{p}{(}\PY{n}{x}\PY{p}{)}\PY{o}{/}\PY{n}{x}
\PY{c+c1}{\PYZsh{} print(f(0)) \PYZsh{} This one gives an error}
\PY{n+nb}{print}\PY{p}{(} \PY{n}{f}\PY{o}{.}\PY{n}{limit}\PY{p}{(}\PY{n}{x}\PY{o}{=}\PY{l+m+mi}{0}\PY{p}{)} \PY{p}{)}

\PY{n+nb}{print}\PY{p}{(} \PY{p}{(}\PY{n}{e}\PY{o}{\PYZca{}}\PY{p}{(}\PY{o}{\PYZhy{}}\PY{n}{x}\PY{p}{)}\PY{p}{)}\PY{o}{.}\PY{n}{limit}\PY{p}{(}\PY{n}{x}\PY{o}{=}\PY{n}{infinity}\PY{p}{)} \PY{p}{)}
\end{Verbatim}
\end{tcolorbox}

    \begin{Verbatim}[commandchars=\\\{\}]
1
0
    \end{Verbatim}

    \textbf{Exercise.} Compute the constant \(e\) using a limit.

    You can also specify a direction for the limit. If you don't, Sage
assumes that you want to take a two-sided limit.

    \begin{tcolorbox}[breakable, size=fbox, boxrule=1pt, pad at break*=1mm,colback=cellbackground, colframe=cellborder]
\prompt{In}{incolor}{55}{\boxspacing}
\begin{Verbatim}[commandchars=\\\{\}]
\PY{n}{f} \PY{o}{=} \PY{n+nb}{abs}\PY{p}{(}\PY{n}{x}\PY{p}{)}\PY{o}{/}\PY{n}{x}  \PY{c+c1}{\PYZsh{} 1 if x\PYZgt{}0, \PYZhy{}1 if x\PYZlt{}0}
\PY{n+nb}{print}\PY{p}{(} \PY{n}{f}\PY{o}{.}\PY{n}{limit}\PY{p}{(}\PY{n}{x}\PY{o}{=}\PY{l+m+mi}{0}\PY{p}{)} \PY{p}{)} \PY{c+c1}{\PYZsh{} undefined}
\PY{n+nb}{print}\PY{p}{(} \PY{n}{f}\PY{o}{.}\PY{n}{limit}\PY{p}{(}\PY{n}{x}\PY{o}{=}\PY{l+m+mi}{0}\PY{p}{,} \PY{n+nb}{dir}\PY{o}{=}\PY{l+s+s2}{\PYZdq{}}\PY{l+s+s2}{+}\PY{l+s+s2}{\PYZdq{}}\PY{p}{)} \PY{p}{)}
\PY{n+nb}{print}\PY{p}{(} \PY{n}{f}\PY{o}{.}\PY{n}{limit}\PY{p}{(}\PY{n}{x}\PY{o}{=}\PY{l+m+mi}{0}\PY{p}{,} \PY{n+nb}{dir}\PY{o}{=}\PY{l+s+s2}{\PYZdq{}}\PY{l+s+s2}{\PYZhy{}}\PY{l+s+s2}{\PYZdq{}}\PY{p}{)} \PY{p}{)}
\end{Verbatim}
\end{tcolorbox}

    \begin{Verbatim}[commandchars=\\\{\}]
und
1
-1
    \end{Verbatim}

    There is also the alternative notation \texttt{limit(f,\ x,\ dir)} which
does the same as \texttt{f.limit(x,\ dir)}.

    You can also compute series expansions up to any order. \textbf{Watch
out:} the notation uses \texttt{==} instead of \texttt{=} as
\texttt{limit()} does.

    \begin{tcolorbox}[breakable, size=fbox, boxrule=1pt, pad at break*=1mm,colback=cellbackground, colframe=cellborder]
\prompt{In}{incolor}{56}{\boxspacing}
\begin{Verbatim}[commandchars=\\\{\}]
\PY{n}{f} \PY{o}{=} \PY{n}{e}\PY{o}{\PYZca{}}\PY{n}{x}
\PY{n}{g} \PY{o}{=} \PY{n}{sin}\PY{p}{(}\PY{n}{x}\PY{p}{)} \PY{o}{\PYZhy{}} \PY{l+m+mi}{2}\PY{o}{*}\PY{n}{cos}\PY{p}{(}\PY{n}{x}\PY{p}{)}
\PY{n}{h} \PY{o}{=} \PY{n}{log}\PY{p}{(}\PY{n}{x}\PY{p}{)}

\PY{n+nb}{print}\PY{p}{(}\PY{n}{f}\PY{o}{.}\PY{n}{series}\PY{p}{(}\PY{n}{x}\PY{o}{==}\PY{l+m+mi}{0}\PY{p}{,} \PY{l+m+mi}{3}\PY{p}{)}\PY{p}{)}
\PY{n+nb}{print}\PY{p}{(}\PY{n}{g}\PY{o}{.}\PY{n}{series}\PY{p}{(}\PY{n}{x}\PY{o}{==}\PY{l+m+mi}{0}\PY{p}{,} \PY{l+m+mi}{7}\PY{p}{)}\PY{p}{)}
\PY{n+nb}{print}\PY{p}{(}\PY{n}{h}\PY{o}{.}\PY{n}{series}\PY{p}{(}\PY{n}{x}\PY{o}{==}\PY{l+m+mi}{1}\PY{p}{,} \PY{l+m+mi}{3}\PY{p}{)}\PY{p}{)}
\end{Verbatim}
\end{tcolorbox}

    \begin{Verbatim}[commandchars=\\\{\}]
1 + 1*x + 1/2*x\^{}2 + Order(x\^{}3)
(-2) + 1*x + 1*x\^{}2 + (-1/6)*x\^{}3 + (-1/12)*x\^{}4 + 1/120*x\^{}5 + 1/360*x\^{}6 +
Order(x\^{}7)
1*(x - 1) + (-1/2)*(x - 1)\^{}2 + Order((x - 1)\^{}3)
    \end{Verbatim}

    \hypertarget{derivatives}{%
\subsection{Derivatives}\label{derivatives}}

\textbf{References:}
{[}\href{https://doc.sagemath.org/html/en/reference/calculus/sage/symbolic/expression.html\#sage.symbolic.expression.Expression.derivative}{7}{]}
and
{[}\href{https://doc.sagemath.org/html/en/reference/calculus/sage/calculus/functional.html\#sage.calculus.functional.derivative}{8}{]}
for derivatives,
{[}\href{https://doc.sagemath.org/html/en/reference/calculus/sage/calculus/functions.html\#sage.calculus.functions.jacobian}{9}{]}
for the Jacobian matrix and
{[}\href{https://doc.sagemath.org/html/en/reference/calculus/sage/symbolic/expression.html\#sage.symbolic.expression.Expression.hessian}{10}{]}
for the Hessian.

    When computing derivatives, you need to specify with respect to which
variables you want to derive, except in case there is only one.

    \begin{tcolorbox}[breakable, size=fbox, boxrule=1pt, pad at break*=1mm,colback=cellbackground, colframe=cellborder]
\prompt{In}{incolor}{57}{\boxspacing}
\begin{Verbatim}[commandchars=\\\{\}]
\PY{n}{var}\PY{p}{(}\PY{l+s+s1}{\PYZsq{}}\PY{l+s+s1}{y}\PY{l+s+s1}{\PYZsq{}}\PY{p}{)}
\PY{n+nb}{print}\PY{p}{(} \PY{p}{(}\PY{n}{x}\PY{o}{\PYZca{}}\PY{l+m+mi}{2}\PY{o}{+}\PY{l+m+mi}{2}\PY{o}{*}\PY{n}{y}\PY{o}{\PYZca{}}\PY{l+m+mi}{4}\PY{p}{)}\PY{o}{.}\PY{n}{derivative}\PY{p}{(}\PY{n}{y}\PY{p}{)} \PY{p}{)} \PY{c+c1}{\PYZsh{} Alternative: derivative(f, y)}
\PY{n+nb}{print}\PY{p}{(} \PY{p}{(}\PY{l+m+mi}{2}\PY{o}{*}\PY{n}{x}\PY{o}{\PYZca{}}\PY{l+m+mi}{3}\PY{o}{\PYZhy{}}\PY{n}{x}\PY{o}{+}\PY{l+m+mi}{2}\PY{p}{)}\PY{o}{.}\PY{n}{derivative}\PY{p}{(}\PY{p}{)} \PY{p}{)}
\end{Verbatim}
\end{tcolorbox}

    \begin{Verbatim}[commandchars=\\\{\}]
8*y\^{}3
6*x\^{}2 - 1
    \end{Verbatim}

    You can also compute higher order derivatives:

    \begin{tcolorbox}[breakable, size=fbox, boxrule=1pt, pad at break*=1mm,colback=cellbackground, colframe=cellborder]
\prompt{In}{incolor}{58}{\boxspacing}
\begin{Verbatim}[commandchars=\\\{\}]
\PY{n+nb}{print}\PY{p}{(} \PY{p}{(}\PY{n}{x}\PY{o}{\PYZca{}}\PY{l+m+mi}{3}\PY{p}{)}\PY{o}{.}\PY{n}{derivative}\PY{p}{(}\PY{n}{x}\PY{p}{,} \PY{n}{x}\PY{p}{)} \PY{p}{)} \PY{c+c1}{\PYZsh{} Same as (x\PYZca{}3).derivative(x, 2)}

\PY{n}{f} \PY{o}{=} \PY{n}{x}\PY{o}{\PYZca{}}\PY{l+m+mi}{7}\PY{o}{*}\PY{n}{y}\PY{o}{\PYZca{}}\PY{l+m+mi}{2} \PY{o}{+} \PY{n}{x}\PY{o}{\PYZca{}}\PY{l+m+mi}{4}\PY{o}{*}\PY{n}{y}\PY{o}{\PYZca{}}\PY{l+m+mi}{2} \PY{o}{\PYZhy{}} \PY{l+m+mi}{2}\PY{o}{*}\PY{n}{x}\PY{o}{\PYZca{}}\PY{l+m+mi}{3} \PY{o}{+} \PY{n}{x}\PY{o}{\PYZca{}}\PY{l+m+mi}{2}\PY{o}{*}\PY{n}{y}\PY{o}{\PYZca{}}\PY{l+m+mi}{5} \PY{o}{+} \PY{n}{y} \PY{o}{+} \PY{l+m+mi}{2}
\PY{n+nb}{print}\PY{p}{(} \PY{n}{f}\PY{o}{.}\PY{n}{derivative}\PY{p}{(}\PY{n}{x}\PY{p}{,} \PY{n}{x}\PY{p}{,} \PY{n}{y}\PY{p}{)} \PY{p}{)}    \PY{c+c1}{\PYZsh{} Twice in x, once in y}
\PY{n+nb}{print}\PY{p}{(} \PY{n}{f}\PY{o}{.}\PY{n}{derivative}\PY{p}{(}\PY{n}{x}\PY{p}{,} \PY{l+m+mi}{4}\PY{p}{,} \PY{n}{y}\PY{p}{,} \PY{l+m+mi}{2}\PY{p}{)} \PY{p}{)} \PY{c+c1}{\PYZsh{} 4 times in x, twice in y}
\end{Verbatim}
\end{tcolorbox}

    \begin{Verbatim}[commandchars=\\\{\}]
6*x
84*x\^{}5*y + 10*y\^{}4 + 24*x\^{}2*y
1680*x\^{}3 + 48
    \end{Verbatim}

    Jacobian and Hessian matrices are also easy to compute:

    \begin{tcolorbox}[breakable, size=fbox, boxrule=1pt, pad at break*=1mm,colback=cellbackground, colframe=cellborder]
\prompt{In}{incolor}{59}{\boxspacing}
\begin{Verbatim}[commandchars=\\\{\}]
\PY{n}{f} \PY{o}{=} \PY{p}{(}\PY{o}{\PYZhy{}}\PY{n}{x}\PY{o}{\PYZca{}}\PY{l+m+mi}{2} \PY{o}{+} \PY{l+m+mi}{2}\PY{o}{*}\PY{n}{x}\PY{o}{*}\PY{n}{y}\PY{p}{,} \PY{n}{y}\PY{o}{\PYZca{}}\PY{l+m+mi}{3}\PY{p}{,} \PY{n}{x}\PY{o}{+}\PY{n}{y}\PY{o}{+}\PY{n}{x}\PY{o}{*}\PY{n}{y}\PY{p}{)}
\PY{n+nb}{print}\PY{p}{(} \PY{n}{jacobian}\PY{p}{(}\PY{n}{f}\PY{p}{,} \PY{p}{[}\PY{n}{x}\PY{p}{,}\PY{n}{y}\PY{p}{]}\PY{p}{)}\PY{p}{,} \PY{l+s+s2}{\PYZdq{}}\PY{l+s+se}{\PYZbs{}n}\PY{l+s+s2}{\PYZdq{}} \PY{p}{)}

\PY{n}{g} \PY{o}{=} \PY{n}{x}\PY{o}{\PYZca{}}\PY{l+m+mi}{2} \PY{o}{+} \PY{n}{x}\PY{o}{*}\PY{n}{y} \PY{o}{+} \PY{n}{y}\PY{o}{\PYZca{}}\PY{l+m+mi}{3} \PY{o}{\PYZhy{}}\PY{l+m+mi}{2}\PY{o}{*}\PY{n}{x}\PY{o}{*}\PY{n}{y}\PY{o}{\PYZca{}}\PY{l+m+mi}{2} \PY{o}{\PYZhy{}}\PY{l+m+mi}{3}
\PY{n+nb}{print}\PY{p}{(} \PY{n}{g}\PY{o}{.}\PY{n}{hessian}\PY{p}{(}\PY{p}{)} \PY{p}{)}
\end{Verbatim}
\end{tcolorbox}

    \begin{Verbatim}[commandchars=\\\{\}]
[-2*x + 2*y        2*x]
[         0      3*y\^{}2]
[     y + 1      x + 1]

[         2   -4*y + 1]
[  -4*y + 1 -4*x + 6*y]
    \end{Verbatim}

    \emph{Note:} the notation \texttt{f.jacobian({[}x,y{]})} is also valid,
but only if you specify that \texttt{f} is vector by declaring it as
\texttt{f\ =\ vector({[}...{]})}.

    \hypertarget{integrals}{%
\subsection{Integrals}\label{integrals}}

\textbf{References:}
{[}\href{https://doc.sagemath.org/html/en/reference/calculus/sage/symbolic/integration/integral.html}{11}{]}
for symbolic integration and
{[}\href{https://doc.sagemath.org/html/en/reference/calculus/sage/calculus/integration.html}{12}{]}
for numerical methods.

You should remember from high school or from your first
calculus/analysis course that derivatives are easy, but integrals are
hard. When using a computer software to solve your integrals, you have
two choices:

\begin{enumerate}
\def\labelenumi{\arabic{enumi}.}
\tightlist
\item
  You can try to compute a primitive function exactly, and then (if you
  are computing a definite integral) substitute the endpoints of your
  integration interval to get the result. We can call this
  \emph{symbolic integration}.
\item
  You can get an \emph{approximated} result with a \emph{numerical
  method}. This method always gives some kind of result, but it cannot
  be used to compute indefinite integrals.
\end{enumerate}

Sage can do both of these things, although people that work in numerical
analysis and use often the second method tend to prefer other programs,
such as Matlab (or its open-source clone Octave).

    \hypertarget{symbolic-integration}{%
\subsubsection{Symbolic integration}\label{symbolic-integration}}

Symbolic integrals work more or less like derivatives. You must specify
an integration variable, but the endpoints of the integration interval
are optional. If they are not given you get an indefinite integral.

    \begin{tcolorbox}[breakable, size=fbox, boxrule=1pt, pad at break*=1mm,colback=cellbackground, colframe=cellborder]
\prompt{In}{incolor}{60}{\boxspacing}
\begin{Verbatim}[commandchars=\\\{\}]
\PY{n}{var}\PY{p}{(}\PY{l+s+s1}{\PYZsq{}}\PY{l+s+s1}{a}\PY{l+s+s1}{\PYZsq{}}\PY{p}{,} \PY{l+s+s1}{\PYZsq{}}\PY{l+s+s1}{b}\PY{l+s+s1}{\PYZsq{}}\PY{p}{)}
\PY{n}{f} \PY{o}{=} \PY{n}{x} \PY{o}{+} \PY{n}{sin}\PY{p}{(}\PY{n}{x}\PY{p}{)}
\PY{n+nb}{print}\PY{p}{(} \PY{n}{f}\PY{o}{.}\PY{n}{integral}\PY{p}{(}\PY{n}{x}\PY{p}{)} \PY{p}{)} \PY{c+c1}{\PYZsh{} Alternative: integral(f, x)}
\PY{n+nb}{print}\PY{p}{(} \PY{n}{f}\PY{o}{.}\PY{n}{integral}\PY{p}{(}\PY{n}{x}\PY{p}{,} \PY{o}{\PYZhy{}}\PY{l+m+mi}{10}\PY{p}{,} \PY{l+m+mi}{10}\PY{p}{)} \PY{p}{)}
\PY{n+nb}{print}\PY{p}{(} \PY{n}{f}\PY{o}{.}\PY{n}{integral}\PY{p}{(}\PY{n}{x}\PY{p}{,} \PY{n}{a}\PY{p}{,} \PY{n}{b}\PY{p}{)} \PY{p}{)}
\end{Verbatim}
\end{tcolorbox}

    \begin{Verbatim}[commandchars=\\\{\}]
1/2*x\^{}2 - cos(x)
0
-1/2*a\^{}2 + 1/2*b\^{}2 + cos(a) - cos(b)
    \end{Verbatim}

    Your endpoints can also be \(\pm\infty\):

    \begin{tcolorbox}[breakable, size=fbox, boxrule=1pt, pad at break*=1mm,colback=cellbackground, colframe=cellborder]
\prompt{In}{incolor}{61}{\boxspacing}
\begin{Verbatim}[commandchars=\\\{\}]
\PY{n+nb}{print}\PY{p}{(} \PY{n}{integral}\PY{p}{(}\PY{n}{e}\PY{o}{\PYZca{}}\PY{p}{(}\PY{o}{\PYZhy{}}\PY{n}{x}\PY{p}{)}\PY{p}{,}   \PY{n}{x}\PY{p}{,} \PY{l+m+mi}{0}\PY{p}{,}         \PY{n}{infinity}\PY{p}{)} \PY{p}{)}
\PY{n+nb}{print}\PY{p}{(} \PY{n}{integral}\PY{p}{(}\PY{n}{e}\PY{o}{\PYZca{}}\PY{p}{(}\PY{o}{\PYZhy{}}\PY{n}{x}\PY{o}{\PYZca{}}\PY{l+m+mi}{2}\PY{p}{)}\PY{p}{,} \PY{n}{x}\PY{p}{,} \PY{o}{\PYZhy{}}\PY{n}{infinity}\PY{p}{,} \PY{n}{infinity}\PY{p}{)} \PY{p}{)}
\end{Verbatim}
\end{tcolorbox}

    \begin{Verbatim}[commandchars=\\\{\}]
1
sqrt(pi)
    \end{Verbatim}

    The last function is also an example of an integral that perhaps you
might want to compute numerically. In fact:

    \begin{tcolorbox}[breakable, size=fbox, boxrule=1pt, pad at break*=1mm,colback=cellbackground, colframe=cellborder]
\prompt{In}{incolor}{65}{\boxspacing}
\begin{Verbatim}[commandchars=\\\{\}]
\PY{n+nb}{print}\PY{p}{(} \PY{n}{integral}\PY{p}{(}\PY{n}{e}\PY{o}{\PYZca{}}\PY{p}{(}\PY{o}{\PYZhy{}}\PY{n}{x}\PY{o}{\PYZca{}}\PY{l+m+mi}{2}\PY{p}{)}\PY{p}{,} \PY{n}{x}\PY{p}{)} \PY{p}{)}
\PY{n+nb}{print}\PY{p}{(} \PY{n}{integral}\PY{p}{(}\PY{n}{e}\PY{o}{\PYZca{}}\PY{p}{(}\PY{o}{\PYZhy{}}\PY{n}{x}\PY{o}{\PYZca{}}\PY{l+m+mi}{2}\PY{p}{)}\PY{p}{,} \PY{n}{x}\PY{p}{,} \PY{l+m+mi}{1}\PY{p}{,} \PY{l+m+mi}{2}\PY{p}{)} \PY{p}{)}
\end{Verbatim}
\end{tcolorbox}

    \begin{Verbatim}[commandchars=\\\{\}]
1/2*sqrt(pi)*erf(x)
1/2*sqrt(pi)*erf(2) - 1/2*sqrt(pi)*erf(1)
    \end{Verbatim}

    Here \texttt{erf(x)} denotes the
\href{https://en.wikipedia.org/wiki/Error_function}{error function}.

    \hypertarget{numerical-integration}{%
\subsubsection{Numerical integration}\label{numerical-integration}}

In order to get an explicit value for the computations above, we can use
a \emph{numerical} method.

The word ``numerical'' does not have much to do with numbers, but it
refers to the fact that we are trying to compute explicit results rather
than symbolic or algebraic ones.
\href{https://en.wikipedia.org/wiki/Numerical_analysis}{Numerical
analysis} is the branch of mathematics that studies methods to
approximate computations over the real or complex numbers. With these
methods there is usually a trade-off between speed and precision.

The Sage function
\href{https://doc.sagemath.org/html/en/reference/calculus/sage/calculus/integration.html\#sage.calculus.integration.numerical_integral}{\texttt{numerical\_integral()}}
takes as a parameter a real-valued one-variable function and the
integration endpoints, and it returns both an approximate value for the
integral and an error estimate.

    \begin{tcolorbox}[breakable, size=fbox, boxrule=1pt, pad at break*=1mm,colback=cellbackground, colframe=cellborder]
\prompt{In}{incolor}{40}{\boxspacing}
\begin{Verbatim}[commandchars=\\\{\}]
\PY{n}{numerical\PYZus{}integral}\PY{p}{(}\PY{n}{e}\PY{o}{\PYZca{}}\PY{p}{(}\PY{o}{\PYZhy{}}\PY{n}{x}\PY{o}{\PYZca{}}\PY{l+m+mi}{2}\PY{p}{)}\PY{p}{,} \PY{l+m+mi}{1}\PY{p}{,} \PY{l+m+mi}{2}\PY{p}{)}
\end{Verbatim}
\end{tcolorbox}

            \begin{tcolorbox}[breakable, size=fbox, boxrule=.5pt, pad at break*=1mm, opacityfill=0]
\prompt{Out}{outcolor}{40}{\boxspacing}
\begin{Verbatim}[commandchars=\\\{\}]
(0.13525725794999466, 1.5016572202374808e-15)
\end{Verbatim}
\end{tcolorbox}
        
    The result above means, in symbols \begin{align*}
\int_1^2 e^{-x^2}\mathrm dx = 0.13525725794999466 \pm 1.5016572202374808\times 10^{-15}
\end{align*}

There is also a
\href{https://doc.sagemath.org/html/en/reference/calculus/sage/calculus/integration.html\#sage.calculus.integration.monte_carlo_integral}{\texttt{monte\_carlo\_integral()}}
method for functions with more than one variable.

    \textbf{Exercise.} Compute the area of the ellipse of equation
\(y^2+\left(\frac x3\right)^2=1\).

    \hypertarget{differential-equations}{%
\subsection{Differential equations}\label{differential-equations}}

\textbf{Reference:}
{[}\href{https://doc.sagemath.org/html/en/reference/calculus/sage/calculus/desolvers.html}{13}{]}

A
\href{https://en.wikipedia.org/wiki/Differential_equation}{differential
equation} is an equation involving an unknwon function and its
derivatives. They can be of two kinds: \emph{ordinary} differential
equations
(\href{https://en.wikipedia.org/wiki/Ordinary_differential_equation}{ODE})
and \emph{partial} differential equations
(\href{https://en.wikipedia.org/wiki/Partial_differential_equation}{PDE}).
The latter involve multivariate functions and their partial derivatives.

Differential equations are in general hard to solve \emph{exactly} (or
\emph{symbolically}): even a simple equation of the form \(f'(x)=g(x)\),
where \(g(x)\) is someknown function, requires solving the integral
\(\int g(x)\mathrm{d}x\) in order to find \(f\), which as we know is not
always easy!

Theoretical results on differential equations usually ensure the
existence and/or uniquess of a solution under certain conditions, but in
general they do not give a way to solve them. There exits many methods
to find approximate solutions, and some of them are implemented in Sage
as well (see
{[}\href{https://doc.sagemath.org/html/en/reference/calculus/sage/calculus/desolvers.html}{13}{]}).
However we will focus on the simple ODEs that can be solved exactly.

Let's start with a simple example. Let's find all functions \(f(x)\)
such that \(f'(x)=f(x)\). In order to do so, we need to use the
\texttt{function()} construct, which allows us to define an ``unknwon''
function inside Sage, like we define variables with \texttt{var()}.

    \begin{tcolorbox}[breakable, size=fbox, boxrule=1pt, pad at break*=1mm,colback=cellbackground, colframe=cellborder]
\prompt{In}{incolor}{4}{\boxspacing}
\begin{Verbatim}[commandchars=\\\{\}]
\PY{n}{var}\PY{p}{(}\PY{l+s+s1}{\PYZsq{}}\PY{l+s+s1}{x}\PY{l+s+s1}{\PYZsq{}}\PY{p}{)}
\PY{n}{function}\PY{p}{(}\PY{l+s+s1}{\PYZsq{}}\PY{l+s+s1}{f}\PY{l+s+s1}{\PYZsq{}}\PY{p}{)}
\PY{n}{equation} \PY{o}{=} \PY{n}{derivative}\PY{p}{(}\PY{n}{f}\PY{p}{(}\PY{n}{x}\PY{p}{)}\PY{p}{)} \PY{o}{==} \PY{n}{f}\PY{p}{(}\PY{n}{x}\PY{p}{)}
\PY{n}{desolve}\PY{p}{(}\PY{n}{equation}\PY{p}{,} \PY{n}{f}\PY{p}{(}\PY{n}{x}\PY{p}{)}\PY{p}{)} \PY{c+c1}{\PYZsh{} f is the unknown function}
\end{Verbatim}
\end{tcolorbox}

            \begin{tcolorbox}[breakable, size=fbox, boxrule=.5pt, pad at break*=1mm, opacityfill=0]
\prompt{Out}{outcolor}{4}{\boxspacing}
\begin{Verbatim}[commandchars=\\\{\}]
\_C*e\^{}x
\end{Verbatim}
\end{tcolorbox}
        
    As you can expect, they are all the functions \(Ce^x\) for some constant
\(C\). The constant \(C\) plays the same role as the constant in the
solution of an integral, but in this case Sage writes it explicitly.

We can also specify \emph{initial conditions} for our function. For
example we can impose that \(f(0)=3\) as follows:

    \begin{tcolorbox}[breakable, size=fbox, boxrule=1pt, pad at break*=1mm,colback=cellbackground, colframe=cellborder]
\prompt{In}{incolor}{5}{\boxspacing}
\begin{Verbatim}[commandchars=\\\{\}]
\PY{n}{desolve}\PY{p}{(}\PY{n}{equation}\PY{p}{,} \PY{n}{f}\PY{p}{(}\PY{n}{x}\PY{p}{)}\PY{p}{,} \PY{p}{(}\PY{l+m+mi}{0}\PY{p}{,}\PY{l+m+mi}{3}\PY{p}{)}\PY{p}{)}
\end{Verbatim}
\end{tcolorbox}

            \begin{tcolorbox}[breakable, size=fbox, boxrule=.5pt, pad at break*=1mm, opacityfill=0]
\prompt{Out}{outcolor}{5}{\boxspacing}
\begin{Verbatim}[commandchars=\\\{\}]
3*e\^{}x
\end{Verbatim}
\end{tcolorbox}
        
    You can also solve \emph{second order} equations, that is equations
where the second derivative also appears. In this case if you want to
specify an initial condition you should write the triple of values
\((x_0, f(x_0), f'(x_0))\).

    \begin{tcolorbox}[breakable, size=fbox, boxrule=1pt, pad at break*=1mm,colback=cellbackground, colframe=cellborder]
\prompt{In}{incolor}{6}{\boxspacing}
\begin{Verbatim}[commandchars=\\\{\}]
\PY{n}{equation} \PY{o}{=} \PY{n}{derivative}\PY{p}{(}\PY{n}{f}\PY{p}{(}\PY{n}{x}\PY{p}{)}\PY{p}{,} \PY{n}{x}\PY{p}{,} \PY{l+m+mi}{2}\PY{p}{)} \PY{o}{+} \PY{n}{x}\PY{o}{*}\PY{n}{derivative}\PY{p}{(}\PY{n}{f}\PY{p}{(}\PY{n}{x}\PY{p}{)}\PY{p}{)} \PY{o}{==} \PY{l+m+mi}{1}
\PY{n}{desolve}\PY{p}{(}\PY{n}{equation}\PY{p}{,} \PY{n}{f}\PY{p}{(}\PY{n}{x}\PY{p}{)}\PY{p}{,} \PY{p}{(}\PY{l+m+mi}{0}\PY{p}{,} \PY{l+m+mi}{0}\PY{p}{,} \PY{l+m+mi}{0}\PY{p}{)}\PY{p}{)}
\end{Verbatim}
\end{tcolorbox}

            \begin{tcolorbox}[breakable, size=fbox, boxrule=.5pt, pad at break*=1mm, opacityfill=0]
\prompt{Out}{outcolor}{6}{\boxspacing}
\begin{Verbatim}[commandchars=\\\{\}]
-1/2*I*sqrt(2)*sqrt(pi)*integrate(erf(1/2*I*sqrt(2)*x)*e\^{}(-1/2*x\^{}2), x)
\end{Verbatim}
\end{tcolorbox}
        
    \textbf{Exercise.} Use Sage to find out the functions \(f(x)\) that
satisfy \begin{align*}
    \begin{array}{rlcrl}
        (A) &
        \begin{cases}
            f(0)   &= 1\\
            f'(0)  &= 0\\
            f''(x) &= -f(x)
        \end{cases}
        & \qquad \qquad &
        (B) &
        \begin{cases}
            f(0)   &= 0\\
            f'(0)  &= 1\\
            f''(x) &= -f(x)
        \end{cases}
    \end{array}
\end{align*}

    \begin{tcolorbox}[breakable, size=fbox, boxrule=1pt, pad at break*=1mm,colback=cellbackground, colframe=cellborder]
\prompt{In}{incolor}{ }{\boxspacing}
\begin{Verbatim}[commandchars=\\\{\}]

\end{Verbatim}
\end{tcolorbox}

    \hypertarget{a-real-world-example}{%
\subsubsection{A real-world example}\label{a-real-world-example}}

Differential equations have countless applications in Science, so it
would be a shame not to see at least a simple one.

Consider an object moving with constant acceleration \(a\). Its velocity
at time \(t\) is described by the formula \(v(t) = v(0) + at\). For
example an object falling from the sky has acceleration
\(g\sim 9.8 m/s^2\) towards the ground, so its velocity is
\(v(t) = -gt\).

However in the real world you need to take into account the air's
resistance, which depends (among other things) on the velocity of the
object. In this case the acceleration \(a(t)\) is not constant anymore,
and it satisfies an equation of the form \(a(t)=-g -kv(t)\), where \(k\)
is some constant that may depend on the shape and mass of the object (in
practice it may be more complicated than this).

Since the acceleration is the derivative of the velocity, we have a
differential equation \begin{align*}
    v'(t) = -g -kv(t)
\end{align*} and we can try to solve it with Sage!

    \begin{tcolorbox}[breakable, size=fbox, boxrule=1pt, pad at break*=1mm,colback=cellbackground, colframe=cellborder]
\prompt{In}{incolor}{7}{\boxspacing}
\begin{Verbatim}[commandchars=\\\{\}]
\PY{n}{var}\PY{p}{(}\PY{l+s+s1}{\PYZsq{}}\PY{l+s+s1}{t}\PY{l+s+s1}{\PYZsq{}}\PY{p}{)}
\PY{n}{function}\PY{p}{(}\PY{l+s+s1}{\PYZsq{}}\PY{l+s+s1}{v}\PY{l+s+s1}{\PYZsq{}}\PY{p}{)}
\PY{n}{g} \PY{o}{=} \PY{l+m+mf}{9.8}
\PY{n}{k} \PY{o}{=} \PY{l+m+mf}{1.5}
\PY{n}{conditions} \PY{o}{=} \PY{p}{(}\PY{l+m+mi}{0}\PY{p}{,} \PY{l+m+mi}{0}\PY{p}{)} \PY{c+c1}{\PYZsh{} Start with velocity 0}
\PY{n}{desolve}\PY{p}{(}\PY{n}{derivative}\PY{p}{(}\PY{n}{v}\PY{p}{(}\PY{n}{t}\PY{p}{)}\PY{p}{)} \PY{o}{==} \PY{o}{\PYZhy{}}\PY{n}{g} \PY{o}{\PYZhy{}}\PY{n}{k}\PY{o}{*}\PY{n}{v}\PY{p}{(}\PY{n}{t}\PY{p}{)}\PY{p}{,} \PY{n}{v}\PY{p}{(}\PY{n}{t}\PY{p}{)}\PY{p}{,} \PY{n}{conditions}\PY{p}{)}
\end{Verbatim}
\end{tcolorbox}

            \begin{tcolorbox}[breakable, size=fbox, boxrule=.5pt, pad at break*=1mm, opacityfill=0]
\prompt{Out}{outcolor}{7}{\boxspacing}
\begin{Verbatim}[commandchars=\\\{\}]
-98/15*(e\^{}(3/2*t) - 1)*e\^{}(-3/2*t)
\end{Verbatim}
\end{tcolorbox}
        
    If you want to solve this equation symbolically (that is, keeping \(g\)
and \(k\) in symbols) you need to specify that \(t\) is the
\emph{independent variable} of the equation:

    \begin{tcolorbox}[breakable, size=fbox, boxrule=1pt, pad at break*=1mm,colback=cellbackground, colframe=cellborder]
\prompt{In}{incolor}{10}{\boxspacing}
\begin{Verbatim}[commandchars=\\\{\}]
\PY{n}{var}\PY{p}{(}\PY{l+s+s1}{\PYZsq{}}\PY{l+s+s1}{t}\PY{l+s+s1}{\PYZsq{}}\PY{p}{,} \PY{l+s+s1}{\PYZsq{}}\PY{l+s+s1}{g}\PY{l+s+s1}{\PYZsq{}}\PY{p}{,} \PY{l+s+s1}{\PYZsq{}}\PY{l+s+s1}{k}\PY{l+s+s1}{\PYZsq{}}\PY{p}{)}
\PY{n}{function}\PY{p}{(}\PY{l+s+s1}{\PYZsq{}}\PY{l+s+s1}{v}\PY{l+s+s1}{\PYZsq{}}\PY{p}{)}
\PY{n}{conditions} \PY{o}{=} \PY{p}{(}\PY{l+m+mi}{0}\PY{p}{,} \PY{l+m+mi}{0}\PY{p}{)} \PY{c+c1}{\PYZsh{} Start with velocity 0}
\PY{n}{desolve}\PY{p}{(}\PY{n}{derivative}\PY{p}{(}\PY{n}{v}\PY{p}{(}\PY{n}{t}\PY{p}{)}\PY{p}{)} \PY{o}{==} \PY{o}{\PYZhy{}}\PY{n}{g} \PY{o}{\PYZhy{}}\PY{n}{k}\PY{o}{*}\PY{n}{v}\PY{p}{(}\PY{n}{t}\PY{p}{)}\PY{p}{,} \PY{n}{v}\PY{p}{(}\PY{n}{t}\PY{p}{)}\PY{p}{,} \PY{n}{conditions}\PY{p}{,} \PY{n}{ivar}\PY{o}{=}\PY{n}{t}\PY{p}{)}
\end{Verbatim}
\end{tcolorbox}

            \begin{tcolorbox}[breakable, size=fbox, boxrule=.5pt, pad at break*=1mm, opacityfill=0]
\prompt{Out}{outcolor}{10}{\boxspacing}
\begin{Verbatim}[commandchars=\\\{\}]
-(g*e\^{}(k*t) - g)*e\^{}(-k*t)/k
\end{Verbatim}
\end{tcolorbox}
        
    \hypertarget{basic-data-analysis-and-visualization}{%
\section{Basic data analysis and
visualization}\label{basic-data-analysis-and-visualization}}

\hypertarget{statistics}{%
\subsection{Statistics}\label{statistics}}

\textbf{References:}
{[}\href{https://doc.sagemath.org/html/en/reference/stats/sage/stats/basic_stats.html}{14}{]}

Sage includes the most basic functions for statistical analysis.

    \begin{tcolorbox}[breakable, size=fbox, boxrule=1pt, pad at break*=1mm,colback=cellbackground, colframe=cellborder]
\prompt{In}{incolor}{20}{\boxspacing}
\begin{Verbatim}[commandchars=\\\{\}]
\PY{n}{L} \PY{o}{=} \PY{p}{[}\PY{l+m+mi}{1}\PY{p}{,} \PY{l+m+mi}{2}\PY{p}{,} \PY{l+m+mi}{3}\PY{p}{,} \PY{l+m+mi}{3}\PY{p}{,} \PY{o}{\PYZhy{}}\PY{l+m+mi}{6}\PY{p}{,} \PY{o}{\PYZhy{}}\PY{l+m+mi}{2}\PY{p}{,} \PY{l+m+mi}{4}\PY{p}{,} \PY{o}{\PYZhy{}}\PY{l+m+mi}{1}\PY{p}{,} \PY{l+m+mi}{0}\PY{p}{,} \PY{l+m+mi}{2}\PY{p}{,} \PY{l+m+mi}{3}\PY{p}{,} \PY{o}{\PYZhy{}}\PY{l+m+mi}{4}\PY{p}{,} \PY{l+m+mi}{0}\PY{p}{]}

\PY{n+nb}{print}\PY{p}{(}\PY{l+s+s2}{\PYZdq{}}\PY{l+s+s2}{Values:}\PY{l+s+se}{\PYZbs{}t}\PY{l+s+s2}{\PYZdq{}}\PY{p}{,} \PY{n}{L}\PY{p}{)}

\PY{n+nb}{print}\PY{p}{(}\PY{l+s+s2}{\PYZdq{}}\PY{l+s+s2}{Mean:}\PY{l+s+se}{\PYZbs{}t}\PY{l+s+se}{\PYZbs{}t}\PY{l+s+se}{\PYZbs{}t}\PY{l+s+s2}{\PYZdq{}}\PY{p}{,}           \PY{n}{mean}\PY{p}{(}\PY{n}{L}\PY{p}{)}\PY{p}{)}
\PY{n+nb}{print}\PY{p}{(}\PY{l+s+s2}{\PYZdq{}}\PY{l+s+s2}{Median:}\PY{l+s+se}{\PYZbs{}t}\PY{l+s+se}{\PYZbs{}t}\PY{l+s+se}{\PYZbs{}t}\PY{l+s+s2}{\PYZdq{}}\PY{p}{,}         \PY{n}{median}\PY{p}{(}\PY{n}{L}\PY{p}{)}\PY{p}{)}
\PY{n+nb}{print}\PY{p}{(}\PY{l+s+s2}{\PYZdq{}}\PY{l+s+s2}{Mode:}\PY{l+s+se}{\PYZbs{}t}\PY{l+s+se}{\PYZbs{}t}\PY{l+s+se}{\PYZbs{}t}\PY{l+s+s2}{\PYZdq{}}\PY{p}{,}           \PY{n}{mode}\PY{p}{(}\PY{n}{L}\PY{p}{)}\PY{p}{)}

\PY{n+nb}{print}\PY{p}{(}\PY{l+s+s2}{\PYZdq{}}\PY{l+s+s2}{Standard deviation:}\PY{l+s+se}{\PYZbs{}t}\PY{l+s+s2}{\PYZdq{}}\PY{p}{,} \PY{n}{std}\PY{p}{(}\PY{n}{L}\PY{p}{)}\PY{p}{)}
\PY{n+nb}{print}\PY{p}{(}\PY{l+s+s2}{\PYZdq{}}\PY{l+s+s2}{Variance:}\PY{l+s+se}{\PYZbs{}t}\PY{l+s+se}{\PYZbs{}t}\PY{l+s+s2}{\PYZdq{}}\PY{p}{,}         \PY{n}{variance}\PY{p}{(}\PY{n}{L}\PY{p}{)}\PY{p}{)}

\PY{n+nb}{print}\PY{p}{(}\PY{l+s+s2}{\PYZdq{}}\PY{l+s+s2}{Moving average (5):}\PY{l+s+s2}{\PYZdq{}}\PY{p}{,} \PY{n}{moving\PYZus{}average}\PY{p}{(}\PY{n}{L}\PY{p}{,}\PY{l+m+mi}{5}\PY{p}{)}\PY{p}{)}
\end{Verbatim}
\end{tcolorbox}

    \begin{Verbatim}[commandchars=\\\{\}]
Values:  [1, 2, 3, 3, -6, -2, 4, -1, 0, 2, 3, -4, 0]
Mean:                    5/13
Median:                  1
Mode:                    [3]
Standard deviation:      2*sqrt(29/13)
Variance:                116/13
Moving average (5): [3/5, 0, 2/5, -2/5, -1, 3/5, 8/5, 0, 1/5]
    \end{Verbatim}

    You can also compare your data to a probability distribution, see
\href{https://doc.sagemath.org/html/en/reference/probability/sage/probability/probability_distribution.html}{this
page}. If you need to do more advanced statistics you should consider
using \href{https://www.r-project.org/}{R}; you can also use it inside
Sage.

    \hypertarget{plotting}{%
\subsection{Plotting}\label{plotting}}

\textbf{Reference:}
{[}\href{https://doc.sagemath.org/html/en/reference/plotting/index.html}{15}{]},
more specifically the subsection
{[}\href{https://doc.sagemath.org/html/en/reference/plotting/sage/plot/plot.html}{16}{]}.

Some Sage objects can be plotted:

    \begin{tcolorbox}[breakable, size=fbox, boxrule=1pt, pad at break*=1mm,colback=cellbackground, colframe=cellborder]
\prompt{In}{incolor}{21}{\boxspacing}
\begin{Verbatim}[commandchars=\\\{\}]
\PY{n}{f} \PY{o}{=} \PY{n}{sin}\PY{p}{(}\PY{n}{x}\PY{p}{)}
\PY{n}{plot}\PY{p}{(}\PY{n}{f}\PY{p}{)}
\end{Verbatim}
\end{tcolorbox}
 
            
\prompt{Out}{outcolor}{21}{}
    
    \begin{center}
    \adjustimage{max size={0.9\linewidth}{0.9\paperheight}}{output_75_0.png}
    \end{center}
    { \hspace*{\fill} \\}
    

    Sage's plotting functions are based on Python's
\href{https://matplotlib.org/}{matplotlib}.

You can give a number of options to adjust the aspect of your plot, see
\href{https://doc.sagemath.org/html/en/reference/plotting/sage/plot/plot.html\#sage.plot.plot.plot}{here}.
Let's see some of them:

    \begin{tcolorbox}[breakable, size=fbox, boxrule=1pt, pad at break*=1mm,colback=cellbackground, colframe=cellborder]
\prompt{In}{incolor}{67}{\boxspacing}
\begin{Verbatim}[commandchars=\\\{\}]
\PY{n}{f} \PY{o}{=} \PY{n}{sin}\PY{p}{(}\PY{n}{x}\PY{p}{)}
\PY{n}{plot}\PY{p}{(}\PY{n}{f}\PY{p}{,}
     \PY{o}{\PYZhy{}}\PY{l+m+mi}{2}\PY{o}{*}\PY{n}{pi}\PY{p}{,} \PY{l+m+mi}{2}\PY{o}{*}\PY{n}{pi}\PY{p}{,}                   \PY{c+c1}{\PYZsh{} bounds for x}
     \PY{n}{ymin}  \PY{o}{=} \PY{o}{\PYZhy{}}\PY{l+m+mf}{0.7}\PY{p}{,} \PY{n}{ymax} \PY{o}{=} \PY{l+m+mf}{0.7}\PY{p}{,}      \PY{c+c1}{\PYZsh{} bounds for y}
     \PY{n}{color} \PY{o}{=} \PY{l+s+s2}{\PYZdq{}}\PY{l+s+s2}{red}\PY{l+s+s2}{\PYZdq{}}\PY{p}{,}
     \PY{n}{title} \PY{o}{=} \PY{l+s+s2}{\PYZdq{}}\PY{l+s+s2}{The sin function}\PY{l+s+s2}{\PYZdq{}}\PY{p}{,}
    \PY{p}{)}
\end{Verbatim}
\end{tcolorbox}
 
            
\prompt{Out}{outcolor}{67}{}
    
    \begin{center}
    \adjustimage{max size={0.9\linewidth}{0.9\paperheight}}{output_77_0.png}
    \end{center}
    { \hspace*{\fill} \\}
    

    Some of the options are not described precisely in Sage's documentation,
but you can find them on
\href{https://matplotlib.org/stable/contents.html}{matplotlib's
documentation}. You can find many examples online for adjusting your
plot as you like!

    If you need to plot more than one object at the time, you can sum two
plots and show them together with \texttt{show()}:

    \begin{tcolorbox}[breakable, size=fbox, boxrule=1pt, pad at break*=1mm,colback=cellbackground, colframe=cellborder]
\prompt{In}{incolor}{36}{\boxspacing}
\begin{Verbatim}[commandchars=\\\{\}]
\PY{n}{cosine}      \PY{o}{=} \PY{n}{plot}\PY{p}{(}\PY{n}{cos}\PY{p}{(}\PY{n}{x}\PY{p}{)}\PY{p}{,} \PY{p}{(}\PY{n}{x}\PY{p}{,}\PY{o}{\PYZhy{}}\PY{n}{pi}\PY{o}{/}\PY{l+m+mi}{2}\PY{p}{,}\PY{n}{pi}\PY{o}{/}\PY{l+m+mi}{2}\PY{p}{)}\PY{p}{,} \PY{n}{color}\PY{o}{=}\PY{l+s+s2}{\PYZdq{}}\PY{l+s+s2}{red}\PY{l+s+s2}{\PYZdq{}}\PY{p}{)}
\PY{n}{exponential} \PY{o}{=} \PY{n}{plot}\PY{p}{(}\PY{n}{exp}\PY{p}{(}\PY{n}{x}\PY{p}{)}\PY{p}{,} \PY{p}{(}\PY{n}{x}\PY{p}{,}\PY{o}{\PYZhy{}}\PY{l+m+mi}{2}\PY{p}{,}\PY{l+m+mf}{0.5}\PY{p}{)}\PY{p}{)}

\PY{n}{show}\PY{p}{(}\PY{n}{cosine} \PY{o}{+} \PY{n}{exponential}\PY{p}{)}
\end{Verbatim}
\end{tcolorbox}

    \begin{center}
    \adjustimage{max size={0.9\linewidth}{0.9\paperheight}}{output_80_0.png}
    \end{center}
    { \hspace*{\fill} \\}
    
    Finally, there are other types of plots that you can use, like
\href{https://doc.sagemath.org/html/en/reference/plotting/sage/plot/scatter_plot.html\#sage.plot.scatter_plot.scatter_plot}{scatter
plots} and
\href{https://doc.sagemath.org/html/en/reference/plotting/sage/plot/bar_chart.html\#sage.plot.bar_chart.bar_chart}{bar
charts}. You can also add
\href{https://doc.sagemath.org/html/en/reference/plotting/sage/plot/text.html\#sage.plot.text.text}{text}
to your plot:

    \begin{tcolorbox}[breakable, size=fbox, boxrule=1pt, pad at break*=1mm,colback=cellbackground, colframe=cellborder]
\prompt{In}{incolor}{53}{\boxspacing}
\begin{Verbatim}[commandchars=\\\{\}]
\PY{n}{b} \PY{o}{=} \PY{n}{bar\PYZus{}chart}\PY{p}{(}\PY{n+nb}{range}\PY{p}{(}\PY{l+m+mi}{1}\PY{p}{,}\PY{l+m+mi}{10}\PY{p}{)}\PY{p}{)}
\PY{n}{s} \PY{o}{=} \PY{n}{scatter\PYZus{}plot}\PY{p}{(}\PY{p}{[}\PY{p}{(}\PY{l+m+mi}{1}\PY{p}{,}\PY{l+m+mi}{5}\PY{p}{)}\PY{p}{,} \PY{p}{(}\PY{l+m+mi}{4}\PY{p}{,}\PY{l+m+mi}{2}\PY{p}{)}\PY{p}{,} \PY{p}{(}\PY{l+m+mi}{8}\PY{p}{,}\PY{l+m+mi}{8}\PY{p}{)}\PY{p}{,} \PY{p}{(}\PY{l+m+mi}{4}\PY{p}{,}\PY{l+m+mi}{7}\PY{p}{)}\PY{p}{]}\PY{p}{,}
                 \PY{n}{marker} \PY{o}{=} \PY{l+s+s2}{\PYZdq{}}\PY{l+s+s2}{*}\PY{l+s+s2}{\PYZdq{}}\PY{p}{,}       \PY{c+c1}{\PYZsh{} symbol}
                 \PY{n}{markersize} \PY{o}{=} \PY{l+m+mi}{100}\PY{p}{,}
                 \PY{n}{edgecolor}  \PY{o}{=} \PY{l+s+s2}{\PYZdq{}}\PY{l+s+s2}{black}\PY{l+s+s2}{\PYZdq{}}\PY{p}{,}
                 \PY{n}{facecolor}  \PY{o}{=} \PY{l+s+s2}{\PYZdq{}}\PY{l+s+s2}{red}\PY{l+s+s2}{\PYZdq{}}
                \PY{p}{)}
\PY{n}{t} \PY{o}{=} \PY{n}{text}\PY{p}{(}\PY{l+s+s2}{\PYZdq{}}\PY{l+s+s2}{wow, such plot!}\PY{l+s+s2}{\PYZdq{}}\PY{p}{,} \PY{p}{(}\PY{l+m+mi}{1}\PY{p}{,} \PY{l+m+mi}{8}\PY{p}{)}\PY{p}{,} \PY{n}{color}\PY{o}{=}\PY{l+s+s2}{\PYZdq{}}\PY{l+s+s2}{black}\PY{l+s+s2}{\PYZdq{}}\PY{p}{,} \PY{n}{fontsize}\PY{o}{=}\PY{l+m+mi}{20}\PY{p}{)}
\PY{n}{show}\PY{p}{(}\PY{n}{b} \PY{o}{+} \PY{n}{s} \PY{o}{+} \PY{n}{t}\PY{p}{)}
\end{Verbatim}
\end{tcolorbox}

    \begin{center}
    \adjustimage{max size={0.9\linewidth}{0.9\paperheight}}{output_82_0.png}
    \end{center}
    { \hspace*{\fill} \\}
    
    \hypertarget{interpolation}{%
\subsection{Interpolation}\label{interpolation}}

\textbf{References:}
{[}\href{https://doc.sagemath.org/html/en/reference/polynomial_rings/sage/rings/polynomial/polynomial_ring.html\#sage.rings.polynomial.polynomial_ring.PolynomialRing_field.lagrange_polynomial}{17}{]}
and
{[}\href{https://doc.sagemath.org/html/en/reference/calculus/sage/calculus/interpolation.html}{18}{]}.

When you need to work with a discrete set of data, like measurements of
real-world quantities, it can be useful to visualize a ``smoothed out''
version of this data, for example by plotting a function that
approximates it.

One way to do so is finding the lowest-degree polynomial that passes
through all your points. This is called
\href{https://en.wikipedia.org/wiki/Lagrange_polynomial}{Lagrange
Polynomial}.

    \begin{tcolorbox}[breakable, size=fbox, boxrule=1pt, pad at break*=1mm,colback=cellbackground, colframe=cellborder]
\prompt{In}{incolor}{65}{\boxspacing}
\begin{Verbatim}[commandchars=\\\{\}]
\PY{n}{points} \PY{o}{=} \PY{p}{[} \PY{p}{(}\PY{l+m+mi}{0}\PY{p}{,}\PY{l+m+mi}{1}\PY{p}{)}\PY{p}{,} \PY{p}{(}\PY{l+m+mi}{1}\PY{p}{,}\PY{l+m+mi}{2}\PY{p}{)}\PY{p}{,} \PY{p}{(}\PY{l+m+mf}{1.5}\PY{p}{,}\PY{l+m+mi}{0}\PY{p}{)}\PY{p}{,} \PY{p}{(}\PY{l+m+mi}{2}\PY{p}{,}\PY{l+m+mi}{4}\PY{p}{)}\PY{p}{,} \PY{p}{(}\PY{l+m+mi}{3}\PY{p}{,}\PY{l+m+mi}{5}\PY{p}{)} \PY{p}{]}
\PY{n}{polring}\PY{o}{.}\PY{o}{\PYZlt{}}\PY{n}{x}\PY{o}{\PYZgt{}} \PY{o}{=} \PY{n}{QQ}\PY{p}{[}\PY{p}{]} \PY{c+c1}{\PYZsh{} you need to specify a polynomial ring}
\PY{n}{lp} \PY{o}{=} \PY{n}{polring}\PY{o}{.}\PY{n}{lagrange\PYZus{}polynomial}\PY{p}{(}\PY{n}{points}\PY{p}{)}
\PY{n}{show}\PY{p}{(}\PY{n}{scatter\PYZus{}plot}\PY{p}{(}\PY{n}{points}\PY{p}{,} \PY{n}{facecolor}\PY{o}{=}\PY{l+s+s2}{\PYZdq{}}\PY{l+s+s2}{red}\PY{l+s+s2}{\PYZdq{}}\PY{p}{)}
     \PY{o}{+} \PY{n}{plot}\PY{p}{(}\PY{n}{lp}\PY{p}{,} \PY{l+m+mi}{0}\PY{p}{,} \PY{l+m+mi}{3}\PY{p}{)} \PY{c+c1}{\PYZsh{} slightly different notation for polynomials}
     \PY{o}{+} \PY{n}{text}\PY{p}{(}\PY{n}{lp}\PY{p}{,} \PY{p}{(}\PY{l+m+mi}{1}\PY{p}{,}\PY{l+m+mi}{8}\PY{p}{)}\PY{p}{,} \PY{n}{color}\PY{o}{=}\PY{l+s+s2}{\PYZdq{}}\PY{l+s+s2}{black}\PY{l+s+s2}{\PYZdq{}}\PY{p}{)}
    \PY{p}{)}
\end{Verbatim}
\end{tcolorbox}

    \begin{center}
    \adjustimage{max size={0.9\linewidth}{0.9\paperheight}}{output_84_0.png}
    \end{center}
    { \hspace*{\fill} \\}
    
    One can compute the Lagrange Polynomial over any base ring, and it has
the advantage that it is a very ``nice'' function (continuous and
differentiable as much as you like, with easily computable derivatives
and primitives).

However, it does not always give you good approximation of your data:

    \begin{tcolorbox}[breakable, size=fbox, boxrule=1pt, pad at break*=1mm,colback=cellbackground, colframe=cellborder]
\prompt{In}{incolor}{2}{\boxspacing}
\begin{Verbatim}[commandchars=\\\{\}]
\PY{n}{R} \PY{o}{=} \PY{p}{[}\PY{n}{x}\PY{o}{/}\PY{l+m+mi}{10} \PY{k}{for} \PY{n}{x} \PY{o+ow}{in} \PY{n+nb}{range}\PY{p}{(}\PY{o}{\PYZhy{}}\PY{l+m+mi}{10}\PY{p}{,}\PY{l+m+mi}{10}\PY{p}{)}\PY{p}{]}
\PY{n}{L} \PY{o}{=} \PY{p}{[}\PY{l+m+mi}{1}\PY{o}{/}\PY{p}{(}\PY{l+m+mi}{1}\PY{o}{+}\PY{l+m+mi}{25}\PY{o}{*}\PY{n}{x}\PY{o}{\PYZca{}}\PY{l+m+mi}{2}\PY{p}{)} \PY{k}{for} \PY{n}{x} \PY{o+ow}{in} \PY{n}{R}\PY{p}{]}
\PY{n}{points} \PY{o}{=} \PY{p}{[}\PY{p}{(}\PY{n}{R}\PY{p}{[}\PY{n}{i}\PY{p}{]}\PY{p}{,} \PY{n}{L}\PY{p}{[}\PY{n}{i}\PY{p}{]}\PY{p}{)} \PY{k}{for} \PY{n}{i} \PY{o+ow}{in} \PY{n+nb}{range}\PY{p}{(}\PY{n+nb}{len}\PY{p}{(}\PY{n}{L}\PY{p}{)}\PY{p}{)}\PY{p}{]}
\PY{n}{polring}\PY{o}{.}\PY{o}{\PYZlt{}}\PY{n}{x}\PY{o}{\PYZgt{}} \PY{o}{=} \PY{n}{RR}\PY{p}{[}\PY{p}{]}
\PY{n}{lp} \PY{o}{=} \PY{n}{polring}\PY{o}{.}\PY{n}{lagrange\PYZus{}polynomial}\PY{p}{(}\PY{n}{points}\PY{p}{)}

\PY{n}{show}\PY{p}{(}\PY{n}{plot}\PY{p}{(}\PY{n}{lp}\PY{p}{,} \PY{o}{\PYZhy{}}\PY{l+m+mf}{0.82}\PY{p}{,} \PY{l+m+mf}{0.72}\PY{p}{)} \PY{o}{+} \PY{n}{scatter\PYZus{}plot}\PY{p}{(}\PY{n}{points}\PY{p}{)}\PY{p}{)}
\end{Verbatim}
\end{tcolorbox}

    \begin{center}
    \adjustimage{max size={0.9\linewidth}{0.9\paperheight}}{output_86_0.png}
    \end{center}
    { \hspace*{\fill} \\}
    
    This particular example is called
\href{https://en.wikipedia.org/wiki/Runge\%27s_phenomenon}{Runge's
phenomenon}. For a better approximation you can use a
\href{https://en.wikipedia.org/wiki/Spline_(mathematics)}{spline}, which
is a \emph{piecewise} polynomial function:

    \begin{tcolorbox}[breakable, size=fbox, boxrule=1pt, pad at break*=1mm,colback=cellbackground, colframe=cellborder]
\prompt{In}{incolor}{90}{\boxspacing}
\begin{Verbatim}[commandchars=\\\{\}]
\PY{n}{show}\PY{p}{(}\PY{n}{plot}\PY{p}{(}\PY{n}{spline}\PY{p}{(}\PY{n}{points}\PY{p}{)}\PY{p}{,} \PY{o}{\PYZhy{}}\PY{l+m+mi}{1}\PY{p}{,} \PY{l+m+mi}{1}\PY{p}{)} \PY{o}{+} \PY{n}{scatter\PYZus{}plot}\PY{p}{(}\PY{n}{points}\PY{p}{)}\PY{p}{)}
\end{Verbatim}
\end{tcolorbox}

    \begin{center}
    \adjustimage{max size={0.9\linewidth}{0.9\paperheight}}{output_88_0.png}
    \end{center}
    { \hspace*{\fill} \\}
    
    A detailed explanation of splines is a good topic for a course of
numerical analysis. For this course it is enough that you know that they
exist and they can be plotted.


    % Add a bibliography block to the postdoc
    
    
    
\end{document}
