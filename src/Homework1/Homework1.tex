\documentclass[12pt,a4paper]{article}
\usepackage[utf8]{inputenc}
\usepackage{amsmath}
\usepackage{amsfonts}
\usepackage{amssymb}
\usepackage{amsthm}
\usepackage[left=2cm,right=2cm,top=2cm,bottom=2cm]{geometry}

\title{Mathematical software - homework 1}
\author{Sebastiano Tronto}

\newtheorem{thm}{Theorem}
\newtheorem{prop}[thm]{Proposition}

\theoremstyle{definition}
\newtheorem{ex}{Exercise}

\theoremstyle{definition}
\newtheorem*{remark}{Remark}

\begin{document}

\noindent\hrulefill

\begin{center}
\Huge{\textbf{Mathematical Software - Homework 1}}
\end{center}

\noindent\hrulefill
\begin{center}
\begin{tabular}{lcr}
\texttt{sebastiano.tronto@uni.lu} & \qquad \qquad \qquad \qquad &
\textbf{Deadline: Sunday, March 28th}
\end{tabular}
\end{center}

\vspace{1cm}

\begin{center}
  \emph{\large
    For each of the following exercises submit a .tex and a .pdf file.
  }
\end{center}

\vspace{1cm}

\begin{ex}
  Write a short Latex document that contains the following theorem-like
  environments using the \texttt{\textbackslash newtheorem} command of the
  \texttt{amsthm} package (the box around the text is not needed):
  \begin{center}
    \fbox{\parbox{0.95\textwidth}{
      \begin{prop}[Fundamental Theorem of Algebra]
        \label{prop:fta}
        Let \(p(x)\) be a non-constant polynomial with coefficients in
        $\mathbb C$. Then there is \(z\in\mathbb C\) such that $p(z)=0$.
      \end{prop}

      \begin{remark}
        Proposition \ref{prop:fta} is not true for polynomials with
        coefficients in $\mathbb R$. For example
        \begin{align}
          p(x) = x^2+1
        \end{align}
        does not have real roots.
      \end{remark}

      \begin{thm}
        If $X$ and $Y$ are $\sigma$-finite measure spaces and $f:X\times Y\to
        \mathbb R$ is measurable and such that
        \begin{align*}
          \int_{X\times Y}|f(x,y)|\mathrm d(x,y) < \infty
        \end{align*}
        then
        \begin{align}
          \label{eq:fubini}
          \int_X\left(\int_Yf(x,y)\mathrm d y\right)\mathrm d x =
          \int_Y\left(\int_Xf(x,y)\mathrm d x\right)\mathrm d y =
          \int_{X\times Y} f(x,y)\mathrm d(x,y)\,.
        \end{align}
      \end{thm}

      \begin{remark}
        In practice, equation \eqref{eq:fubini} means that we can switch the
        order of integration in a double integral.
      \end{remark}
    }}
  \end{center}
  Notice that Propositions, Theorems and some of the equations are numbered,
  and some of them are referred to in the Remarks. This numbering should change
  accordingly if more numbered Theorems and equations are added before this
  part of the text.
\end{ex}

\vspace{0.8cm}

\begin{ex}
[This is basically a ``do whatever you want, just do something'' exercise]
Write a Mathematical document in Latex, where ``Mathematical'' means any text
that contains both inline and displaystyle symbols and formulas, and where most
of the sentences contain a Mathematical symbol.
For example, you can submit (part of) your homework or notes for another course
that you are following at the moment.

You must follow the following rules:
\begin{itemize}
  \item Your document must be at least 2 pages long.
  \item Use the following in your preamble:
    \begin{verbatim}
\documentaclass[12pt,a4paper]{article}
\usepackage[left=2.5cm,right=2.5cm,top=2.5cm,bottom=2.5cm]{geometry}
    \end{verbatim}
\end{itemize}
The document itself can be about any area of Mathematics, Physics or similar.
The correctness of the Mathematics itself will not be judged for this course.
\end{ex}

\section*{Grading}

This homework assignment is worth 25\% of your final grade.

\vspace{0.3cm}
\textbf{Exercise 1 (10 points).}
\begin{itemize}
  \item A correct use of the \texttt{\textbackslash newtheorem} command is
        worth 4 out of 10 points.
  \item A correct use of the labelling and reference system is worth 4 points.
  \item Reproducing correctly the mathematical formulas is worth 2 points.
\end{itemize}

\textbf{Exercise 2 (10 points).}
\begin{itemize}
  \item A resulting pdf file that looks readable and correct is worth 6 out of
        10 points.
  \item The correct use of latex commands and environments is worth 4 out of
        10 points. \emph{Examples of mistakes in this sense: adjusting manually
        the size of parenthesis instead of using \texttt{\textbackslash left}
        and \texttt{\textbackslash right}; using \texttt{\textbackslash Sigma}
        when you actually want to write a sum symbol.}
\end{itemize}


\end{document}