\documentclass[11pt]{beamer}
\usetheme{Madrid}
\usepackage[utf8]{inputenc}
\usepackage{amsmath}

\usepackage{svg}
\usepackage{color}
\usepackage{listings}
\usepackage{mathtools}
\usepackage{tikz-cd}
\usepackage{adjustbox}

\definecolor{myblue}{rgb}{0,0,0.5}
\lstset{
	language=Python,
	tabsize=4,
	basicstyle=\footnotesize,
	keywordstyle=\bf\color{myblue},
	commentstyle=\it\color{gray},
	numbers=left,
	numbersep=3pt,
	numberstyle=\tiny\color{gray},
}

\author[\texttt{sebastiano.tronto@uni.lu}]{Sebastiano Tronto}
\title[Students requests]%
{Students requests}
\logo{\includegraphics[scale=0.1]{img/unilu.jpg}} 
%\institute{University of Luxembourg} 

\date{2021-05-21} 

\begin{document}

\begin{frame}
  \titlepage
\end{frame}

\begin{frame}[plain]
	\begin{center} {\Huge More cryptography} \end{center}
\end{frame}

\begin{frame}{Cryptography}
	What we have seen:

	\vspace{0.3cm}
	\begin{itemize}
		\item \textbf{RSA:}
			sending messages using a private key / public key pair
		\item \textbf{Flip-a-coin:}
			cryptographic ``proof'' that the opponent is not cheating
	\end{itemize}
\end{frame}

\begin{frame}{Cryptography}
	\begin{itemize}
		\item Rely on integer factorization being hard

			\vspace{0.3cm}
			\textbf{Example:} the best-known factorization algorithm
			(\href{https://en.wikipedia.org/wiki/General\_number\_field\_sieve}%
			{\emph{General number field sieve}}) has complexity
			\begin{align*}
				\sim O\left(
					e^{\sqrt[3]{\frac{64}{9}\log_2n\cdot(\log_2\log_2n)^2}}
				\right)
			\end{align*}

			Factoring a number with $300$ digits:
			\begin{itemize}
				\item Your laptop: $10^{13}$ billion years
				\item Best supercomputer: $13$ billion years
					(age of the universe)
			\end{itemize}
	\end{itemize}
\end{frame}

\begin{frame}{Symmetric and asymmetric cryptography}
	\begin{itemize}
		\item Our examples are \emph{asymmetric}: different public/private keys
		\item Safe against eavesdroppers
		\item Symmetric protocols can be faster and simpler, but you need
			a secure way to exchange a key
	\end{itemize}
\end{frame}

\begin{frame}{Diffie-Hellman key exchange}
	\begin{itemize}
		\item Generate a ``password'' without communicating it directly
		\item It can then be used for symmetric cryptography
		\item Based on a different hard problem:
			\href{https://en.wikipedia.org/wiki/Discrete\_logarithm}%
			{\emph{discrete logarithm}}
	\end{itemize}
\end{frame}

\begin{frame}{Diffie-Hellman key exchange}
	\begin{itemize}
		\item Alice and Bob agree on a prime number $p$ and an integer $g$
		\item Alice picks an integer $a$ and sends $(g^a\bmod p)$ to Bob
		\item Bob picks an integer $b$ and sends $(g^b\bmod p)$ to Alice
		\item Alice can compute $(g^b)^a\bmod p$ and Bob can compute
			$(g^a)^b\bmod p$. This is their shared secret (key).
	\end{itemize}
\end{frame}

\begin{frame}{Diffie-Hellman with colors (from Wikipedia)}
	\begin{center}\includesvg[scale=0.45]{img/DH}\end{center}
\end{frame}

\begin{frame}{Diffie-Hellman key exchange}
	\begin{itemize}
		\item Knowing $h$ and $a$, it is hard to find $g$ such that
			$g^a \bmod p =h$ (discrete logarithm problem)
		\item Very simple, many variants
		\item Any group can be used, e.g. Elliptic Curves (see
\href{https://en.wikipedia.org/wiki/Elliptic-curve_Diffie\%E2\%80\%93Hellman}%
			{Wikipedia: elliptic-curve Diffie-Hellman})
	\end{itemize}
\end{frame}


\begin{frame}[plain]
	\begin{center} {\Huge Numerical methods for PDEs} \end{center}
\end{frame}

\begin{frame}{Solving partial differential equations}
	\begin{itemize}
		\item Very, very hard
		\item Very important in practical applications (physics and such)
		\item Approximations are necessary, might as well use numerical methods
	\end{itemize}
\end{frame}

\begin{frame}{Numerical methods for ODEs}
	\begin{block}{Problem}
		Given $f(x,y)$, $x_0$ and $y_0$, find an approximation
		for $y(x)$ such that
		\begin{align*}
			\begin{cases}
				y'(x) = f(x, y(x))\\
				y(x_0) =y_0
			\end{cases}
		\end{align*}
	\end{block}

	\begin{block}{Approximation}
		We can describe $y(x)$ in an interval $[x_0,x_1]$ by giving the
		(approximate) values $y(s_0)$, \dots, $y(s_n)$ for many
		values of $s_i\in [x_0, x_1]$.
	\end{block}
\end{frame}

\begin{frame}{Euler's method}
	\begin{block}{Idea}
		For $h$ small
		\begin{align*}
			y'(x)\approx\frac{y(x+h)-y(x)}{h}
		\end{align*}
		which implies
		\begin{align*}
			y(x+h) \approx y(x) + h\cdot f(x, y(x))
		\end{align*}
	\end{block}
\end{frame}

\begin{frame}{Euler's method}
	\begin{block}{Algorithm}
		\textbf{Input:} the data $f(x,y)$, $x_0$, $y_0$ and $x_1$ describing
			the problem and the desired range for the solution.

		\vspace{0.3cm}
		\textbf{Output:} $x_0=s_0 < s_1 < \dots < s_n=x_1$ and
			$y_0, \dots, y_n$ such that $y_i\approx y(s_i)$.

		\vspace{0.3cm}
		\begin{enumerate}
			\item Choose a value $n$ and let
				$h=\frac{x_1-x_0}{n}$ and $s_i=x_0+ih$
			\item For $i=0,\dots, n-1$ compute
				$y_{i+1}=y_i+h\cdot f(s_i, y_i)$
			\item Return $s_0, \dots, s_n$ and $y_0, \dots, y_n$
		\end{enumerate}
	\end{block}
\end{frame}

\begin{frame}{Euler's method}
	\begin{itemize}
		\item Very simple and fast
		\item Generalization for higher-order equations: Runge-Kutta methods
		\item A similar idea works for some PDEs
	\end{itemize}
\end{frame}

\begin{frame}{The heat equation (PDE)}
	\begin{align*}
		\frac{\partial u}{\partial t} = \frac{\partial^2 u}{\partial x_1^2} +
		\frac{\partial^2 u}{\partial x_2^2} + \cdots +
		\frac{\partial^2 u}{\partial x_n^2}
	\end{align*}
	
	Where
	\[u(x_1,x_2,\dots,x_n,t): \mathbb R^n\times \mathbb R_+\to \mathbb R\]
	describes the quantity of heat at the point $(x_1,\dots x_n)$ at time $t$.

	\vspace{0.3cm} It appears also outside thermodynamics: mathematical finance
	(\href{https://en.wikipedia.org/wiki/Black\%E2\%80\%93Scholes\_equation}%
	{Black-Scholes equation}), quantum mechanics
	(\href{https://en.wikipedia.org/wiki/Schr\%C3\%B6dinger\_equation}%
	{Schrödinger equation}), image analysis\dots
\end{frame}

\begin{frame}{A simple case ($n=1$, in $[0,1]^2$)}
	\begin{block}{Problem}
		Given $u_0(t)$, $u_1(t)$ and $u^0(x)$, find an approximation
		for $u(x,t)$ such that
		\begin{align*}
			\begin{cases}
				\frac{\partial u}{\partial t} =
					\frac{\partial^2 u}{\partial x^2} \\
				u(0,t) = u_{(0)}(t) \quad \text{(boundary condition)}\\
				u(1,t) = u_{(1)}(t) \quad \text{(boundary condition)}\\
				u(x,0) = u^0(x) \quad \text{(initial condition)}
			\end{cases}
		\end{align*}
	\end{block}

	\begin{block}{Approximation}
		Values $u_i^j\approx u(s_i, r^j)$ for
		$(s_i,r^j)\in [0,1]\times [0,1]$
	\end{block}
\end{frame}

\begin{frame}{Idea}
	For $k$ small:
	\begin{align*}
		\frac{\partial u(x,t)}{\partial t} \approx \frac{u(x,t+k)-u(x,t)}{k}\\
	\end{align*}
	For $h$ small (left limit + right limit):
	\begin{align*}
		\frac{\partial^2 u(x,t)}{\partial x^2} &\approx
			\frac{\partial}{\partial x}\left(
				\frac{u(x,t) - u(x-h,t)}{h}
			\right)\\
		&\approx \frac1h\left(
			\frac{\partial u(x,t)}{\partial x} -
			\frac{\partial u(x-h,t)}{\partial x}
		\right)\\
		&\approx \frac1h\left(
			\frac{u(x+h,t) - u(x,t)}{h} - \frac{u(x,t)-u(x-h,t)}{h}
		\right)\\
		&\approx \frac{u(x+h,t)-2u(x,t)+u(x-h,t)}{h^2}
	\end{align*}
\end{frame}

\begin{frame}{Idea}
	From the equation
	\begin{align*}
		\frac{u_i^{j+1}-u_i^j}{k}= \frac{u_{i+1}^j-2u_{i}^j+u_{i-1}^j}{h^2}
	\end{align*}
	we find the formula
	\begin{align*}
		u_i^{j+1} = \frac{k}{h^2}\left(u_{i+1}^j - 2u_i^j + u_{i-1}^j\right)
			 + u_i^j
	\end{align*}
\end{frame}

\begin{frame}{Finite difference method for the heat equation}
	\begin{block}{Algorithm}
		\textbf{Input:} $u_{(0)}^j$, $u_{(1)}^j$ (boundary)
			and $u_i^0$ (initial).

		\vspace{0.3cm}
		\textbf{Output:} values $u_i^j$ approximating a solution.

		\vspace{0.3cm}
		\begin{enumerate}
			\item Let  $m=\operatorname{len}(u_0)-1$,
				$n=\operatorname{len}(u^0)-1$ and $k=1/m$, $h=1/n$
				%\begin{align*}
				%	\begin{array}{cccc}
				%		k=\frac{t_1-t_0}{m}, & h=\frac{x_1-x_0}{n}, &
				%		r^j = t_0 +jk, & s_i = x_0+ih
				%	\end{array}
				%\end{align*}
			\item For $j=0,\dots, m-1$ do the following:
				\begin{itemize}
					\item For $i=1,\dots, n-1$ compute
						\begin{align*}
							u_i^{j+1} = \frac{k}{h^2}\left(u_{i+1}^j - 
							             2u_i^j + u_{i-1}^j\right) + u_i^j
						\end{align*}
				\end{itemize}
			\item Return the $u_i^j$
		\end{enumerate}
	\end{block}
\end{frame}

\begin{frame}{Other PDEs}
	\begin{itemize}
		\item In general, there is no generic method
		\item You might need to write specific code for your equation
		\item Some packages exists
			(e.g. \href{https://wiki.octave.org/Fem-fenics}{fem-fenics} for
			\href{https://www.gnu.org/software/octave/index}{Gnu Octave})
	\end{itemize}
\end{frame}

\end{document}
