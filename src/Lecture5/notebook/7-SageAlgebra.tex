\documentclass[11pt]{article}

    \usepackage[breakable]{tcolorbox}
    \usepackage{parskip} % Stop auto-indenting (to mimic markdown behaviour)
    
    \usepackage{iftex}
    \ifPDFTeX
    	\usepackage[T1]{fontenc}
    	\usepackage{mathpazo}
    \else
    	\usepackage{fontspec}
    \fi

    % Basic figure setup, for now with no caption control since it's done
    % automatically by Pandoc (which extracts ![](path) syntax from Markdown).
    \usepackage{graphicx}
    % Maintain compatibility with old templates. Remove in nbconvert 6.0
    \let\Oldincludegraphics\includegraphics
    % Ensure that by default, figures have no caption (until we provide a
    % proper Figure object with a Caption API and a way to capture that
    % in the conversion process - todo).
    \usepackage{caption}
    \DeclareCaptionFormat{nocaption}{}
    \captionsetup{format=nocaption,aboveskip=0pt,belowskip=0pt}

    \usepackage[Export]{adjustbox} % Used to constrain images to a maximum size
    \adjustboxset{max size={0.9\linewidth}{0.9\paperheight}}
    \usepackage{float}
    \floatplacement{figure}{H} % forces figures to be placed at the correct location
    \usepackage{xcolor} % Allow colors to be defined
    \usepackage{enumerate} % Needed for markdown enumerations to work
    \usepackage{geometry} % Used to adjust the document margins
    \usepackage{amsmath} % Equations
    \usepackage{amssymb} % Equations
    \usepackage{textcomp} % defines textquotesingle
    % Hack from http://tex.stackexchange.com/a/47451/13684:
    \AtBeginDocument{%
        \def\PYZsq{\textquotesingle}% Upright quotes in Pygmentized code
    }
    \usepackage{upquote} % Upright quotes for verbatim code
    \usepackage{eurosym} % defines \euro
    \usepackage[mathletters]{ucs} % Extended unicode (utf-8) support
    \usepackage{fancyvrb} % verbatim replacement that allows latex
    \usepackage{grffile} % extends the file name processing of package graphics 
                         % to support a larger range
    \makeatletter % fix for grffile with XeLaTeX
    \def\Gread@@xetex#1{%
      \IfFileExists{"\Gin@base".bb}%
      {\Gread@eps{\Gin@base.bb}}%
      {\Gread@@xetex@aux#1}%
    }
    \makeatother

    % The hyperref package gives us a pdf with properly built
    % internal navigation ('pdf bookmarks' for the table of contents,
    % internal cross-reference links, web links for URLs, etc.)
    \usepackage{hyperref}
    % The default LaTeX title has an obnoxious amount of whitespace. By default,
    % titling removes some of it. It also provides customization options.
    \usepackage{titling}
    \usepackage{longtable} % longtable support required by pandoc >1.10
    \usepackage{booktabs}  % table support for pandoc > 1.12.2
    \usepackage[inline]{enumitem} % IRkernel/repr support (it uses the enumerate* environment)
    \usepackage[normalem]{ulem} % ulem is needed to support strikethroughs (\sout)
                                % normalem makes italics be italics, not underlines
    \usepackage{mathrsfs}
    

    
    % Colors for the hyperref package
    \definecolor{urlcolor}{rgb}{0,.145,.698}
    \definecolor{linkcolor}{rgb}{.71,0.21,0.01}
    \definecolor{citecolor}{rgb}{.12,.54,.11}

    % ANSI colors
    \definecolor{ansi-black}{HTML}{3E424D}
    \definecolor{ansi-black-intense}{HTML}{282C36}
    \definecolor{ansi-red}{HTML}{E75C58}
    \definecolor{ansi-red-intense}{HTML}{B22B31}
    \definecolor{ansi-green}{HTML}{00A250}
    \definecolor{ansi-green-intense}{HTML}{007427}
    \definecolor{ansi-yellow}{HTML}{DDB62B}
    \definecolor{ansi-yellow-intense}{HTML}{B27D12}
    \definecolor{ansi-blue}{HTML}{208FFB}
    \definecolor{ansi-blue-intense}{HTML}{0065CA}
    \definecolor{ansi-magenta}{HTML}{D160C4}
    \definecolor{ansi-magenta-intense}{HTML}{A03196}
    \definecolor{ansi-cyan}{HTML}{60C6C8}
    \definecolor{ansi-cyan-intense}{HTML}{258F8F}
    \definecolor{ansi-white}{HTML}{C5C1B4}
    \definecolor{ansi-white-intense}{HTML}{A1A6B2}
    \definecolor{ansi-default-inverse-fg}{HTML}{FFFFFF}
    \definecolor{ansi-default-inverse-bg}{HTML}{000000}

    % commands and environments needed by pandoc snippets
    % extracted from the output of `pandoc -s`
    \providecommand{\tightlist}{%
      \setlength{\itemsep}{0pt}\setlength{\parskip}{0pt}}
    \DefineVerbatimEnvironment{Highlighting}{Verbatim}{commandchars=\\\{\}}
    % Add ',fontsize=\small' for more characters per line
    \newenvironment{Shaded}{}{}
    \newcommand{\KeywordTok}[1]{\textcolor[rgb]{0.00,0.44,0.13}{\textbf{{#1}}}}
    \newcommand{\DataTypeTok}[1]{\textcolor[rgb]{0.56,0.13,0.00}{{#1}}}
    \newcommand{\DecValTok}[1]{\textcolor[rgb]{0.25,0.63,0.44}{{#1}}}
    \newcommand{\BaseNTok}[1]{\textcolor[rgb]{0.25,0.63,0.44}{{#1}}}
    \newcommand{\FloatTok}[1]{\textcolor[rgb]{0.25,0.63,0.44}{{#1}}}
    \newcommand{\CharTok}[1]{\textcolor[rgb]{0.25,0.44,0.63}{{#1}}}
    \newcommand{\StringTok}[1]{\textcolor[rgb]{0.25,0.44,0.63}{{#1}}}
    \newcommand{\CommentTok}[1]{\textcolor[rgb]{0.38,0.63,0.69}{\textit{{#1}}}}
    \newcommand{\OtherTok}[1]{\textcolor[rgb]{0.00,0.44,0.13}{{#1}}}
    \newcommand{\AlertTok}[1]{\textcolor[rgb]{1.00,0.00,0.00}{\textbf{{#1}}}}
    \newcommand{\FunctionTok}[1]{\textcolor[rgb]{0.02,0.16,0.49}{{#1}}}
    \newcommand{\RegionMarkerTok}[1]{{#1}}
    \newcommand{\ErrorTok}[1]{\textcolor[rgb]{1.00,0.00,0.00}{\textbf{{#1}}}}
    \newcommand{\NormalTok}[1]{{#1}}
    
    % Additional commands for more recent versions of Pandoc
    \newcommand{\ConstantTok}[1]{\textcolor[rgb]{0.53,0.00,0.00}{{#1}}}
    \newcommand{\SpecialCharTok}[1]{\textcolor[rgb]{0.25,0.44,0.63}{{#1}}}
    \newcommand{\VerbatimStringTok}[1]{\textcolor[rgb]{0.25,0.44,0.63}{{#1}}}
    \newcommand{\SpecialStringTok}[1]{\textcolor[rgb]{0.73,0.40,0.53}{{#1}}}
    \newcommand{\ImportTok}[1]{{#1}}
    \newcommand{\DocumentationTok}[1]{\textcolor[rgb]{0.73,0.13,0.13}{\textit{{#1}}}}
    \newcommand{\AnnotationTok}[1]{\textcolor[rgb]{0.38,0.63,0.69}{\textbf{\textit{{#1}}}}}
    \newcommand{\CommentVarTok}[1]{\textcolor[rgb]{0.38,0.63,0.69}{\textbf{\textit{{#1}}}}}
    \newcommand{\VariableTok}[1]{\textcolor[rgb]{0.10,0.09,0.49}{{#1}}}
    \newcommand{\ControlFlowTok}[1]{\textcolor[rgb]{0.00,0.44,0.13}{\textbf{{#1}}}}
    \newcommand{\OperatorTok}[1]{\textcolor[rgb]{0.40,0.40,0.40}{{#1}}}
    \newcommand{\BuiltInTok}[1]{{#1}}
    \newcommand{\ExtensionTok}[1]{{#1}}
    \newcommand{\PreprocessorTok}[1]{\textcolor[rgb]{0.74,0.48,0.00}{{#1}}}
    \newcommand{\AttributeTok}[1]{\textcolor[rgb]{0.49,0.56,0.16}{{#1}}}
    \newcommand{\InformationTok}[1]{\textcolor[rgb]{0.38,0.63,0.69}{\textbf{\textit{{#1}}}}}
    \newcommand{\WarningTok}[1]{\textcolor[rgb]{0.38,0.63,0.69}{\textbf{\textit{{#1}}}}}
    
    
    % Define a nice break command that doesn't care if a line doesn't already
    % exist.
    \def\br{\hspace*{\fill} \\* }
    % Math Jax compatibility definitions
    \def\gt{>}
    \def\lt{<}
    \let\Oldtex\TeX
    \let\Oldlatex\LaTeX
    \renewcommand{\TeX}{\textrm{\Oldtex}}
    \renewcommand{\LaTeX}{\textrm{\Oldlatex}}
    % Document parameters
    % Document title
    \title{Algebra and Cryptography with SageMath}
    \date{2021-04-23}
    \author{Sebastiano Tronto - \texttt{sebastiano.tronto@uni.lu}}
    
    
    
    
    
% Pygments definitions
\makeatletter
\def\PY@reset{\let\PY@it=\relax \let\PY@bf=\relax%
    \let\PY@ul=\relax \let\PY@tc=\relax%
    \let\PY@bc=\relax \let\PY@ff=\relax}
\def\PY@tok#1{\csname PY@tok@#1\endcsname}
\def\PY@toks#1+{\ifx\relax#1\empty\else%
    \PY@tok{#1}\expandafter\PY@toks\fi}
\def\PY@do#1{\PY@bc{\PY@tc{\PY@ul{%
    \PY@it{\PY@bf{\PY@ff{#1}}}}}}}
\def\PY#1#2{\PY@reset\PY@toks#1+\relax+\PY@do{#2}}

\expandafter\def\csname PY@tok@w\endcsname{\def\PY@tc##1{\textcolor[rgb]{0.73,0.73,0.73}{##1}}}
\expandafter\def\csname PY@tok@c\endcsname{\let\PY@it=\textit\def\PY@tc##1{\textcolor[rgb]{0.25,0.50,0.50}{##1}}}
\expandafter\def\csname PY@tok@cp\endcsname{\def\PY@tc##1{\textcolor[rgb]{0.74,0.48,0.00}{##1}}}
\expandafter\def\csname PY@tok@k\endcsname{\let\PY@bf=\textbf\def\PY@tc##1{\textcolor[rgb]{0.00,0.50,0.00}{##1}}}
\expandafter\def\csname PY@tok@kp\endcsname{\def\PY@tc##1{\textcolor[rgb]{0.00,0.50,0.00}{##1}}}
\expandafter\def\csname PY@tok@kt\endcsname{\def\PY@tc##1{\textcolor[rgb]{0.69,0.00,0.25}{##1}}}
\expandafter\def\csname PY@tok@o\endcsname{\def\PY@tc##1{\textcolor[rgb]{0.40,0.40,0.40}{##1}}}
\expandafter\def\csname PY@tok@ow\endcsname{\let\PY@bf=\textbf\def\PY@tc##1{\textcolor[rgb]{0.67,0.13,1.00}{##1}}}
\expandafter\def\csname PY@tok@nb\endcsname{\def\PY@tc##1{\textcolor[rgb]{0.00,0.50,0.00}{##1}}}
\expandafter\def\csname PY@tok@nf\endcsname{\def\PY@tc##1{\textcolor[rgb]{0.00,0.00,1.00}{##1}}}
\expandafter\def\csname PY@tok@nc\endcsname{\let\PY@bf=\textbf\def\PY@tc##1{\textcolor[rgb]{0.00,0.00,1.00}{##1}}}
\expandafter\def\csname PY@tok@nn\endcsname{\let\PY@bf=\textbf\def\PY@tc##1{\textcolor[rgb]{0.00,0.00,1.00}{##1}}}
\expandafter\def\csname PY@tok@ne\endcsname{\let\PY@bf=\textbf\def\PY@tc##1{\textcolor[rgb]{0.82,0.25,0.23}{##1}}}
\expandafter\def\csname PY@tok@nv\endcsname{\def\PY@tc##1{\textcolor[rgb]{0.10,0.09,0.49}{##1}}}
\expandafter\def\csname PY@tok@no\endcsname{\def\PY@tc##1{\textcolor[rgb]{0.53,0.00,0.00}{##1}}}
\expandafter\def\csname PY@tok@nl\endcsname{\def\PY@tc##1{\textcolor[rgb]{0.63,0.63,0.00}{##1}}}
\expandafter\def\csname PY@tok@ni\endcsname{\let\PY@bf=\textbf\def\PY@tc##1{\textcolor[rgb]{0.60,0.60,0.60}{##1}}}
\expandafter\def\csname PY@tok@na\endcsname{\def\PY@tc##1{\textcolor[rgb]{0.49,0.56,0.16}{##1}}}
\expandafter\def\csname PY@tok@nt\endcsname{\let\PY@bf=\textbf\def\PY@tc##1{\textcolor[rgb]{0.00,0.50,0.00}{##1}}}
\expandafter\def\csname PY@tok@nd\endcsname{\def\PY@tc##1{\textcolor[rgb]{0.67,0.13,1.00}{##1}}}
\expandafter\def\csname PY@tok@s\endcsname{\def\PY@tc##1{\textcolor[rgb]{0.73,0.13,0.13}{##1}}}
\expandafter\def\csname PY@tok@sd\endcsname{\let\PY@it=\textit\def\PY@tc##1{\textcolor[rgb]{0.73,0.13,0.13}{##1}}}
\expandafter\def\csname PY@tok@si\endcsname{\let\PY@bf=\textbf\def\PY@tc##1{\textcolor[rgb]{0.73,0.40,0.53}{##1}}}
\expandafter\def\csname PY@tok@se\endcsname{\let\PY@bf=\textbf\def\PY@tc##1{\textcolor[rgb]{0.73,0.40,0.13}{##1}}}
\expandafter\def\csname PY@tok@sr\endcsname{\def\PY@tc##1{\textcolor[rgb]{0.73,0.40,0.53}{##1}}}
\expandafter\def\csname PY@tok@ss\endcsname{\def\PY@tc##1{\textcolor[rgb]{0.10,0.09,0.49}{##1}}}
\expandafter\def\csname PY@tok@sx\endcsname{\def\PY@tc##1{\textcolor[rgb]{0.00,0.50,0.00}{##1}}}
\expandafter\def\csname PY@tok@m\endcsname{\def\PY@tc##1{\textcolor[rgb]{0.40,0.40,0.40}{##1}}}
\expandafter\def\csname PY@tok@gh\endcsname{\let\PY@bf=\textbf\def\PY@tc##1{\textcolor[rgb]{0.00,0.00,0.50}{##1}}}
\expandafter\def\csname PY@tok@gu\endcsname{\let\PY@bf=\textbf\def\PY@tc##1{\textcolor[rgb]{0.50,0.00,0.50}{##1}}}
\expandafter\def\csname PY@tok@gd\endcsname{\def\PY@tc##1{\textcolor[rgb]{0.63,0.00,0.00}{##1}}}
\expandafter\def\csname PY@tok@gi\endcsname{\def\PY@tc##1{\textcolor[rgb]{0.00,0.63,0.00}{##1}}}
\expandafter\def\csname PY@tok@gr\endcsname{\def\PY@tc##1{\textcolor[rgb]{1.00,0.00,0.00}{##1}}}
\expandafter\def\csname PY@tok@ge\endcsname{\let\PY@it=\textit}
\expandafter\def\csname PY@tok@gs\endcsname{\let\PY@bf=\textbf}
\expandafter\def\csname PY@tok@gp\endcsname{\let\PY@bf=\textbf\def\PY@tc##1{\textcolor[rgb]{0.00,0.00,0.50}{##1}}}
\expandafter\def\csname PY@tok@go\endcsname{\def\PY@tc##1{\textcolor[rgb]{0.53,0.53,0.53}{##1}}}
\expandafter\def\csname PY@tok@gt\endcsname{\def\PY@tc##1{\textcolor[rgb]{0.00,0.27,0.87}{##1}}}
\expandafter\def\csname PY@tok@err\endcsname{\def\PY@bc##1{\setlength{\fboxsep}{0pt}\fcolorbox[rgb]{1.00,0.00,0.00}{1,1,1}{\strut ##1}}}
\expandafter\def\csname PY@tok@kc\endcsname{\let\PY@bf=\textbf\def\PY@tc##1{\textcolor[rgb]{0.00,0.50,0.00}{##1}}}
\expandafter\def\csname PY@tok@kd\endcsname{\let\PY@bf=\textbf\def\PY@tc##1{\textcolor[rgb]{0.00,0.50,0.00}{##1}}}
\expandafter\def\csname PY@tok@kn\endcsname{\let\PY@bf=\textbf\def\PY@tc##1{\textcolor[rgb]{0.00,0.50,0.00}{##1}}}
\expandafter\def\csname PY@tok@kr\endcsname{\let\PY@bf=\textbf\def\PY@tc##1{\textcolor[rgb]{0.00,0.50,0.00}{##1}}}
\expandafter\def\csname PY@tok@bp\endcsname{\def\PY@tc##1{\textcolor[rgb]{0.00,0.50,0.00}{##1}}}
\expandafter\def\csname PY@tok@fm\endcsname{\def\PY@tc##1{\textcolor[rgb]{0.00,0.00,1.00}{##1}}}
\expandafter\def\csname PY@tok@vc\endcsname{\def\PY@tc##1{\textcolor[rgb]{0.10,0.09,0.49}{##1}}}
\expandafter\def\csname PY@tok@vg\endcsname{\def\PY@tc##1{\textcolor[rgb]{0.10,0.09,0.49}{##1}}}
\expandafter\def\csname PY@tok@vi\endcsname{\def\PY@tc##1{\textcolor[rgb]{0.10,0.09,0.49}{##1}}}
\expandafter\def\csname PY@tok@vm\endcsname{\def\PY@tc##1{\textcolor[rgb]{0.10,0.09,0.49}{##1}}}
\expandafter\def\csname PY@tok@sa\endcsname{\def\PY@tc##1{\textcolor[rgb]{0.73,0.13,0.13}{##1}}}
\expandafter\def\csname PY@tok@sb\endcsname{\def\PY@tc##1{\textcolor[rgb]{0.73,0.13,0.13}{##1}}}
\expandafter\def\csname PY@tok@sc\endcsname{\def\PY@tc##1{\textcolor[rgb]{0.73,0.13,0.13}{##1}}}
\expandafter\def\csname PY@tok@dl\endcsname{\def\PY@tc##1{\textcolor[rgb]{0.73,0.13,0.13}{##1}}}
\expandafter\def\csname PY@tok@s2\endcsname{\def\PY@tc##1{\textcolor[rgb]{0.73,0.13,0.13}{##1}}}
\expandafter\def\csname PY@tok@sh\endcsname{\def\PY@tc##1{\textcolor[rgb]{0.73,0.13,0.13}{##1}}}
\expandafter\def\csname PY@tok@s1\endcsname{\def\PY@tc##1{\textcolor[rgb]{0.73,0.13,0.13}{##1}}}
\expandafter\def\csname PY@tok@mb\endcsname{\def\PY@tc##1{\textcolor[rgb]{0.40,0.40,0.40}{##1}}}
\expandafter\def\csname PY@tok@mf\endcsname{\def\PY@tc##1{\textcolor[rgb]{0.40,0.40,0.40}{##1}}}
\expandafter\def\csname PY@tok@mh\endcsname{\def\PY@tc##1{\textcolor[rgb]{0.40,0.40,0.40}{##1}}}
\expandafter\def\csname PY@tok@mi\endcsname{\def\PY@tc##1{\textcolor[rgb]{0.40,0.40,0.40}{##1}}}
\expandafter\def\csname PY@tok@il\endcsname{\def\PY@tc##1{\textcolor[rgb]{0.40,0.40,0.40}{##1}}}
\expandafter\def\csname PY@tok@mo\endcsname{\def\PY@tc##1{\textcolor[rgb]{0.40,0.40,0.40}{##1}}}
\expandafter\def\csname PY@tok@ch\endcsname{\let\PY@it=\textit\def\PY@tc##1{\textcolor[rgb]{0.25,0.50,0.50}{##1}}}
\expandafter\def\csname PY@tok@cm\endcsname{\let\PY@it=\textit\def\PY@tc##1{\textcolor[rgb]{0.25,0.50,0.50}{##1}}}
\expandafter\def\csname PY@tok@cpf\endcsname{\let\PY@it=\textit\def\PY@tc##1{\textcolor[rgb]{0.25,0.50,0.50}{##1}}}
\expandafter\def\csname PY@tok@c1\endcsname{\let\PY@it=\textit\def\PY@tc##1{\textcolor[rgb]{0.25,0.50,0.50}{##1}}}
\expandafter\def\csname PY@tok@cs\endcsname{\let\PY@it=\textit\def\PY@tc##1{\textcolor[rgb]{0.25,0.50,0.50}{##1}}}

\def\PYZbs{\char`\\}
\def\PYZus{\char`\_}
\def\PYZob{\char`\{}
\def\PYZcb{\char`\}}
\def\PYZca{\char`\^}
\def\PYZam{\char`\&}
\def\PYZlt{\char`\<}
\def\PYZgt{\char`\>}
\def\PYZsh{\char`\#}
\def\PYZpc{\char`\%}
\def\PYZdl{\char`\$}
\def\PYZhy{\char`\-}
\def\PYZsq{\char`\'}
\def\PYZdq{\char`\"}
\def\PYZti{\char`\~}
% for compatibility with earlier versions
\def\PYZat{@}
\def\PYZlb{[}
\def\PYZrb{]}
\makeatother


    % For linebreaks inside Verbatim environment from package fancyvrb. 
    \makeatletter
        \newbox\Wrappedcontinuationbox 
        \newbox\Wrappedvisiblespacebox 
        \newcommand*\Wrappedvisiblespace {\textcolor{red}{\textvisiblespace}} 
        \newcommand*\Wrappedcontinuationsymbol {\textcolor{red}{\llap{\tiny$\m@th\hookrightarrow$}}} 
        \newcommand*\Wrappedcontinuationindent {3ex } 
        \newcommand*\Wrappedafterbreak {\kern\Wrappedcontinuationindent\copy\Wrappedcontinuationbox} 
        % Take advantage of the already applied Pygments mark-up to insert 
        % potential linebreaks for TeX processing. 
        %        {, <, #, %, $, ' and ": go to next line. 
        %        _, }, ^, &, >, - and ~: stay at end of broken line. 
        % Use of \textquotesingle for straight quote. 
        \newcommand*\Wrappedbreaksatspecials {% 
            \def\PYGZus{\discretionary{\char`\_}{\Wrappedafterbreak}{\char`\_}}% 
            \def\PYGZob{\discretionary{}{\Wrappedafterbreak\char`\{}{\char`\{}}% 
            \def\PYGZcb{\discretionary{\char`\}}{\Wrappedafterbreak}{\char`\}}}% 
            \def\PYGZca{\discretionary{\char`\^}{\Wrappedafterbreak}{\char`\^}}% 
            \def\PYGZam{\discretionary{\char`\&}{\Wrappedafterbreak}{\char`\&}}% 
            \def\PYGZlt{\discretionary{}{\Wrappedafterbreak\char`\<}{\char`\<}}% 
            \def\PYGZgt{\discretionary{\char`\>}{\Wrappedafterbreak}{\char`\>}}% 
            \def\PYGZsh{\discretionary{}{\Wrappedafterbreak\char`\#}{\char`\#}}% 
            \def\PYGZpc{\discretionary{}{\Wrappedafterbreak\char`\%}{\char`\%}}% 
            \def\PYGZdl{\discretionary{}{\Wrappedafterbreak\char`\$}{\char`\$}}% 
            \def\PYGZhy{\discretionary{\char`\-}{\Wrappedafterbreak}{\char`\-}}% 
            \def\PYGZsq{\discretionary{}{\Wrappedafterbreak\textquotesingle}{\textquotesingle}}% 
            \def\PYGZdq{\discretionary{}{\Wrappedafterbreak\char`\"}{\char`\"}}% 
            \def\PYGZti{\discretionary{\char`\~}{\Wrappedafterbreak}{\char`\~}}% 
        } 
        % Some characters . , ; ? ! / are not pygmentized. 
        % This macro makes them "active" and they will insert potential linebreaks 
        \newcommand*\Wrappedbreaksatpunct {% 
            \lccode`\~`\.\lowercase{\def~}{\discretionary{\hbox{\char`\.}}{\Wrappedafterbreak}{\hbox{\char`\.}}}% 
            \lccode`\~`\,\lowercase{\def~}{\discretionary{\hbox{\char`\,}}{\Wrappedafterbreak}{\hbox{\char`\,}}}% 
            \lccode`\~`\;\lowercase{\def~}{\discretionary{\hbox{\char`\;}}{\Wrappedafterbreak}{\hbox{\char`\;}}}% 
            \lccode`\~`\:\lowercase{\def~}{\discretionary{\hbox{\char`\:}}{\Wrappedafterbreak}{\hbox{\char`\:}}}% 
            \lccode`\~`\?\lowercase{\def~}{\discretionary{\hbox{\char`\?}}{\Wrappedafterbreak}{\hbox{\char`\?}}}% 
            \lccode`\~`\!\lowercase{\def~}{\discretionary{\hbox{\char`\!}}{\Wrappedafterbreak}{\hbox{\char`\!}}}% 
            \lccode`\~`\/\lowercase{\def~}{\discretionary{\hbox{\char`\/}}{\Wrappedafterbreak}{\hbox{\char`\/}}}% 
            \catcode`\.\active
            \catcode`\,\active 
            \catcode`\;\active
            \catcode`\:\active
            \catcode`\?\active
            \catcode`\!\active
            \catcode`\/\active 
            \lccode`\~`\~ 	
        }
    \makeatother

    \let\OriginalVerbatim=\Verbatim
    \makeatletter
    \renewcommand{\Verbatim}[1][1]{%
        %\parskip\z@skip
        \sbox\Wrappedcontinuationbox {\Wrappedcontinuationsymbol}%
        \sbox\Wrappedvisiblespacebox {\FV@SetupFont\Wrappedvisiblespace}%
        \def\FancyVerbFormatLine ##1{\hsize\linewidth
            \vtop{\raggedright\hyphenpenalty\z@\exhyphenpenalty\z@
                \doublehyphendemerits\z@\finalhyphendemerits\z@
                \strut ##1\strut}%
        }%
        % If the linebreak is at a space, the latter will be displayed as visible
        % space at end of first line, and a continuation symbol starts next line.
        % Stretch/shrink are however usually zero for typewriter font.
        \def\FV@Space {%
            \nobreak\hskip\z@ plus\fontdimen3\font minus\fontdimen4\font
            \discretionary{\copy\Wrappedvisiblespacebox}{\Wrappedafterbreak}
            {\kern\fontdimen2\font}%
        }%
        
        % Allow breaks at special characters using \PYG... macros.
        \Wrappedbreaksatspecials
        % Breaks at punctuation characters . , ; ? ! and / need catcode=\active 	
        \OriginalVerbatim[#1,codes*=\Wrappedbreaksatpunct]%
    }
    \makeatother

    % Exact colors from NB
    \definecolor{incolor}{HTML}{303F9F}
    \definecolor{outcolor}{HTML}{D84315}
    \definecolor{cellborder}{HTML}{CFCFCF}
    \definecolor{cellbackground}{HTML}{F7F7F7}
    
    % prompt
    \makeatletter
    \newcommand{\boxspacing}{\kern\kvtcb@left@rule\kern\kvtcb@boxsep}
    \makeatother
    \newcommand{\prompt}[4]{
        \ttfamily\llap{{\color{#2}[#3]:\hspace{3pt}#4}}\vspace{-\baselineskip}
    }
    

    
    % Prevent overflowing lines due to hard-to-break entities
    \sloppy 
    % Setup hyperref package
    \hypersetup{
      breaklinks=true,  % so long urls are correctly broken across lines
      colorlinks=true,
      urlcolor=urlcolor,
      linkcolor=linkcolor,
      citecolor=citecolor,
      }
    % Slightly bigger margins than the latex defaults
    
    \geometry{verbose,tmargin=1in,bmargin=1in,lmargin=1in,rmargin=1in}
    
    

\begin{document}
    
    \maketitle
    
    

    
    This lecture's notes are in a different format: the presentations for
the \(\LaTeX\) part were made with \(\LaTeX\), so this one is made with
Sage, or rather with the \href{https://jupyter.org/}{Jupyter Notebook}.

\hypertarget{the-jupyter-notebook}{%
\section{The Jupyter Notebook}\label{the-jupyter-notebook}}

\textbf{Reference:} {[}\href{https://jupyter.org/documentation}{1}{]}

The Jupyter Notebook is one of the default interfaces for SageMath,
along with the command line interface. You can access it via web
browser, but it is running locally on your device (notice the strange
url: \texttt{http://localhost:8888/notebooks...}).

You can create a new notebook by clicking on
\texttt{New\ \textgreater{}\ SageMath\ 9.2}. You can also create a
Python 3 notebook to write Python code.

Jupyter saves and reads files in the \texttt{.ipynb} format. If you
download the file for this lecture you can open it and follow the
examples interactively.

\hypertarget{cells}{%
\subsection{Cells}\label{cells}}

The notebook contains one or more \emph{interactive cells} that you can
run, like this one below:

    \begin{tcolorbox}[breakable, size=fbox, boxrule=1pt, pad at break*=1mm,colback=cellbackground, colframe=cellborder]
\prompt{In}{incolor}{2}{\boxspacing}
\begin{Verbatim}[commandchars=\\\{\}]
\PY{c+c1}{\PYZsh{} Exercise: modify this cell to use the print() command}
\PY{l+m+mi}{2}\PY{o}{+}\PY{l+m+mi}{2}
\PY{l+m+mi}{2}\PY{o}{/}\PY{l+m+mi}{5}
\end{Verbatim}
\end{tcolorbox}

            \begin{tcolorbox}[breakable, size=fbox, boxrule=.5pt, pad at break*=1mm, opacityfill=0]
\prompt{Out}{outcolor}{2}{\boxspacing}
\begin{Verbatim}[commandchars=\\\{\}]
2/5
\end{Verbatim}
\end{tcolorbox}
        
    If you are reading this from Jupyter rather than from the pdf file, you
can edit the cell above and run it again. You can also add more cells by
selecting \texttt{Insert} from the menu bar.

Notice that only the last statement produces an output. You can force
anything to be written as output with the \texttt{print()} command,
which works like in Python. As an exercise, try to modify the cell above
to provide more output!

    \hypertarget{markdown}{%
\subsection{Markdown}\label{markdown}}

\href{https://en.wikipedia.org/wiki/Markdown}{Markdown} is a simple
markup language - think of LaTeX or html, but much simpler. You can add
text to your notebook with Markdown cells by selecting
\texttt{Cell\ \textgreater{}\ Cell\ Type\ \textgreater{}\ Markdown}.

You can also include some LaTeX code in Markdown cells, with dollar
signs \$ or align environments:

\begin{align*}
\frac{(x+y)^2}{x+1} = \frac{x^2+y^2}{x+1}
\end{align*}

When you are done writing a Markdown cell, you can run it to see the
well-formatted text. To edit the text again, double-click on the cell.
Try doing it now to fix the formula above!

    \hypertarget{symbolic-expressions}{%
\section{Symbolic expressions}\label{symbolic-expressions}}

\textbf{Reference:}
{[}\href{https://doc.sagemath.org/html/en/reference/calculus/sage/symbolic/expression.html}{2}{]}

Now, let's get started with Sage. One thing you might want to do is
manipulating symbolic expressions, like the following:

    \begin{tcolorbox}[breakable, size=fbox, boxrule=1pt, pad at break*=1mm,colback=cellbackground, colframe=cellborder]
\prompt{In}{incolor}{3}{\boxspacing}
\begin{Verbatim}[commandchars=\\\{\}]
\PY{n}{f} \PY{o}{=}  \PY{n}{x}\PY{o}{\PYZca{}}\PY{l+m+mi}{2} \PY{o}{+} \PY{l+m+mi}{2}\PY{o}{*}\PY{n}{x} \PY{o}{\PYZhy{}} \PY{l+m+mi}{5} \PY{o}{==} \PY{l+m+mi}{0}
\PY{n}{solve}\PY{p}{(}\PY{n}{f}\PY{p}{,}\PY{n}{x}\PY{p}{)}
\end{Verbatim}
\end{tcolorbox}

            \begin{tcolorbox}[breakable, size=fbox, boxrule=.5pt, pad at break*=1mm, opacityfill=0]
\prompt{Out}{outcolor}{3}{\boxspacing}
\begin{Verbatim}[commandchars=\\\{\}]
[x == -sqrt(6) - 1, x == sqrt(6) - 1]
\end{Verbatim}
\end{tcolorbox}
        
    Notice that the single \texttt{=} is part of an assignment, as in
Python: we are \emph{assigning} to the variable \texttt{f} the value
\texttt{x\^{}2\ +\ 2*x\ -\ 5\ \textgreater{}=\ 0}, which in this case is
an equation, so it contains the symbol \texttt{==}. Keep in mind the
difference between the two!

\textbf{Exercise:} change the code above to solve the corresponding
inequality \(x^2+2x-5\geq 0\).

    \hypertarget{mathematical-variables}{%
\subsection{Mathematical variables}\label{mathematical-variables}}

Last time we saw what \emph{variables} are in Python, and that they are
a little bit different from the \emph{Mathematical variables} that you
use in Mathematics. In Sage, both concepts are present, but they are
still distinct. For example in the cell above \texttt{f} is a variable
in the sense of computer science, while \texttt{x} is a Mathematical
variable.

If you want to use Mathematical variables other than \texttt{x}, you
first need to \emph{declare} them with the \texttt{var()} command:

    \begin{tcolorbox}[breakable, size=fbox, boxrule=1pt, pad at break*=1mm,colback=cellbackground, colframe=cellborder]
\prompt{In}{incolor}{14}{\boxspacing}
\begin{Verbatim}[commandchars=\\\{\}]
\PY{n}{var}\PY{p}{(}\PY{l+s+s1}{\PYZsq{}}\PY{l+s+s1}{y}\PY{l+s+s1}{\PYZsq{}}\PY{p}{)}
\PY{n}{solve}\PY{p}{(}\PY{n}{y}\PY{o}{\PYZca{}}\PY{l+m+mi}{2} \PY{o}{+} \PY{p}{(}\PY{n}{x}\PY{o}{+}\PY{l+m+mi}{1}\PY{p}{)}\PY{o}{*}\PY{n}{y} \PY{o}{\PYZhy{}} \PY{l+m+mi}{2} \PY{o}{==} \PY{l+m+mi}{0}\PY{p}{,} \PY{n}{y}\PY{p}{)}
\end{Verbatim}
\end{tcolorbox}

            \begin{tcolorbox}[breakable, size=fbox, boxrule=.5pt, pad at break*=1mm, opacityfill=0]
\prompt{Out}{outcolor}{14}{\boxspacing}
\begin{Verbatim}[commandchars=\\\{\}]
[y == -1/2*x - 1/2*sqrt(x\^{}2 + 2*x + 9) - 1/2, y == -1/2*x + 1/2*sqrt(x\^{}2 + 2*x +
9) - 1/2]
\end{Verbatim}
\end{tcolorbox}
        
    Try removing the first line in the cell above and see what error you
get!

Here is another example:

    \begin{tcolorbox}[breakable, size=fbox, boxrule=1pt, pad at break*=1mm,colback=cellbackground, colframe=cellborder]
\prompt{In}{incolor}{16}{\boxspacing}
\begin{Verbatim}[commandchars=\\\{\}]
\PY{n}{var}\PY{p}{(}\PY{l+s+s1}{\PYZsq{}}\PY{l+s+s1}{a}\PY{l+s+s1}{\PYZsq{}}\PY{p}{,} \PY{l+s+s1}{\PYZsq{}}\PY{l+s+s1}{b}\PY{l+s+s1}{\PYZsq{}}\PY{p}{)}
\PY{n}{f} \PY{o}{=} \PY{n}{x}\PY{o}{\PYZca{}}\PY{l+m+mi}{2}\PY{o}{+}\PY{n}{a}\PY{o}{*}\PY{n}{x}\PY{o}{+}\PY{n}{b}
\PY{n}{solve}\PY{p}{(}\PY{n}{f}\PY{p}{,}\PY{n}{x}\PY{p}{)}
\end{Verbatim}
\end{tcolorbox}

            \begin{tcolorbox}[breakable, size=fbox, boxrule=.5pt, pad at break*=1mm, opacityfill=0]
\prompt{Out}{outcolor}{16}{\boxspacing}
\begin{Verbatim}[commandchars=\\\{\}]
[x == -1/2*a - 1/2*sqrt(a\^{}2 - 4*b), x == -1/2*a + 1/2*sqrt(a\^{}2 - 4*b)]
\end{Verbatim}
\end{tcolorbox}
        
    Some common constants are
\href{https://doc.sagemath.org/html/en/reference/calculus/sage/symbolic/expression.html}{already
defined} in Sage:

    \begin{tcolorbox}[breakable, size=fbox, boxrule=1pt, pad at break*=1mm,colback=cellbackground, colframe=cellborder]
\prompt{In}{incolor}{17}{\boxspacing}
\begin{Verbatim}[commandchars=\\\{\}]
\PY{n}{e}\PY{o}{\PYZca{}}\PY{p}{(}\PY{n}{pi}\PY{o}{*}\PY{n}{I}\PY{p}{)}
\end{Verbatim}
\end{tcolorbox}

            \begin{tcolorbox}[breakable, size=fbox, boxrule=.5pt, pad at break*=1mm, opacityfill=0]
\prompt{Out}{outcolor}{17}{\boxspacing}
\begin{Verbatim}[commandchars=\\\{\}]
-1
\end{Verbatim}
\end{tcolorbox}
        
    We will study symbolic expressions more in detail next time, in the
context of calculus/analysis.

    \hypertarget{basic-rings-and-fields}{%
\section{Basic rings and fields}\label{basic-rings-and-fields}}

\textbf{References:}
{[}\href{https://doc.sagemath.org/html/en/reference/rings_standard/index.html}{3}{]}
{[}\href{https://doc.sagemath.org/html/en/reference/rings_numerical/index.html}{4}{]}
{[}\href{https://doc.sagemath.org/html/en/reference/finite_rings/index.html}{5}{]}

As you should know, a \emph{field} is a Mathematical structure with two
operations, addition and multiplication, which respect certain rules
(distributivity, associativity, commutativity\ldots). Some examples of
fields are the Rational numbers \(\mathbb Q\), the Real numbers
\(\mathbb R\) and the Complex numbers \(\mathbb C\), but there are many
more. As you should also know, a \emph{(commutative) ring} is like a
field, except not all elements different from \(0\) need have a
multiplicative inverse. For example the integers
\(\mathbb Z = \{ \dots, -1, 0, 1, 2, \dots\}\) are a ring, but not a
field.

These structures are already implemented in Sage. Some of the most
common are listed in the following table:

\begin{longtable}[]{@{}rcl@{}}
\toprule
Mathematical object & Math symbol & Sage name \\
\midrule
\endhead
Integers & \(\mathbb Z\) & \texttt{ZZ} \\
Rational numbers & \(\mathbb Q\) & \texttt{QQ} \\
Real numbers & \(\mathbb R\) & \texttt{RR} \\
Complex numbers & \(\mathbb C\) & \texttt{CC} \\
Integers modulo \(n\) & \(\mathbb Z/n\mathbb Z\) &
\texttt{Integers(n)} \\
Finite fields & \(\mathbb F_p\) & GF(p) \\
\(\dots\) & \(\dots\) & \(\dots\) \\
\bottomrule
\end{longtable}

    If you write a number or an expression, Sage will figure out where it
``lives'', choosing the most restrictive interpretation possible. For
example \texttt{3} will be interpreted to be an integer, even if it is
also a rational number, a real number and a complex number.

    \hypertarget{parents-and-coercion}{%
\subsection{Parents and coercion}\label{parents-and-coercion}}

\textbf{Reference:}
{[}\href{https://doc.sagemath.org/html/en/tutorial/tour_coercion.html}{6}{]}

You can check where an object ``lives'' with the \texttt{parent()}
command. It works more or less like the Python command \texttt{type()},
but it gives a more Mathematically inclined answer. Check the reference
link {[}6{]} above if you want more details.

    \begin{tcolorbox}[breakable, size=fbox, boxrule=1pt, pad at break*=1mm,colback=cellbackground, colframe=cellborder]
\prompt{In}{incolor}{18}{\boxspacing}
\begin{Verbatim}[commandchars=\\\{\}]
\PY{c+c1}{\PYZsh{}Edit this cell to find out the type of other objects that we used}
\PY{n}{parent}\PY{p}{(}\PY{l+m+mi}{3}\PY{o}{/}\PY{l+m+mi}{5}\PY{p}{)}
\end{Verbatim}
\end{tcolorbox}

            \begin{tcolorbox}[breakable, size=fbox, boxrule=.5pt, pad at break*=1mm, opacityfill=0]
\prompt{Out}{outcolor}{18}{\boxspacing}
\begin{Verbatim}[commandchars=\\\{\}]
Rational Field
\end{Verbatim}
\end{tcolorbox}
        
    Sometimes Sage does not give you the best possible interpretation, so
you can force something to be interpreted as living in a smaller ring as
follows:

    \begin{tcolorbox}[breakable, size=fbox, boxrule=1pt, pad at break*=1mm,colback=cellbackground, colframe=cellborder]
\prompt{In}{incolor}{4}{\boxspacing}
\begin{Verbatim}[commandchars=\\\{\}]
\PY{n}{minus\PYZus{}one} \PY{o}{=} \PY{n}{e}\PY{o}{\PYZca{}}\PY{p}{(}\PY{n}{pi}\PY{o}{*}\PY{n}{I}\PY{p}{)}
\PY{n}{minus\PYZus{}one\PYZus{}coerced} \PY{o}{=} \PY{n}{ZZ}\PY{p}{(}\PY{n}{e}\PY{o}{\PYZca{}}\PY{p}{(}\PY{n}{pi}\PY{o}{*}\PY{n}{I}\PY{p}{)}\PY{p}{)}  \PY{c+c1}{\PYZsh{} coercion}
\PY{n+nb}{print}\PY{p}{(}\PY{n}{parent}\PY{p}{(}\PY{n}{minus\PYZus{}one}\PY{p}{)}\PY{p}{)}
\PY{n+nb}{print}\PY{p}{(}\PY{n}{parent}\PY{p}{(}\PY{n}{minus\PYZus{}one\PYZus{}coerced}\PY{p}{)}\PY{p}{)}
\end{Verbatim}
\end{tcolorbox}

    \begin{Verbatim}[commandchars=\\\{\}]
Symbolic Ring
Integer Ring
    \end{Verbatim}

    \textbf{Remark.} Notice that there is a fundamental difference between
the rings \texttt{RR} and \texttt{CC} and all the others in the table
above: the real and complex numbers are \emph{approximated}.

    \begin{tcolorbox}[breakable, size=fbox, boxrule=1pt, pad at break*=1mm,colback=cellbackground, colframe=cellborder]
\prompt{In}{incolor}{1}{\boxspacing}
\begin{Verbatim}[commandchars=\\\{\}]
\PY{n+nb}{print}\PY{p}{(}\PY{n}{QQ}\PY{p}{(}\PY{l+m+mi}{3}\PY{p}{)}\PY{p}{)}
\PY{n+nb}{print}\PY{p}{(}\PY{n}{RR}\PY{p}{(}\PY{l+m+mi}{3}\PY{p}{)}\PY{p}{)}
\end{Verbatim}
\end{tcolorbox}

    \begin{Verbatim}[commandchars=\\\{\}]
3
3.00000000000000
    \end{Verbatim}

    You can also choose the precision of this approximation using the
alternative name \texttt{RealField}.

    \begin{tcolorbox}[breakable, size=fbox, boxrule=1pt, pad at break*=1mm,colback=cellbackground, colframe=cellborder]
\prompt{In}{incolor}{4}{\boxspacing}
\begin{Verbatim}[commandchars=\\\{\}]
\PY{n+nb}{print}\PY{p}{(}\PY{n}{RR}\PY{p}{)}
\PY{n+nb}{print}\PY{p}{(}\PY{n}{RealField}\PY{p}{(}\PY{n}{prec}\PY{o}{=}\PY{l+m+mi}{1000}\PY{p}{)}\PY{p}{)}
\end{Verbatim}
\end{tcolorbox}

    \begin{Verbatim}[commandchars=\\\{\}]
Real Field with 53 bits of precision
Real Field with 1000 bits of precision
    \end{Verbatim}

    \hypertarget{polynomial-rings}{%
\section{Polynomial rings}\label{polynomial-rings}}

\textbf{Reference:}
{[}\href{https://doc.sagemath.org/html/en/reference/polynomial_rings/index.html}{7}{]}

If you want to work with polynomials over a certain ring it is better to
use this specific construction, rather than the symbolic expressions
introduced above.

    \begin{tcolorbox}[breakable, size=fbox, boxrule=1pt, pad at break*=1mm,colback=cellbackground, colframe=cellborder]
\prompt{In}{incolor}{5}{\boxspacing}
\begin{Verbatim}[commandchars=\\\{\}]
\PY{n}{polring}\PY{o}{.}\PY{o}{\PYZlt{}}\PY{n}{x}\PY{p}{,}\PY{n}{y}\PY{p}{,}\PY{n}{z}\PY{o}{\PYZgt{}} \PY{o}{=} \PY{n}{RR}\PY{p}{[}\PY{p}{]}  \PY{c+c1}{\PYZsh{} Alternative: polring.\PYZlt{}x,y,z\PYZgt{} = PolynomialRing(RR)}
\PY{n}{polring}
\end{Verbatim}
\end{tcolorbox}

            \begin{tcolorbox}[breakable, size=fbox, boxrule=.5pt, pad at break*=1mm, opacityfill=0]
\prompt{Out}{outcolor}{5}{\boxspacing}
\begin{Verbatim}[commandchars=\\\{\}]
Multivariate Polynomial Ring in x, y, z over Real Field with 53 bits of
precision
\end{Verbatim}
\end{tcolorbox}
        
    You can use as many variables as you like, and you can replace
\texttt{RR} with any ring. In the example above \texttt{polring} is just
the name of the variable (in the computer science sense) associated with
this polynomial ring.

\hypertarget{operations-on-polynomials}.
There is also the single-slash division \texttt{/}, but the result may
not be a polynomial anymore.

\textbf{Exercise:} use the \texttt{parent()} command to find out what
the quotient of two polynomials is.

\textbf{Question:} what happens if you remove the first line in the cell
below? What if we used the variable \texttt{y} instead of \texttt{x}?

    \begin{tcolorbox}[breakable, size=fbox, boxrule=1pt, pad at break*=1mm,colback=cellbackground, colframe=cellborder]
\prompt{In}{incolor}{6}{\boxspacing}
\begin{Verbatim}[commandchars=\\\{\}]
\PY{n}{polring}\PY{o}{.}\PY{o}{\PYZlt{}}\PY{n}{x}\PY{o}{\PYZgt{}} \PY{o}{=} \PY{n}{QQ}\PY{p}{[}\PY{p}{]}
\PY{n}{p} \PY{o}{=} \PY{n}{x}\PY{o}{\PYZca{}}\PY{l+m+mi}{2} \PY{o}{+} \PY{l+m+mi}{2}\PY{o}{*}\PY{n}{x} \PY{o}{\PYZhy{}} \PY{l+m+mi}{3}   \PY{c+c1}{\PYZsh{} Don\PYZsq{}t forget * for multiplication!}
\PY{n}{q} \PY{o}{=} \PY{n}{p} \PY{o}{/}\PY{o}{/} \PY{p}{(}\PY{n}{x}\PY{o}{+}\PY{l+m+mi}{1}\PY{p}{)}
\PY{n}{r} \PY{o}{=} \PY{n}{p} \PY{o}{\PYZpc{}}  \PY{p}{(}\PY{n}{x}\PY{o}{+}\PY{l+m+mi}{1}\PY{p}{)}
\PY{n}{f} \PY{o}{=} \PY{n}{p} \PY{o}{/}  \PY{p}{(}\PY{n}{x}\PY{o}{+}\PY{l+m+mi}{1}\PY{p}{)}
\PY{n+nb}{print}\PY{p}{(}\PY{n}{q}\PY{p}{)}
\PY{n+nb}{print}\PY{p}{(}\PY{n}{r}\PY{p}{)}
\PY{n+nb}{print}\PY{p}{(}\PY{n}{f}\PY{p}{)}
\end{Verbatim}
\end{tcolorbox}

    \begin{Verbatim}[commandchars=\\\{\}]
x + 1
-4
(x\^{}2 + 2*x - 3)/(x + 1)
    \end{Verbatim}

    You can do more complex operations. Try out \texttt{roots()} and
\texttt{factor} in the cell below.

\textbf{Remark.} Notice how the result can change substantially if you
change the base ring.

\textbf{Remark.}
\href{https://doc.sagemath.org/html/en/reference/structure/sage/structure/factorization.html}{Factorizations}
are a particular object in Sage. They are kinda like a list, but not
really. You can get a list of pairs (factor, power) with
\texttt{list(factor(f))}.

    \begin{tcolorbox}[breakable, size=fbox, boxrule=1pt, pad at break*=1mm,colback=cellbackground, colframe=cellborder]
\prompt{In}{incolor}{7}{\boxspacing}
\begin{Verbatim}[commandchars=\\\{\}]
\PY{n}{polring\PYZus{}onevar}\PY{o}{.}\PY{o}{\PYZlt{}}\PY{n}{t}\PY{o}{\PYZgt{}} \PY{o}{=} \PY{n}{QQ}\PY{p}{[}\PY{p}{]}

\PY{n}{f} \PY{o}{=} \PY{n}{t}\PY{o}{\PYZca{}}\PY{l+m+mi}{5} \PY{o}{+} \PY{n}{t}\PY{o}{\PYZca{}}\PY{l+m+mi}{4} \PY{o}{\PYZhy{}} \PY{l+m+mi}{2}\PY{o}{*}\PY{n}{t}\PY{o}{\PYZca{}}\PY{l+m+mi}{3} \PY{o}{\PYZhy{}} \PY{l+m+mi}{2}\PY{o}{*}\PY{n}{t}\PY{o}{\PYZca{}}\PY{l+m+mi}{2} \PY{o}{\PYZhy{}} \PY{l+m+mi}{3}\PY{o}{*}\PY{n}{t} \PY{o}{\PYZhy{}} \PY{l+m+mi}{3}
\PY{n+nb}{print}\PY{p}{(}\PY{n}{factor}\PY{p}{(}\PY{n}{f}\PY{p}{)}\PY{p}{)}
\PY{n+nb}{print}\PY{p}{(}\PY{n}{f}\PY{o}{.}\PY{n}{roots}\PY{p}{(}\PY{p}{)}\PY{p}{)}  \PY{c+c1}{\PYZsh{} Result: list of pairs (root,multiplicity)}

\PY{n}{polring\PYZus{}manyvar}\PY{o}{.}\PY{o}{\PYZlt{}}\PY{n}{x}\PY{p}{,}\PY{n}{y}\PY{p}{,}\PY{n}{z}\PY{o}{\PYZgt{}} \PY{o}{=} \PY{n}{QQ}\PY{p}{[}\PY{p}{]}
\PY{n}{factor}\PY{p}{(}\PY{n}{x}\PY{o}{*}\PY{n}{y}\PY{o}{+}\PY{n}{x}\PY{p}{)}

\PY{c+c1}{\PYZsh{} The following line gives an error, because the polynomial}
\PY{c+c1}{\PYZsh{} is understood to possibly have many variables:}
\PY{c+c1}{\PYZsh{}(x\PYZca{}2\PYZhy{}1).roots()}
\end{Verbatim}
\end{tcolorbox}

    \begin{Verbatim}[commandchars=\\\{\}]
(t + 1) * (t\^{}2 - 3) * (t\^{}2 + 1)
[(-1, 1)]
    \end{Verbatim}

            \begin{tcolorbox}[breakable, size=fbox, boxrule=.5pt, pad at break*=1mm, opacityfill=0]
\prompt{Out}{outcolor}{7}{\boxspacing}
\begin{Verbatim}[commandchars=\\\{\}]
(y + 1) * x
\end{Verbatim}
\end{tcolorbox}
        
    \hypertarget{matrices-and-vectors}{%
\section{Matrices and vectors}\label{matrices-and-vectors}}

\textbf{References:}
{[}\href{https://doc.sagemath.org/html/en/reference/matrices/index.html}{8}{]},
but in particular the subections
{[}\href{https://doc.sagemath.org/html/en/reference/matrices/sage/matrix/docs.html}{9}{]}
and
{[}\href{https://doc.sagemath.org/html/en/reference/matrices/sage/matrix/matrix2.html}{10}{]}

In Sage you can easily manipulate matrices and vectors

    \begin{tcolorbox}[breakable, size=fbox, boxrule=1pt, pad at break*=1mm,colback=cellbackground, colframe=cellborder]
\prompt{In}{incolor}{77}{\boxspacing}
\begin{Verbatim}[commandchars=\\\{\}]
\PY{n}{A} \PY{o}{=} \PY{n}{matrix}\PY{p}{(}\PY{p}{[}\PY{p}{[}\PY{l+m+mi}{1}\PY{p}{,}\PY{l+m+mi}{2}\PY{p}{,}\PY{l+m+mi}{3}\PY{p}{]}\PY{p}{,}\PY{p}{[}\PY{l+m+mi}{0}\PY{p}{,}\PY{l+m+mi}{0}\PY{p}{,}\PY{l+m+mi}{1}\PY{p}{]}\PY{p}{,}\PY{p}{[}\PY{l+m+mi}{4}\PY{p}{,}\PY{o}{\PYZhy{}}\PY{l+m+mi}{3}\PY{p}{,}\PY{l+m+mi}{22}\PY{o}{/}\PY{l+m+mi}{7}\PY{p}{]}\PY{p}{]}\PY{p}{)}
\PY{n}{B} \PY{o}{=} \PY{n}{matrix}\PY{p}{(}\PY{p}{[}\PY{p}{[}\PY{l+m+mi}{1}\PY{o}{/}\PY{l+m+mi}{2}\PY{p}{,}\PY{l+m+mi}{0}\PY{p}{,}\PY{l+m+mi}{0}\PY{p}{]}\PY{p}{,}\PY{p}{[}\PY{l+m+mi}{7}\PY{p}{,}\PY{l+m+mi}{0}\PY{p}{,}\PY{l+m+mi}{0}\PY{p}{]}\PY{p}{,}\PY{p}{[}\PY{l+m+mi}{1}\PY{p}{,}\PY{l+m+mi}{1}\PY{p}{,}\PY{l+m+mi}{1}\PY{p}{]}\PY{p}{]}\PY{p}{)}
\PY{n}{v} \PY{o}{=} \PY{n}{vector}\PY{p}{(}\PY{p}{[}\PY{l+m+mi}{3}\PY{p}{,}\PY{l+m+mi}{4}\PY{p}{,}\PY{o}{\PYZhy{}}\PY{l+m+mi}{1}\PY{p}{]}\PY{p}{)}

\PY{n+nb}{print}\PY{p}{(}\PY{n}{A}\PY{p}{,} \PY{l+s+s2}{\PYZdq{}}\PY{l+s+se}{\PYZbs{}n}\PY{l+s+s2}{\PYZdq{}}\PY{p}{)}  \PY{c+c1}{\PYZsh{} \PYZbs{}n just means \PYZdq{}newline\PYZdq{}}
\PY{n+nb}{print}\PY{p}{(}\PY{n}{B}\PY{p}{,} \PY{l+s+s2}{\PYZdq{}}\PY{l+s+se}{\PYZbs{}n}\PY{l+s+s2}{\PYZdq{}}\PY{p}{)}
\PY{n+nb}{print}\PY{p}{(}\PY{n}{B}\PY{o}{*}\PY{n}{v}\PY{p}{,} \PY{l+s+s2}{\PYZdq{}}\PY{l+s+se}{\PYZbs{}n}\PY{l+s+s2}{\PYZdq{}}\PY{p}{)}
\PY{n+nb}{print}\PY{p}{(}\PY{n}{A}\PY{o}{\PYZca{}}\PY{l+m+mi}{2} \PY{o}{+} \PY{l+m+mi}{2}\PY{o}{*}\PY{n}{B} \PY{o}{\PYZhy{}} \PY{n}{A}\PY{o}{*}\PY{n}{B}\PY{p}{,} \PY{l+s+s2}{\PYZdq{}}\PY{l+s+se}{\PYZbs{}n}\PY{l+s+s2}{\PYZdq{}}\PY{p}{)}

\PY{n+nb}{print}\PY{p}{(}\PY{l+s+s2}{\PYZdq{}}\PY{l+s+s2}{Rank of A =}\PY{l+s+s2}{\PYZdq{}}\PY{p}{,} \PY{n}{rank}\PY{p}{(}\PY{n}{A}\PY{p}{)}\PY{p}{)} \PY{c+c1}{\PYZsh{} You can also use A.rank()}
\PY{n+nb}{print}\PY{p}{(}\PY{l+s+s2}{\PYZdq{}}\PY{l+s+s2}{Rank of B =}\PY{l+s+s2}{\PYZdq{}}\PY{p}{,} \PY{n}{rank}\PY{p}{(}\PY{n}{B}\PY{p}{)}\PY{p}{)}
\end{Verbatim}
\end{tcolorbox}

    \begin{Verbatim}[commandchars=\\\{\}]
[   1    2    3]
[   0    0    1]
[   4   -3 22/7]

[1/2   0   0]
[  7   0   0]
[  1   1   1]

(3/2, 21, 6)

[  -7/2    -10   80/7]
[    17     -4   15/7]
[ 241/7  -18/7 869/49]

Rank of A = 3
Rank of B = 2
    \end{Verbatim}

    \textbf{Exercise:} in the cell above, compute the determinant, inverse
and characteristic polynomial of the matrix \texttt{A}. \emph{Hint: look
at the reference {[}10{]} above (the functions are listed in alphabetic
order).}

As for polynomials, you can specify where a matrix or a vector lives

    \begin{tcolorbox}[breakable, size=fbox, boxrule=1pt, pad at break*=1mm,colback=cellbackground, colframe=cellborder]
\prompt{In}{incolor}{57}{\boxspacing}
\begin{Verbatim}[commandchars=\\\{\}]
\PY{n}{M} \PY{o}{=} \PY{n}{matrix}\PY{p}{(}\PY{n}{CC}\PY{p}{,} \PY{p}{[}\PY{p}{[}\PY{l+m+mi}{0}\PY{p}{,}\PY{l+m+mi}{1}\PY{p}{]}\PY{p}{,}\PY{p}{[}\PY{l+m+mi}{1}\PY{p}{,}\PY{l+m+mi}{0}\PY{p}{]}\PY{p}{]}\PY{p}{)}
\PY{n}{parent}\PY{p}{(}\PY{n}{M}\PY{p}{)}
\end{Verbatim}
\end{tcolorbox}

            \begin{tcolorbox}[breakable, size=fbox, boxrule=.5pt, pad at break*=1mm, opacityfill=0]
\prompt{Out}{outcolor}{57}{\boxspacing}
\begin{Verbatim}[commandchars=\\\{\}]
Full MatrixSpace of 2 by 2 dense matrices over Complex Field with 53 bits of
precision
\end{Verbatim}
\end{tcolorbox}
        
    You can also solve linear systems and compute eigenvalues and
eigenvectors of a matrix

\textbf{Warning.} In linear algebra there are distinct concepts of
\emph{left} and \emph{right} eigenvalues (and eigenvector). The one you
know is probably that of \textbf{right} eigen-\{value,vector\}, that is
an element \(\lambda\) of the base field and a non-zero vector
\(\mathbf v\) with \(A\mathbf v=\lambda\mathbf v\). The other concept
corresponds to the equality \(\mathbf v^TA=\lambda \mathbf v\).

    \begin{tcolorbox}[breakable, size=fbox, boxrule=1pt, pad at break*=1mm,colback=cellbackground, colframe=cellborder]
\prompt{In}{incolor}{60}{\boxspacing}
\begin{Verbatim}[commandchars=\\\{\}]
\PY{n}{A} \PY{o}{=} \PY{n}{Matrix}\PY{p}{(}\PY{n}{RR}\PY{p}{,} \PY{p}{[}\PY{p}{[}\PY{n}{sqrt}\PY{p}{(}\PY{l+m+mi}{59}\PY{p}{)}\PY{p}{,}\PY{l+m+mi}{32}\PY{p}{]}\PY{p}{,}\PY{p}{[}\PY{o}{\PYZhy{}}\PY{l+m+mi}{1}\PY{o}{/}\PY{l+m+mi}{4}\PY{p}{,}\PY{l+m+mi}{3}\PY{p}{]}\PY{p}{]}\PY{p}{)}
\PY{n}{v} \PY{o}{=} \PY{n}{vector}\PY{p}{(}\PY{n}{RR}\PY{p}{,} \PY{p}{[}\PY{l+m+mi}{3}\PY{p}{,}\PY{l+m+mi}{0}\PY{p}{]}\PY{p}{)}
\PY{n}{A}\PY{o}{.}\PY{n}{solve\PYZus{}right}\PY{p}{(}\PY{n}{v}\PY{p}{)} \PY{c+c1}{\PYZsh{} Solve Ax=v. Alternative: A \PYZbs{} v}
\end{Verbatim}
\end{tcolorbox}

            \begin{tcolorbox}[breakable, size=fbox, boxrule=.5pt, pad at break*=1mm, opacityfill=0]
\prompt{Out}{outcolor}{60}{\boxspacing}
\begin{Verbatim}[commandchars=\\\{\}]
(0.289916349448506, 0.0241596957873755)
\end{Verbatim}
\end{tcolorbox}
        
    \begin{tcolorbox}[breakable, size=fbox, boxrule=1pt, pad at break*=1mm,colback=cellbackground, colframe=cellborder]
\prompt{In}{incolor}{64}{\boxspacing}
\begin{Verbatim}[commandchars=\\\{\}]
\PY{n}{A} \PY{o}{=} \PY{n}{Matrix}\PY{p}{(}\PY{n}{QQ}\PY{p}{,} \PY{p}{[}\PY{p}{[}\PY{l+m+mi}{1}\PY{p}{,}\PY{l+m+mi}{2}\PY{p}{]}\PY{p}{,}\PY{p}{[}\PY{l+m+mi}{3}\PY{p}{,}\PY{l+m+mi}{4}\PY{p}{]}\PY{p}{]}\PY{p}{)}
\PY{n}{A}\PY{o}{.}\PY{n}{eigenspaces\PYZus{}right}\PY{p}{(}\PY{p}{)} \PY{c+c1}{\PYZsh{} Also: A.eigenvalues(), A.eigenvectors\PYZus{}right()}
\end{Verbatim}
\end{tcolorbox}

            \begin{tcolorbox}[breakable, size=fbox, boxrule=.5pt, pad at break*=1mm, opacityfill=0]
\prompt{Out}{outcolor}{64}{\boxspacing}
\begin{Verbatim}[commandchars=\\\{\}]
[
(-0.3722813232690144?, Vector space of degree 2 and dimension 1 over Algebraic
Field
User basis matrix:
[                   1 -0.6861406616345072?]),
(5.372281323269015?, Vector space of degree 2 and dimension 1 over Algebraic
Field
User basis matrix:
[                 1 2.186140661634508?])
]
\end{Verbatim}
\end{tcolorbox}
        
    We can also extract a specific submatrix by selecting only some rows and
columns, with a syntax similar to that of Python's lists. Check out more
examples in the reference {[}9{]} above, and try them in the cell below.

    \begin{tcolorbox}[breakable, size=fbox, boxrule=1pt, pad at break*=1mm,colback=cellbackground, colframe=cellborder]
\prompt{In}{incolor}{94}{\boxspacing}
\begin{Verbatim}[commandchars=\\\{\}]
\PY{n}{A} \PY{o}{=} \PY{n}{MatrixSpace}\PY{p}{(}\PY{n}{ZZ}\PY{p}{,} \PY{l+m+mi}{7}\PY{p}{)}\PY{o}{.}\PY{n}{random\PYZus{}element}\PY{p}{(}\PY{p}{)}
\PY{n+nb}{print}\PY{p}{(}\PY{n}{A}\PY{p}{,} \PY{l+s+s2}{\PYZdq{}}\PY{l+s+se}{\PYZbs{}n}\PY{l+s+s2}{\PYZdq{}}\PY{p}{)}
\PY{n+nb}{print}\PY{p}{(}\PY{n}{A}\PY{p}{[}\PY{l+m+mi}{1}\PY{p}{:}\PY{l+m+mi}{3}\PY{p}{,}\PY{l+m+mi}{2}\PY{p}{:}\PY{l+m+mi}{5}\PY{p}{]}\PY{p}{,} \PY{l+s+s2}{\PYZdq{}}\PY{l+s+se}{\PYZbs{}n}\PY{l+s+s2}{\PYZdq{}}\PY{p}{)} \PY{c+c1}{\PYZsh{} Rows from 1 to 3, columns from 2 to 5}
\PY{n+nb}{print}\PY{p}{(}\PY{n}{A}\PY{p}{[}\PY{l+m+mi}{0}\PY{p}{,}\PY{l+m+mi}{0}\PY{p}{:}\PY{p}{]}\PY{p}{,} \PY{l+s+s2}{\PYZdq{}}\PY{l+s+se}{\PYZbs{}n}\PY{l+s+s2}{\PYZdq{}}\PY{p}{)}    \PY{c+c1}{\PYZsh{} First row, all columns}
\PY{n+nb}{print}\PY{p}{(}\PY{n}{A}\PY{p}{[}\PY{p}{[}\PY{l+m+mi}{0}\PY{p}{,}\PY{l+m+mi}{5}\PY{p}{,}\PY{l+m+mi}{2}\PY{p}{]}\PY{p}{,}\PY{l+m+mi}{0}\PY{p}{:}\PY{l+m+mi}{5}\PY{p}{]}\PY{p}{)}   \PY{c+c1}{\PYZsh{} Rows 0, 5 and 2 (in this order) and columns 0 to 5}
\end{Verbatim}
\end{tcolorbox}

    \begin{Verbatim}[commandchars=\\\{\}]
[-14   2   0  -1   1  -2  -1]
[  0  -8   0   9  -2  11   1]
[  0   3   1  -1   1   1 221]
[ -1   2   1 -25 -10   4   0]
[ -3   0   0   2  16  -1  -2]
[  1  -3   3 -41   1   0   0]
[ -2   1   0   0  -6   2  12]

[ 0  9 -2]
[ 1 -1  1]

[-14   2   0  -1   1  -2  -1]

[-14   2   0  -1   1]
[  1  -3   3 -41   1]
[  0   3   1  -1   1]
    \end{Verbatim}

    \textbf{Exercise:} write a sage function that computes the determinant
of an \(n\times n\) matrix \(A=(a_{ij})\) using Laplace's rule by the
first row, that is \begin{align*}
    \operatorname{det}A = \sum_{j=1}^n (-1)^ja_{0j}M_{0j}
\end{align*} where \(M_{0j}\) is the determinant of the
\((n-1)\times(n-1)\) matrix obtained by removing the \(0\)-th row and
the \(j\)-th column from \(A\).

    \begin{tcolorbox}[breakable, size=fbox, boxrule=1pt, pad at break*=1mm,colback=cellbackground, colframe=cellborder]
\prompt{In}{incolor}{91}{\boxspacing}
\begin{Verbatim}[commandchars=\\\{\}]
\PY{k}{def} \PY{n+nf}{my\PYZus{}det}\PY{p}{(}\PY{n}{A}\PY{p}{)}\PY{p}{:}
    \PY{k}{if} \PY{o+ow}{not} \PY{n}{A}\PY{o}{.}\PY{n}{is\PYZus{}square}\PY{p}{(}\PY{p}{)}\PY{p}{:}
        \PY{n+nb}{print}\PY{p}{(}\PY{l+s+s2}{\PYZdq{}}\PY{l+s+s2}{Error: matrix is not square}\PY{l+s+s2}{\PYZdq{}}\PY{p}{)}
    
    \PY{n}{n} \PY{o}{=} \PY{n}{A}\PY{o}{.}\PY{n}{nrows}\PY{p}{(}\PY{p}{)}  \PY{c+c1}{\PYZsh{} size of the matrix}
    
    \PY{c+c1}{\PYZsh{} Continue from here!}
\end{Verbatim}
\end{tcolorbox}

    \hypertarget{number-theory}{%
\section{Number Theory}\label{number-theory}}

\textbf{Reference:}
{[}\href{https://doc.sagemath.org/html/en/reference/rings_standard/sage/rings/integer.html}{11}{]}

Sage includes a large library of functions for computing with the
integers, see the link above.

    \begin{tcolorbox}[breakable, size=fbox, boxrule=1pt, pad at break*=1mm,colback=cellbackground, colframe=cellborder]
\prompt{In}{incolor}{8}{\boxspacing}
\begin{Verbatim}[commandchars=\\\{\}]
\PY{n}{n} \PY{o}{=} \PY{l+m+mi}{123456789}
\PY{n}{m} \PY{o}{=} \PY{l+m+mi}{987654321}
\PY{n}{p} \PY{o}{=} \PY{l+m+mi}{3607}

\PY{n+nb}{print}\PY{p}{(}\PY{n}{factor}\PY{p}{(}\PY{n}{n}\PY{p}{)}\PY{p}{)}
\PY{n+nb}{print}\PY{p}{(}\PY{n}{is\PYZus{}prime}\PY{p}{(}\PY{n}{p}\PY{p}{)}\PY{p}{)}
\PY{n+nb}{print}\PY{p}{(}\PY{n}{p}\PY{o}{.}\PY{n}{divides}\PY{p}{(}\PY{n}{n}\PY{p}{)}\PY{p}{)}
\PY{n+nb}{print}\PY{p}{(}\PY{n}{euler\PYZus{}phi}\PY{p}{(}\PY{n}{m}\PY{p}{)}\PY{p}{)}
\PY{n+nb}{print}\PY{p}{(}\PY{n}{gcd}\PY{p}{(}\PY{n}{n}\PY{p}{,} \PY{n}{m}\PY{p}{)}\PY{p}{)}
\PY{n+nb}{print}\PY{p}{(}\PY{n}{lcm}\PY{p}{(}\PY{n}{n}\PY{p}{,} \PY{n}{m}\PY{p}{)}\PY{p}{)}
\end{Verbatim}
\end{tcolorbox}

    \begin{Verbatim}[commandchars=\\\{\}]
3\^{}2 * 3607 * 3803
True
True
619703040
9
13548070123626141
    \end{Verbatim}

    \hypertarget{primes}{%
\subsection{Primes}\label{primes}}

\textbf{Reference:}
{[}\href{https://doc.sagemath.org/html/en/reference/sets/sage/sets/primes.html}{12}{]}

The set of prime numbers is called \texttt{Primes()}. It is like an
infinite list: for example you can get the one-millionth prime number or
you can use this list to create other lists. You can also check what the
first prime number larger than a given number is.

    \begin{tcolorbox}[breakable, size=fbox, boxrule=1pt, pad at break*=1mm,colback=cellbackground, colframe=cellborder]
\prompt{In}{incolor}{9}{\boxspacing}
\begin{Verbatim}[commandchars=\\\{\}]
\PY{n}{PP} \PY{o}{=} \PY{n}{Primes}\PY{p}{(}\PY{p}{)}
\PY{n+nb}{print}\PY{p}{(}\PY{n}{PP}\PY{p}{)}
\PY{n+nb}{print}\PY{p}{(}\PY{n}{PP}\PY{p}{[}\PY{l+m+mi}{10}\PY{p}{]}\PY{p}{,} \PY{n}{PP}\PY{p}{[}\PY{l+m+mi}{10}\PY{o}{\PYZca{}}\PY{l+m+mi}{6}\PY{p}{]}\PY{p}{)}
\PY{n+nb}{print}\PY{p}{(}\PY{n}{PP}\PY{o}{.}\PY{n}{next}\PY{p}{(}\PY{l+m+mi}{44}\PY{p}{)}\PY{p}{)}

\PY{n}{First\PYZus{}Thousand\PYZus{}Primes} \PY{o}{=} \PY{n}{PP}\PY{p}{[}\PY{l+m+mi}{0}\PY{p}{:}\PY{l+m+mi}{1000}\PY{p}{]}
\PY{n+nb}{print}\PY{p}{(}\PY{p}{[}\PY{n}{p} \PY{k}{for} \PY{n}{p} \PY{o+ow}{in} \PY{n}{First\PYZus{}Thousand\PYZus{}Primes} \PY{k}{if} \PY{n}{p} \PY{o}{\PYZlt{}} \PY{l+m+mi}{100} \PY{o+ow}{and} \PY{n}{p} \PY{o}{\PYZgt{}} \PY{l+m+mi}{75}\PY{p}{]}\PY{p}{)}
\end{Verbatim}
\end{tcolorbox}

    \begin{Verbatim}[commandchars=\\\{\}]
Set of all prime numbers: 2, 3, 5, 7, {\ldots}
31 15485867
47
[79, 83, 89, 97]
    \end{Verbatim}

    \hypertarget{the-chinese-remainder-theorem-crt}{%
\subsection{The Chinese remainder theorem
(CRT)}\label{the-chinese-remainder-theorem-crt}}

We say that two integers \(a\) and \(b\) are \emph{congruent} modulo
another integer \(n>0\) if they have the same remainder when divided by
\(n\). We denote this by \(a\equiv b\pmod n\), or in Python/Sage syntax
\texttt{a\ \%\ n\ ==\ b\ \%\ n}.

The Chinese remainder theorem states that if \(a,b\in\mathbb Z\) and
\(n,m\in \mathbb Z_{>0}\) are such that \(\gcd(n,m)=1\) then the system
of congruences

\begin{align*}
\begin{cases}
    x \equiv a \pmod n\\
    x \equiv b \pmod m
\end{cases}
\end{align*}

has exactly one solution modulo \(mn\). This means that there is one and
only one number \(x\) with \(0\leq x<mn\) such that \(x\equiv a\pmod n\)
and \(x\equiv b\pmod m\).

The procedure to find such a number is not too hard to describe (you
might see it in an algebra or number theory course), but it can be a bit
long. Luckily, Sage can do this for you:

    \begin{tcolorbox}[breakable, size=fbox, boxrule=1pt, pad at break*=1mm,colback=cellbackground, colframe=cellborder]
\prompt{In}{incolor}{10}{\boxspacing}
\begin{Verbatim}[commandchars=\\\{\}]
\PY{n}{a} \PY{o}{=} \PY{l+m+mi}{2}
\PY{n}{b} \PY{o}{=} \PY{o}{\PYZhy{}}\PY{l+m+mi}{1}
\PY{n}{n} \PY{o}{=} \PY{l+m+mi}{172}
\PY{n}{m} \PY{o}{=} \PY{l+m+mi}{799}

\PY{k}{if} \PY{n}{gcd}\PY{p}{(}\PY{n}{n}\PY{p}{,}\PY{n}{m}\PY{p}{)} \PY{o}{!=} \PY{l+m+mi}{1}\PY{p}{:}
    \PY{n+nb}{print}\PY{p}{(}\PY{l+s+s2}{\PYZdq{}}\PY{l+s+s2}{The numbers are not comprime, I can}\PY{l+s+s2}{\PYZsq{}}\PY{l+s+s2}{t solve this!}\PY{l+s+s2}{\PYZdq{}}\PY{p}{)}
\PY{k}{else}\PY{p}{:}
    \PY{n}{x} \PY{o}{=} \PY{n}{crt}\PY{p}{(}\PY{n}{a}\PY{p}{,} \PY{n}{b}\PY{p}{,} \PY{n}{n}\PY{p}{,} \PY{n}{m}\PY{p}{)}
    \PY{n+nb}{print}\PY{p}{(}\PY{n}{x}\PY{p}{,} \PY{n}{x}\PY{o}{\PYZpc{}}\PY{k}{n}, x\PYZpc{}m)
\end{Verbatim}
\end{tcolorbox}

    \begin{Verbatim}[commandchars=\\\{\}]
74306 2 798
    \end{Verbatim}

    \textbf{Exercise.} There is a more general version of the Chinese
remainder theorem which says that if
\(a_0, a_1, \dots, a_k\in\mathbb Z\) and
\(n_0, n_2, \dots, n_k\in\mathbb Z_{>0}\) are such that
\(\gcd(n_i, n_j)=1\) for \(i\neq j\), then the system of congruences

\begin{align*}
\begin{cases}
    x \equiv a_0 \pmod {n_0}\\
    x \equiv a_1 \pmod {n_1}\\
    \dots \\
    x \equiv a_k \pmod {n_k}
\end{cases}
\end{align*}

has exactly one solution modulo \(\prod_{i=0}^kn_i\). Use the
\texttt{crt()} function to find a solution to such a system. *Hint:
start by running the command \texttt{help(crt)}.

    \begin{tcolorbox}[breakable, size=fbox, boxrule=1pt, pad at break*=1mm,colback=cellbackground, colframe=cellborder]
\prompt{In}{incolor}{127}{\boxspacing}
\begin{Verbatim}[commandchars=\\\{\}]
\PY{c+c1}{\PYZsh{}help(crt)}
\end{Verbatim}
\end{tcolorbox}

    \hypertarget{cryptography-rsa}{%
\section{Cryptography: RSA}\label{cryptography-rsa}}

\href{https://en.wikipedia.org/wiki/Cryptography}{Cryptography} is the
discipline that studies methods to communicate secrets in such a way
that any unauthorized listener would not be able to understand the
message.

A simple cryptographic protocol could be changing every letter of your
text following a fixed scheme (or \emph{cypher}), for example by turning
every A into a B, every B into a C and so on. However this is not a very
secure method, for many reasons. One of them is that at some point the
people who want to communicate need to agree on what method to use, and
anyone listening to that conversation would be able to decypher every
subsequent conversation. A public-key cryptographic protocol solves this
problem.

\hypertarget{public-key-cryptography}{%
\subsection{Public-key cryptography}\label{public-key-cryptography}}

Public-key cryptographic protocols, such as RSA, work like this: there
are two keys, a \emph{private} key that is only known to person A
(traditionally called Alice in every example), and a \emph{public} key
that does not need to be secret.

The public key is used to \emph{encrypt} the message (that is to
``lock'' it, or ``hyde'' it), but one needs the private key to
\emph{decrypt} it. Imagine having two keys for your door, but one can
only be used to lock it, while the other only to open it.

The message exchange works like this: suppose that person B (Bob) wants
to send a secret message to Alice. Then Alice secretely generates a
private and a public key and sends only the public one to Bob. Now Bob
encrypts the message and sends it to Alice, who can use her private key
to decrypt it. Even if Eve (short for \emph{eavesdropper}, an
unauthorized listener) listens to every message exchanged, she won't be
able to decypher the secret: the private key has never left Alice's
house!

Notice that such a protocol is \emph{asymmetric}: if Alice wanted to
send a secret to Bob in reply, Bob would need to generate a pair of keys
of his own.

Let's see how we can do this in practice, using number theory!

\hypertarget{rsa}{%
\subsection{RSA}\label{rsa}}

As many other cryptography protocols, RSA is based on a Mathematical
process that is easy to do in one direction, but very hard to invert. In
this case the hard process is integer factorization, that is decomposing
an integer number as a product of primes.

    \begin{tcolorbox}[breakable, size=fbox, boxrule=1pt, pad at break*=1mm,colback=cellbackground, colframe=cellborder]
\prompt{In}{incolor}{2}{\boxspacing}
\begin{Verbatim}[commandchars=\\\{\}]
\PY{n}{p} \PY{o}{=} \PY{l+m+mi}{100003100019100043100057100069}
\PY{n}{q} \PY{o}{=} \PY{l+m+mi}{100144655312449572059845328443}
\PY{n}{n} \PY{o}{=} \PY{n}{p}\PY{o}{*}\PY{n}{q}
\PY{n+nb}{print}\PY{p}{(}\PY{n}{is\PYZus{}prime}\PY{p}{(}\PY{n}{p}\PY{p}{)}\PY{p}{,} \PY{n}{is\PYZus{}prime}\PY{p}{(}\PY{n}{q}\PY{p}{)}\PY{p}{,} \PY{n}{is\PYZus{}prime}\PY{p}{(}\PY{n}{p}\PY{o}{*}\PY{n}{q}\PY{p}{)}\PY{p}{)}

\PY{c+c1}{\PYZsh{} Use the command below to see how long it takes}
\PY{c+c1}{\PYZsh{}timeit(\PYZdq{}factor(n)\PYZdq{}, number=1, repeat=1)}
\end{Verbatim}
\end{tcolorbox}

    \begin{Verbatim}[commandchars=\\\{\}]
True True False
    \end{Verbatim}

    In order to generate the keys, Alice picks a number \(n\) which is the
product of two large primes \(p\) and \(q\) of more or less the same
size. Finding such primes is relatively easy compared to factoring the
number \(n\) she obtained. Then she computes the Euler totient
\(\varphi(n)=(p-1)(q-1)\) of \(n\), which she can do because she knows
that \(n=pq\) - it would be impossible otherwise!

Then Alice can compute two integers \((d,e)\) such that
\(de\equiv 1\pmod{\varphi(n)}\). She will send the numbers \(n\) and
\(d\) to Bob and keep \(e\) secret. In this case the public key is the
pair \((n,d)\), while \(e\) is the private key.

Of course, she does all of this using Sage!

    \begin{tcolorbox}[breakable, size=fbox, boxrule=1pt, pad at break*=1mm,colback=cellbackground, colframe=cellborder]
\prompt{In}{incolor}{105}{\boxspacing}
\begin{Verbatim}[commandchars=\\\{\}]
\PY{k}{def} \PY{n+nf}{two\PYZus{}large\PYZus{}primes}\PY{p}{(}\PY{p}{)}\PY{p}{:}
    \PY{n}{p}\PY{p}{,} \PY{n}{q} \PY{o}{=} \PY{l+m+mi}{0}\PY{p}{,} \PY{l+m+mi}{0}
    \PY{c+c1}{\PYZsh{} We make sure that they are different}
    \PY{k}{while} \PY{n}{p} \PY{o}{==} \PY{n}{q}\PY{p}{:}
        \PY{n}{p} \PY{o}{=} \PY{n}{Primes}\PY{p}{(}\PY{p}{)}\PY{p}{[}\PY{n}{randint}\PY{p}{(}\PY{l+m+mi}{10}\PY{o}{\PYZca{}}\PY{l+m+mi}{6}\PY{p}{,} \PY{l+m+mi}{2}\PY{o}{*}\PY{l+m+mi}{10}\PY{o}{\PYZca{}}\PY{l+m+mi}{6}\PY{p}{)}\PY{p}{]}
        \PY{n}{q} \PY{o}{=} \PY{n}{Primes}\PY{p}{(}\PY{p}{)}\PY{p}{[}\PY{n}{randint}\PY{p}{(}\PY{l+m+mi}{10}\PY{o}{\PYZca{}}\PY{l+m+mi}{6}\PY{p}{,} \PY{l+m+mi}{2}\PY{o}{*}\PY{l+m+mi}{10}\PY{o}{\PYZca{}}\PY{l+m+mi}{6}\PY{p}{)}\PY{p}{]}
    \PY{k}{return} \PY{n}{p}\PY{p}{,} \PY{n}{q}

\PY{k}{def} \PY{n+nf}{random\PYZus{}unit\PYZus{}mod}\PY{p}{(}\PY{n}{N}\PY{p}{)}\PY{p}{:}
    \PY{n}{R} \PY{o}{=} \PY{n}{Integers}\PY{p}{(}\PY{n}{N}\PY{p}{)}
    \PY{n}{d} \PY{o}{=} \PY{n}{R}\PY{p}{(}\PY{l+m+mi}{0}\PY{p}{)}
    \PY{c+c1}{\PYZsh{} We make sure that it is invertible}
    \PY{k}{while} \PY{o+ow}{not} \PY{n}{d}\PY{o}{.}\PY{n}{is\PYZus{}unit}\PY{p}{(}\PY{p}{)}\PY{p}{:}
        \PY{n}{d} \PY{o}{=} \PY{n}{R}\PY{o}{.}\PY{n}{random\PYZus{}element}\PY{p}{(}\PY{p}{)}
    \PY{k}{return} \PY{n}{d}

\PY{k}{def} \PY{n+nf}{Alice\PYZus{}generate\PYZus{}keys}\PY{p}{(}\PY{p}{)}\PY{p}{:}
    \PY{n}{p}\PY{p}{,} \PY{n}{q} \PY{o}{=} \PY{n}{two\PYZus{}large\PYZus{}primes}\PY{p}{(}\PY{p}{)}
    \PY{n}{n} \PY{o}{=} \PY{n}{p}\PY{o}{*}\PY{n}{q}
    \PY{n}{phi\PYZus{}n} \PY{o}{=} \PY{p}{(}\PY{n}{p}\PY{o}{\PYZhy{}}\PY{l+m+mi}{1}\PY{p}{)}\PY{o}{*}\PY{p}{(}\PY{n}{q}\PY{o}{\PYZhy{}}\PY{l+m+mi}{1}\PY{p}{)} \PY{c+c1}{\PYZsh{} euler\PYZus{}phi(n) is slow!}
    
    \PY{n}{d} \PY{o}{=} \PY{n}{random\PYZus{}unit\PYZus{}mod}\PY{p}{(}\PY{n}{phi\PYZus{}n}\PY{p}{)}
    \PY{n}{e} \PY{o}{=} \PY{n}{d}\PY{o}{\PYZca{}}\PY{o}{\PYZhy{}}\PY{l+m+mi}{1}
    \PY{k}{return} \PY{n}{n}\PY{p}{,} \PY{n}{d}\PY{p}{,} \PY{n}{e}

\PY{n}{Alice\PYZus{}generate\PYZus{}keys}\PY{p}{(}\PY{p}{)}
\end{Verbatim}
\end{tcolorbox}

            \begin{tcolorbox}[breakable, size=fbox, boxrule=.5pt, pad at break*=1mm, opacityfill=0]
\prompt{Out}{outcolor}{105}{\boxspacing}
\begin{Verbatim}[commandchars=\\\{\}]
(419199544978969, 235530823946467, 80799425863927)
\end{Verbatim}
\end{tcolorbox}
        
    Now, how does Bob encrypt his message? Let's say he wants to send to
Alice the number \(m\) with \(1<m<n\) (In practice he would like to send
her some text with emojis, or maybe a voice message; but for computers
everything is a number, and there are different ways to translate any
sort of information to a number. He just chooses one of the many
standard methods that already exist, no cryptography is needed in this
step. If the message \(m\) is too long, he can split it up in some
pieces and repeat the process multiple times.)

Now he computes \(m^d\pmod n\) and sends it back to Alice.

    \begin{tcolorbox}[breakable, size=fbox, boxrule=1pt, pad at break*=1mm,colback=cellbackground, colframe=cellborder]
\prompt{In}{incolor}{3}{\boxspacing}
\begin{Verbatim}[commandchars=\\\{\}]
\PY{k}{def} \PY{n+nf}{Bob\PYZus{}encrypt}\PY{p}{(}\PY{n}{m}\PY{p}{,} \PY{n}{n}\PY{p}{,} \PY{n}{d}\PY{p}{)}\PY{p}{:}
    \PY{n}{R} \PY{o}{=} \PY{n}{Integers}\PY{p}{(}\PY{n}{n}\PY{p}{)}
    \PY{k}{return} \PY{n}{R}\PY{p}{(}\PY{n}{m}\PY{p}{)}\PY{o}{\PYZca{}}\PY{n}{d}  \PY{c+c1}{\PYZsh{} Assume that n is large enough}
    
\PY{n}{message} \PY{o}{=} \PY{l+m+mi}{42424242}
\PY{n}{Bob\PYZus{}encrypt}\PY{p}{(}\PY{n}{message}\PY{p}{,} \PY{l+m+mi}{419199544978969}\PY{p}{,} \PY{l+m+mi}{235530823946467}\PY{p}{)}
\end{Verbatim}
\end{tcolorbox}

            \begin{tcolorbox}[breakable, size=fbox, boxrule=.5pt, pad at break*=1mm, opacityfill=0]
\prompt{Out}{outcolor}{3}{\boxspacing}
\begin{Verbatim}[commandchars=\\\{\}]
149461597163501
\end{Verbatim}
\end{tcolorbox}
        
    Since \(de\equiv 1\pmod{\varphi(n)}\), it follows that
\((m^d)^e\equiv m\pmod n\) (see
\href{https://en.wikipedia.org/wiki/Euler\%27s_theorem}{Wikipedia:
Euler's theorem}). So for Alice it is very easy to get back the original
message:

    \begin{tcolorbox}[breakable, size=fbox, boxrule=1pt, pad at break*=1mm,colback=cellbackground, colframe=cellborder]
\prompt{In}{incolor}{108}{\boxspacing}
\begin{Verbatim}[commandchars=\\\{\}]
\PY{k}{def} \PY{n+nf}{Alice\PYZus{}decrypt}\PY{p}{(}\PY{n}{m\PYZus{}encrypted}\PY{p}{,} \PY{n}{n}\PY{p}{,} \PY{n}{e}\PY{p}{)}\PY{p}{:}
    \PY{n}{R} \PY{o}{=} \PY{n}{Integers}\PY{p}{(}\PY{n}{n}\PY{p}{)}
    \PY{k}{return} \PY{n}{R}\PY{p}{(}\PY{n}{m\PYZus{}encrypted}\PY{p}{)}\PY{o}{\PYZca{}}\PY{n}{e}

\PY{n}{Alice\PYZus{}decrypt}\PY{p}{(}\PY{l+m+mi}{149461597163501}\PY{p}{,} \PY{l+m+mi}{419199544978969}\PY{p}{,} \PY{l+m+mi}{80799425863927}\PY{p}{)}
\end{Verbatim}
\end{tcolorbox}

            \begin{tcolorbox}[breakable, size=fbox, boxrule=.5pt, pad at break*=1mm, opacityfill=0]
\prompt{Out}{outcolor}{108}{\boxspacing}
\begin{Verbatim}[commandchars=\\\{\}]
42424242
\end{Verbatim}
\end{tcolorbox}
        
    Another assumption on which RSA relies is that even if one knows
\(M=m^e\) and \(e\), extracting the \(e\)-th root of \(M\) modulo \(n\)
(and thus obtaining \(m\)) is very hard. Currently the best known way to
do this is by factorizing \(n\) first, which is considered to be a very
hard problem. However, there is no proof that faster algorithms can't be
devised.

Moreover, one day we will overcome the current technological
difficulties and quantum computers will be available. Quantum computers
are not just ``more powerful'' than classical hardware, but they work
based on completely different logical foundations and they make the
factorization problem much easier to solve: for example
\href{https://en.wikipedia.org/wiki/Shor\%27s_algorithm}{Shor's
algorithm} takes advantage of this different logic and can factorize
numbers quickly, if run on a quantum computer.

To this day the largest number factorized with a quantum computer is
\(21=3\times 7\). Nonetheless, quantum-safe cryptography protocols
(i.e.~based on problems that are hard to solve also with quantum
computers) have already been developed.


    % Add a bibliography block to the postdoc
    
    
    
\end{document}
