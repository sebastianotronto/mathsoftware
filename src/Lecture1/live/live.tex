\documentclass[10pt,a4paper]{article}
\usepackage[utf8]{inputenc}
\usepackage{amsmath}
\usepackage{amsfonts}
\usepackage{amssymb}
\usepackage{amsthm}
\usepackage[left=2cm,right=2cm,top=2cm,bottom=2cm]{geometry}

\usepackage{enumitem}
\title{Latex Example Live}
\author{Sebastiano Tronto}
\date{20-02-2021}

\newcommand{\reals}{\mathbb{R}}
\DeclareMathOperator{\sinus}{sinus}


\newtheorem{mythm}{My Theorem}[section]

\theoremstyle{definition}
\newtheorem{prop}[mythm]{Proposition}

\theoremstyle{remark}
\newtheorem*{warning}{Achtung}




\begin{document}

\maketitle

\section{Introduction}

Hello, world!

This is a comment

\section{Text}

\textbf{This sentence is in boldface}

\underline{\textit{italicized} maybe in a sentence \textbf{something}}

\underline{\textit{one inside the other}}

\emph{also italicized???}

This is an important sentence, maybe a quote or something, and this \emph{word} is very important. Let's make this sentence longer than one line.

{\Huge Large words}

%\appendix

\section{Math mode}

This is an inline formula \( \sum_i \frac{i}{22} \), it appears within the text

This is a displaystyle formula \[ \sum_{\alpha=0}^{2^{10}}   \frac2 \alpha{22} \] it appears on its own line

How sqrt works: \( \sqrt[\phi]{25} \)



\begin{align}
\label{eq}
e^x &= \left(\sum_{i=0}^\infty \frac{x^i}{i!} \right) = \\
&= \left( 1 + x  + \frac{x^2}2  \right)+ \frac{x^3}{6} + \cdots \nonumber
\end{align}

\[
   \left\{ x \in \reals \quad \text{such that} \quad \frac{ \sinus(x)}{x^2}>0 \right\}\reals
\]

\[ \sum_i \]

The first equation we wrote is \eqref{eq}

\section{Environments}

\subsection{Lists}
\label{subsectionLists}

\begin{itemize}
	\item One \textbf{item}
	\item Another \(2+2=4\)
	\item a third one \[\sum_{i=0}^n\]
	\item A sublist:
		\begin{itemize}
			\item[+] First subitem
			\item[+] and so on
		\end{itemize}
	\item Again in the main list
\end{itemize}

\begin{enumerate}[label=\Roman*]
	\item One
	\item Two
	\item Actually three
	\item Three (or not)
\end{enumerate}

\subsection*{Tables}

Let's write a table:

\vspace{1cm}
\begin{tabular}{r||l|c}
\hline
This is a table & second column & third column \\
\hline
Things          & a             & \( 2+2 = 4 \)\\
\hline
more things     & b             & c
\end{tabular}

\vspace{1cm}
\[
	\left(\begin{array}{cc}
		\int_0^1 e^x  &  \frac{2}{25} \\
		0 & 0 \\
		1111 & 234\alpha
	\end{array}\right)
\]

\[
	\begin{pmatrix}
		\int_0^1 e^x  &  \frac{2}{25} \\
		0 & 0 \\
		1111 & 234\alpha
	\end{pmatrix}
\]

\[
\begin{pmatrix}
1 & 2\\
3 & 4
\end{pmatrix}
\overset{L2\rightarrow L2+L3}\longrightarrow
\begin{pmatrix}
1 & 2\\
4 & 6
\end{pmatrix}
\]

\section{Last section}

In section \ref{subsectionLists} we saw how to write lists

\begin{mythm}[Gauss]
The equation \(2+x=4\) is true for \(x=2\).
\end{mythm}

\begin{prop}
A less important fact
\end{prop}

\setcounter{mythm}{\arabic{mythm}+100}
\begin{mythm}
Another important fact.
\end{mythm}

\begin{warning}
It is a common mistake to think that \(2+2=5\)
\end{warning}
\[\binom45\]

\end{document}